% Preambulo

\documentclass[11pt,a5paper,twoside]{book} % Texto a 11 puntos, tamaño A5, impreso a doble cara, tipo libro.
\usepackage[utf8]{inputenc} % Para escribir con acentos (Español).
\usepackage[T1]{fontenc} % Soporte opcional para la tipografía Linux Libertine. Mejora la tilde en los acentos.
\usepackage[spanish,mexico]{babel} % Idioma (Español México).

% Linux Libertine 

\usepackage[oldstyle]{libertine} % Tipografía Linux Libertine para todo el documento.

% Linux Biolinum

%\usepackage[oldstyle,sfdefault]{libertine} % Tipografía Linux Biolinum para todo el documento. Útil para pantalla (día y noche)

% Variaciones tipográficas (Preambulo):

% oldstyle o osf = Números antigüos
% lining o nf o lf = Números lineales (por defecto)
% proportional o p = Números proporcionales
% tabular o t = Números tabulares (por defecto)
% sb = Activa las seminegritas (libertine, únicamente)
% scale=<numer> o scaled=<number> = Escalado de fuentes (biolinum, unicamente)
% ttscale=<number> o ttscaled=<number> =  Escalado de las monoespaciadas
% osf=true o false = Activa y desactiva los números antiguos.
% defaultfeatures={Variant=01} = Agrega características especiales de fuentes OTF.
 
% Uso local (Texto):

% \tt produce monoespaciadas
% \it produce italicas
% \sc produce versalitas
% \scit produce versalitas-itálicas
% \oldstylenums{1234567890} produce números elzeverianos (Libertine)
% \oldstylenumsf{1234567890} produce números elzeverianos (Biolinum)
% \liningnums{1234567890} usa números lineales (Libertine)
% \liningnumsf{1234567890} usa números lineales (Biolinum)
% \tabularnums{1234567890} usa números tabulares (Libertine)
% \tabularnumsf{1234567890} usa números ltabulares (Biolinum)
% \proportionalnums{1234567890} usa números proporcionales (Libertine)
% \proportionalnumsf{1234567890} usa números proporcionales (Biolinum)
% \sufigures produce superíndices
% \textsu{1} produce superíndices
% \libertine usa Linux Libertine en romanas
% \libertineSB produce seminegritas
% \libertineOsF produce números elzeverianos
% \libertineLF produce números lineales
% \libertineDisplay produce letra para pantalla
% \libmono produce letras monoespaciadas
% \libertineInitial produce letras capitales
% \biolinum usa Biolinum
% \biolinumOsF produce numeros elzeverianos
% \biolinumLF produce nuumeros lineales
% \libertineInitialGlyph{1234567890abc...} produce letra capital. Veáse manual de Linux LIbertine (texdoc libertine)

\usepackage{libertinust1math} % Soporte matemático para Linux Libertine.
\usepackage{hanging} % Crea sangrías francesas. Muy útil cuando queremos nuestras referencias bibliográficas en esta forma.
%\usepackage[hang]{footmisc} % No sangra las notas al pie de página.
%\setlength{\footnotemargin}{0pt} % Espacio de separación entre el número (ancla) y el texto en las notas a pie de página.

% Índice de contenido (Estilo Robert Bringhurst) con barras.

\usepackage{titletoc}

%\titlecontents{chapter}[2.25em]{}%
%{\contentslabel{2.25em}}{}%
%{\hspace{0.5em}{|}\hspace{0.5em}{\thecontentspage}}
%\titlecontents{section}[3.45em]{}%
%{\contentslabel{2.25em}}{}%
%{\hspace{0.5em}{|}\hspace{0.5em}{\thecontentspage}}
%\titlecontents{subsection}[5em]{}%
%{\contentslabel{2.25em}}{}%
%{\hspace{0.5em}{|}\hspace{0.5em}{\thecontentspage}}
%\titlecontents{subsubsection}[5em]{}%
%{\contentslabel{2.25em}}{}%
%{\hspace{0.5em}{|}\hspace{0.5em}{\thecontentspage}}

% Índice de contenido (Estilo Robert Bringhurst) con bullets.

\titlecontents{chapter}[2.25em]{}%
{\contentslabel{2.25em}}{}%
{\hspace{0.5em}{$\bullet$}\hspace{0.5em}{\thecontentspage}}
\titlecontents{section}[3.45em]{}%
{\contentslabel{2.25em}}{}%
{\hspace{0.5em}{$\bullet$}\hspace{0.5em}{\thecontentspage}}
\titlecontents{subsection}[5em]{}%
{\contentslabel{2.25em}}{}%
{\hspace{0.5em}{$\bullet$}\hspace{0.5em}{\thecontentspage}}
\titlecontents{subsubsection}[5em]{}%
{\contentslabel{2.25em}}{}%
{\hspace{0.5em}{$\bullet$}\hspace{0.5em}{\thecontentspage}}

% Indice de imágenes (Estilo Robert Bringhurst) con barras.

%\titlecontents{figure}[2.25em]{}%
%{\contentslabel{2.25em}}{}%
%{\hspace{0.5em}{|}\hspace{0.5em}{\thecontentspage}}

% Indice de imágenes (Estilo Robert Bringhurst) con bullets.

%\titlecontents{figure}[2.25em]{}%
%{\contentslabel{2.25em}}{}%
%{\hspace{0.5em}{$\bullet$}\hspace{0.5em}{\thecontentspage}}

% Indice de tablas (Estilo Robert Bringhurst) con barras.

%\titlecontents{table}[2.25em]{}%
%{\contentslabel{2.25em}}{}%
%{\hspace{0.5em}{|}\hspace{0.5em}{\thecontentspage}}

% Indice de tablas (Estilo Robert Bringhurst) con bullets.

%\titlecontents{table}[2.25em]{}%
%{\contentslabel{2.25em}}{}%
%{\hspace{0.5em}{$\bullet$}\hspace{0.5em}{\thecontentspage}}

% Opciones avanzadas en microtipografía (interlineado, interletrado, protrusión, etcétera).

\usepackage[activate={true,nocompatibility},final=true,babel=true,tracking=true,kerning=true,spacing=false,factor=1100,stretch=10,shrink=10]{microtype}
\SetTracking[no ligatures=q]{encoding=*,shape=sc}{30} % Activa el espaciado de las letras en las versalitas (3 %). Hemos desactivado la ligatura 'q' en las versalitas pues afea las palabras 'enfoque', 'Querétaro', arqueológicas, etcétera.

% activate={true,nocompatibility} - Activa la protrusion y la expansión.
% final - Activa la microtipografia; usar "draft" para desactivar.
% tracking=true - Activa el espaciado (interletraje o interletrado) en palabras de varios caractéres.
% kerning=true - Activa el espaciado únicamente en un par de caractéres. Ejemplos: AV, Ta o Yo.
% spacing=false - Elimina los espacios elásticos (blanco) empleados por defecto en LaTeX en títulos, subtítulos, subsubtítulos y texto principal.
% factor=1100 - Agrega 10% a la protrusión (1000 por defecto).
% stretch=10, shrink=10 - Reduce el estrechamiento y el encogimiento (20/20 por defecto)

\usepackage[includehead,includefoot,headsep=1cm,footskip=1cm,left=3cm,right=2cm,top=2cm,bottom=2cm]{geometry} % Para los márgenes de página y la separación del encabezado y el pie.
\usepackage{fancyhdr} % Encabezados y pies de página personalizados.
\usepackage{emptypage} % Elimina el encabezado y número de página si una página no tiene texto.
\usepackage[all]{nowidow} % Correción de viudas
\usepackage[makeindex]{imakeidx} % Para la creación de nuestro Índice General y analítico.

%\makeindex[name=nombres,title=Onomástico,columns=2]
%\makeindex[name=lugares,title=Toponímico,columns=2]

\usepackage{etoolbox} % Remueve todas las negritas del Índice general, excepto el título :-( Ugly hack para el paquetes imakeidx.

% Remueve 'Capítulo N' al principio de cada capítulo.

\makeatletter
\def\@makechapterhead#1{%
  \vspace*{50\p@}%
  {\parindent \z@ \raggedright \normalfont
    \interlinepenalty\@M
    \Huge \mdseries #1\par\nobreak
    \vskip 40\p@
  }}
\def\@makeschapterhead#1{%
  \vspace*{50\p@}%
  {\parindent \z@ \raggedright \normalfont
    \interlinepenalty\@M
    \Huge \mdseries  #1\par\nobreak
    \vskip 40\p@
  }}
\makeatother

\makeatletter
\patchcmd{\l@chapter}{\bfseries}{}{}{}
\makeatother

\usepackage{graphicx} % Para insertar figuras en nuestro documento.
\graphicspath{{imagenes/}} % Carpeta en donde están nuestras imágenes.
\usepackage{rotating} % Permite rotar las leyendas en las figuras
\usepackage{ccicons} % Para insertar iconos de Creative Commons.
\usepackage{hologo} % Para escribir palabras LaTeX, LaTeX 2e, ConTeXt, etcétera, en lenguaje TeX.
%\usepackage{draftwatermark} % Agrega marca de agua al documento.
%\SetWatermarkText{Borrador} % Personaliza la marca de agua. Borrador por defecto.
%\SetWatermarkLightness{0.8} % Establece el color de la marca de agua. Por defecto, el paquete está a 0.8
%\usepackage{fancybox} % Agrega cajas personalizadas al texto.
%\usepackage[grid=true]{eso-pic} % Muestra la cuadrícula de la página.
%\usepackage{showframe} % Muestra la disposición de los elementos de la página. (Encabezado, cuerpo de texto, pie y notas al margen).
%\usepackage{showkeys} % Muestra las etiquetas de las referencias cruzadas.
\usepackage{float} % Para colocar mejor las imágenes.
%\usepackage[right]{lineno} % Muestra números de líneas (párrafo) del documento. Muy útil cuando estamos revisando ortografía, viudas, huérfanas, etcétera, y deseamos corregir los errores.
%\linenumbers % Agrega líneas de párrafo al documento.

\usepackage{pdfpages} % Inserta documentos PDF en LaTeX. Muy últil cuando queremos agregar la portada diseñada en otro programa.

% PDF para pantalla (Día y noche)

\usepackage{xcolor} % Agrega color a elementos de la página (texto, página, etcétera). Útil si vamos a generar un PDF que sera leído en pantalla por la noche y no canse la vista.

% Color durazno (Noche)

%\definecolor{peach}{RGB}{255 218 185} % Definimos nuestro color (Durazno, por defecto).
%\pagecolor{peach} % Según estudios científicos, el color durazno es el ideal para leer un PDF en pantalla por la noche.

% Color amarillo claro (Día)

%\definecolor{amarillo claro}{RGB}{255 255 224} % Definimos nuestro color (Amarillo claro, por defecto).
%\pagecolor{amarillo claro} % Según estudios científicos, el color amarillo claro es el ideal para leer un PDF en pantalla por el día.

\usepackage{shapepar} % Para el colofón. En forma de corazón por defecto.
\usepackage{keyval} % Para hyperref.

\usepackage[hyperindex=true,final=true,bookmarks=true,bookmarksnumbered=true,bookmarksopen=true,breaklinks=true,citecolor=black,colorlinks=true,linkcolor=black,urlcolor=black,pdftitle={La fiesta del Sagrado Corazón de Jesús: un legado de la actividad devocional de la fábrica La Estrella, Tlaxcala},pdfsubject={Licenciatura en Antropología Social, Benemérita Universidad Autónoma de Puebla (BUAP)},pdfauthor={Blanca Irma Alejo Aguilar},pdfkeywords={fiesta mayordomía fabrica Estrella Chiautempan Tlaxcala},pdfproducer={pdflatex 3.14159265-2.6-1.40.17 (TeX Live 2016/Debian)},pdfcreator={Noel Merino Hernández (muxkernel@gmail.com)}]{hyperref} % Índice general con marcadores y enlaces dinámicos.

\urlstyle{rm} % Desactiva las fuentes monoespaciadas y usa las romanas (Libertine)
%\urlstyle{sf} % Desactiva las fuentes monoespaciadas y usa las romanas. (Biolinum)

% Sangre, sudor y lágrimas inician aquí.

\begin{document}
\pagenumbering{arabic} % Número de página en romanos.
\setcounter{page}{1} % Inicia la numeración. Esta será la página {1}.
\parindent=5mm % Establece la sangría para todo el documento
\parskip=0mm % Establece el espacio entre párrafos para todo el documento. Por defecto lo dejamos en 0mm.

% Portada.

% ¡Que hueva diseñar una portada en LaTeX! Mejor la creamos con LibreOffice Draw, generamos el PDF y la insertamos como como pagina 1 (Anverso) y 2 (Reverso), para que se conserve la paginación con el paquete pdfpages. Mas tarde pondré el código de la portada para que se genere automáticamente. Esto queda pendiente.

%\pdfbookmark{Portada}{Portada}
%\includepdf{02}

%\newpage
%\pagestyle{empty}
%\null\vfill

% Portadilla.

\pagestyle{empty}
\begin{titlepage}
\begin{flushright}
\pagenumbering{arabic}
\setcounter{page}{1}
\pdfbookmark{Portadilla}{Portadilla}
\biolinum
\large Miguel Ángel Esparza Ontiveros\\
\vspace{12pt}
\huge \emph{Historia e historiografía}\\
\huge \emph{del fútbol mexicano} \\
\vspace{12pt}
\large Orígenes, debates, controversias \\
\vfill
\large Publicia \\
\end{flushright}
\end{titlepage}

% Copyright

\newpage
\pagestyle{empty}
\null\vfill
\pdfbookmark{Copyright}{Copyright}

\begin{flushleft}
\begin{tiny}
\noindent \copyright \thinspace 2017 Miguel Ángel Esparza Ontiveros, del texto \\
\noindent \copyright \thinspace 2017 \href{noel_merino@yahoo.com.mx}{Noel Merino Hernández}, del diseño y maquetación \\
\end{tiny}
\end{flushleft}
%\\
\begin{minipage}{5cm}
\begin{tiny}

\begin{minipage}{5cm}
\includegraphics[width=1.5cm]{01}\label{fig:01} \\
\end{minipage}

\noindent Este libro se distribuye bajo una licencia Creative Commons Atribución 4.0 Internacional (\textsc{cc by} {4.0}). Por lo tanto, eres libre de \emph{compartir, copiar y redistribuir} el material en cualquier medio o formato. \emph{Adaptar, remezclar, transformar y construir} a partir del material para cualquier propósito, incluso comercialmente. El licenciante no puede revocar estas libertades en tanto usted siga los terminos de la licencia, bajo los siguientes términos: \\

\noindent \ccAttribution \thinspace Atribución. Usted debe dar crédito de manera adecuada, brindar un enlace a la licencia, e indicar si se han realizado cambios. Puede hacerlo en cualquier forma razonable, pero no de forma tal que sugiera que usted o su uso tienen el apoyo del licenciante. \\

\noindent No hay restricciones adicionales. No puede aplicar términos legales ni medidas tecnológicas que restrinjan legalmente a otras a hacer cualquier uso permitido por la licencia. \\

\noindent Avisos: \\

\noindent Usted no tiene que cumplir con la licencia para elementos del material en el dominio público o cuando su uso esté permitido por una excepción o limitación aplicable. \\

\noindent No se dan garantías. La licencia podría no darle todos los permisos que necesita para el uso que tenga previsto. Por ejemplo, otros derechos como publicidad, privacidad, o derechos morales pueden limitar la forma en que utilice el material. \\

\noindent Para mayores informes sobre esta licencia, visite: \\

\href{https://creativecommons.org/licenses/by/4.0/deed.es}{\textbf{https://creativecommons.org/licenses/by/4.0/deed.es}} \\

Esta libro está disponible en: \\

\href{https://github.com/tuxkernel/tesis/raw/master/tesis.pdf}{\textbf{https://github.com/tuxkernel/tesis/raw/master/tesis.pdf}}

\end{tiny}
\end{minipage}

\newpage
\pagestyle{empty}
\null\vfill

\newpage
\pagestyle{empty}
\null\vfill

% Índice

% Remueve los puntos en los índices

\makeatletter
\renewcommand\@dotsep{200}
\makeatother

\newpage
\pagestyle{fancy}
\pagenumbering{arabic}
\setcounter{page}{5}
\renewcommand{\contentsname}{Contenido}
\pdfbookmark{Contenido}{Contenido}
\tableofcontents

\newpage
\pagestyle{empty}
\null\vfill

\chapter*{\centering\mdseries\Large\textsc{Historia e historiografía \\ del fútbol mexicano}}
\pagestyle{fancy}
\fancyhf{}
\fancyhead[CO]{\small\textit{Historia e historiografía del fútbol mexicano}} 
\fancyhead[CE]{\small\textit{Miguel Ángel Esparza Ontiveros}}
\fancyfoot[RO,LE]{\small\thepage}
\renewcommand{\headrulewidth}{0pt}
\pagenumbering{arabic}
\setcounter{page}{7}
\addcontentsline{toc}{chapter}{Historia e historiografía del fútbol mexicano}

\section*{\mdseries\Large\textsc{Introducción}}
\addcontentsline{toc}{section}{Introducción}

\noindent En la actualidad, el fútbol es un fenómeno global que sobrepasa las distancias, las religiones, la raza, la clase y el género, ya que repercute en todas las actividades humanas. Al respecto, Eloy Altuve señala que el fútbol \emph{soccer} a través de la Federación Internacional de Fútbol Asociación (\textsc{fifa}), ha llegado a tener más países afiliados que la Organización de las Naciones Unidas (\textsc{onu}). Asimismo, el fútbol, en el año 2000, por sí solo movió un aproximado de 800 mil millones de dólares, convirtiendo a los deportes en la cuarta industria del mundo, detrás del petróleo, las comunicaciones y la manufactura de vehículos.\footnote{Altuve, <<Deporte ¿Fenómeno?>>, pp. 7-23.}

Nuestro país no ha sido ajeno al fenómeno futbolístico y al encanto que despierta, ya que en la actualidad el fútbol es considerado el deporte más popular y el que tiene el mayor número de espectadores y practicantes y el cual, por momentos, (parece) hace olvidar las cosas importantes de la vida de los mexicanos, como la escuela o el trabajo. Por ejemplo, el inicio de la Copa Mundial de Fútbol de Brasil 2014, despertó la euforia y la pasión de los aficionados, quienes, sin importar las consecuencias, se reportaron enfermos o de plano faltaron a la escuela o al trabajo con tal de ver los partidos de la selección mexicana.\footnote{Esmeralda Vázquez, <<La fiebre del mundial llega a la oficina>>, \textsc{cnn}expansión, 10 de junio de 2014, \url{http://www.cnnexpansion.com/negocios/2014/06/10/empresas-se-preparan-para-el-mundial}}

Sin embargo, a pesar de contar con la preferencia de millones de aficionados en el mundo y de tener gran impacto en la sociedad actual, el fútbol ha sido negado como tema de análisis académico, porque prejuiciosamente ha sido tildado de actividad populista, manipuladora de las masas y de diversión trivial carente de relevancia en comparación con los principales temas de la Historia y las Ciencias Sociales, como la política, la religión o el trabajo, aspectos considerados como los <<básicos y universales de los sistemas sociales>>.\footnote{Elias, Dunning, \emph{Deporte y ocio}, p. 11. Alabarces, <<¿De qué hablamos?>>, pp. 74-86. Collins, ``Early football'',
pp. 1127-1142. Ramírez, <<Lineamientos>>, pp. 153-181.}

Hoy en día, el estado que guarda el fútbol en la Historia resulta contrastante con la cobertura otorgada por el periodismo deportivo. Es decir, el rechazo que por años han mantenido los historiadores por los asuntos futbolísticos, ha provocado, que todo el complejo de relaciones sociales, económicas, políticas y culturales inmersas en el fútbol, queden marginadas con respecto a otros temas y campos, propiciando que la historia del fútbol ---académicamente hablando--- siga sin constituirse, al menos en el ámbito mexicano.\footnote{Fábregas, <<Identidades>>, p. 29. Alabarces, <<Deporte>>, pp. 11-28. Alabarces, <<El deporte>>, pp. 1-11.}

Mientras los estudios críticos de la historia del fútbol se caracterizan por ser escasos, en contraste, abundan ---incluso llegando a la saturación--- los trabajos de índole periodística (crónica) y de corte panegírico y apologista.\footnote{Varela, <<Goligarquías latinoamericanas>>, pp. 1-5.} Muchos de estos trabajos han sido realizados por periodistas y exdeportistas y se caracterizan por ser empíricos y anecdóticos (basados en dichos más que en hechos comprobables), faltos de análisis y rigor académico y suelen tener
innumerables errores, lagunas e imprecisiones en sus planteamientos y además, tienden a
crear y reiterar mitos más que explicar procesos o solventar problemas.\footnote{Ovalle, \emph{Historia del fútbol}, pp. 13-26. Segura, Trejo, <<Una pincelada de fútbol>>, p. 6.}

El fútbol a fin de establecerse como un tema para el análisis académico, debe superar <<la línea del anecdotario y la estadística de los campeonatos, para con ello vencer el aislamiento, la marginalidad, la apatía y los prejuicios de los académicos, para finalmente mostrar que puede explicar procesos sociales de diversa índole incluidos los históricos.>>\footnote{Además de los resultados y de las características que distinguen al fútbol de otros deportes, su estudio contribuye a la comprensión de los procesos que componen un sistema deportivo: los procesos sociales que les dieron surgimiento, los contextos político y económico en los que ocurrieron, la organización y construcción de las comunidades que lo practican en escala local, nacional e internacional, así como las formas y usos del fútbol (mercadotecnia, propaganda política y patriótica, legitimación, cohesión, control e ingeniería social). Riess, ``The new sport'', pp. 311-325. Macías, <<El fútbol y el Bajío>>, pp. 1-16.} Por ejemplo, en Brasil, el fútbol ha sido un espejo social en donde la política, la economía, el nacionalismo, la clase, el género o la raza, son analizados en un microcosmos. El fútbol ha tenido la capacidad de proclamar la igualdad entre las clases y las razas además de sublimar los principales conflictos sociales y políticos de una sociedad pluriétnica como la brasileña donde el fútbol ha fungido como un efectivo cemento social.\footnote{Lever, \emph{La locura por el futbol}, pp. 9-117.}

En síntesis, resulta necesario ---como una manera de implantar el rigor académico, de consolidar el campo de la historia del fútbol y de desmarcar el trabajo científico de los textos de los no historiadores---\footnote{Bass, ``State of the field'', pp. 148-172.} realizar una labor crítica y revisionista pues a la fecha, persisten datos y planteamientos incorrectos, además de grandes vacíos y lagunas en la historia del fútbol mexicano, principalmente, en lo referente a sus orígenes.\footnote{El revisionismo es parte medular del oficio historiográfico, ya que permite conocer el campo de estudio al evaluar las aportaciones al conocimiento realizadas por otros trabajos y poner a debate sus paradigmas, metodologías, fuentes y resultados, con el fin de ubicar sus debilidades y omisiones, ya sea para refutarlas y corregirlas o para encontrar un campo, línea o tema aún no trabajado y que permita, por una parte, producir nuevos conocimientos y por otra, evitar la duplicidad de las investigaciones y de los resultados. En México, David Bailey denominó como <<revisionista>> a una historiografía surgida en los años 60 que, con nuevos enfoques, interpretaciones, fuentes e hipótesis, realizaban una crítica y se contraponían a los planteamientos clásicos de la Revolución. El revisionismo tiende a ser generacional y funciona de manera dialéctica, es decir, primero se presenta un relato sintético, después una antítesis y finalmente una nueva síntesis que pretende trascender a las dos primeras. Matute, <<Orígenes del revisionismo>>, pp. 29-48. Miganjos, <<La Revolución Mexicana>>, pp. 144-158. Knight, <<Interpretaciones>>, pp. 23-43.}

Debido a que los hechos históricos <<son siempre susceptibles de interpretación y reinterpretación desde nuevas perspectivas y por aparición de nuevos datos>>,\footnote{Montesinos, <<Estrategias>>, pp. 41-102.} el revisionismo representa una corriente historiográfica que se destaca por proponer nuevas formas de abordar y explicar la historia, que supera los estancamientos temáticos, teóricos, metodológicos y paradigmáticos y que se caracterizan por criticar, poner en duda y rechazar los postulados, las verdades y conclusiones ortodoxas y oficialistas que en múltiples ocasiones contienen lagunas, errores e imprecisiones por defecto de interpretación y en algunos casos deformaciones y falsedades dolosas para justificar algún orden político.\footnote{Serrano, <<Historiografía regional>>, pp. 49-57. Jauretche, \emph{Política nacional}, pp. 15-74. Camargo, <<La construcción de la historiografía>>, pp. 1-20. Knight, <<Punto de vista>>, pp. 91-127.}

En la Historia es necesario analizar críticamente los pasos dados en la disciplina, con
el fin de evidenciar si el camino recorrido ha sido el correcto o si conviene modificar el
rumbo. José Gaos señalaba que toda preposición histórica sostenida como <<verdadera>> debía
ante todo ser sometida a rigurosa comprobación.\footnote{Gaos, <<Notas sobre la historiografía>>, pp. 481-508.} En ese mismo sentido, Eric Dunning
establece que los historiadores no trabajan con hechos, sino con sus restos, por tanto, no es
posible acceder a la verdad de esos hechos, tan sólo se puede obtener una <<figuración>>, es decir, una interpretación apegada a la realidad que permite conocer el contexto histórico, las
acciones de los individuos, su modo de vida, comportamientos y formas de pensar.\footnote{Dunning, ``Sport in space and time'', p. 335.}

Teniendo en cuenta que todas las historias son tan sólo una figuración producto de la
forma en que los historiadores trabajan e interpretan las evidencias, resulta necesario realizar
una revisión y reinterpretación de la historia del fútbol mexicano, ya que se observa la
existencia de lagunas y vacíos, además de errores, distorsiones e incluso falsedades que hacen
necesaria la precisión y comprobación de datos, planteamientos, paradigmas y resultados.
Así que el objetivo de este estudio es el de revisar (a través del paradigma indiciario) las
principales premisas que versan sobre el origen del fútbol en México.\footnote{Además del paradigma indiciario, complementariamente se hará uso de los términos \emph{patrón} y \emph{situación}, para ubicar heurísticamente en la información empírica cómo surge el fútbol \emph{soccer} en México y cómo era su dinámica deportiva. El término \emph{patrón} se define como aquellas acciones y estrategias llevadas a cabo por individuos con el fin de fomentar y desarrollar las actividades deportivas, mientras que el término \emph{situación}, se refiere al contexto donde esas acciones y estrategias tienen lugar. Dunning, \emph{El fenómeno deportivo}, p. 50.}

No se busca repetir, sino reinterpretar cómo surgió el fútbol \emph{soccer} en México
haciendo uso del paradigma indiciario para reinterpretar las fuentes primarias y secundarias
y finalmente construir una nueva figuración que demuestre que el surgimiento de este deporte
es diferente a como se había pensado. La historia es una ciencia basada en las significaciones
atribuidas a las evidencias y el paradigma indiciario es un método que se caracteriza por
recuperar y dotar de sentido a todos esos detalles que pasan desapercibidos.\footnote{Ginzburg, \emph{El queso y los gusanos}, pp. 13-90. Ginzburg, <<Indicios>>, pp. 185-239.}

El paradigma indiciario es capaz de extender el conocimiento histórico superando las
inexactitudes, deformaciones y falsedades de las interpretaciones ortodoxas, al exponer
nuevos datos y al replantear la manera como se han interpretado y explicado los procesos
históricos. Con el paradigma indiciario reescribiremos y reinterpretaremos la historia del fútbol mexicano al recuperar los <<indicios>> (un conglomerado de datos que han sido negados, marginados e ignorados por otros historiadores, porque a primera vista no son evidentes o por aparentar ser secundarios e intrascendentes) mediante una <<lectura intensiva de los textos>> que permita descifrar y extraer <<las realidades ocultas y profundas>> de toda la información considerada como fragmentaria o incompleta de los hechos históricos.\footnote{El paradigma indiciario, rechaza toda interpretación obvia, evidente y trillada de los hechos históricos y en su lugar, establece nuevas interpretaciones, a partir de <<las realidades ocultas>> que se encuentran inmersas en los indicios. Por ejemplo, a pesar de ser eminentemente oral, la cultura de las clases subalternas ha sido descifrada por Carlo Ginzburg a partir de una lectura detallada de sus indicios, logrando con ello, reconstruir la estructura principal de la cultura popular campesina de Italia en el siglo XVI. Aguirre, <<Indicios>>, pp. 09-44.}

Historiográficamente hablando, la aplicación del paradigma indiciario al estudio del
fútbol mexicano permitirá, por una parte, revisar los trabajos previos (sus fuentes,
paradigmas, enfoques y resultados). En segundo término, permitirá recuperar, descubrir y
reinterpretar las fuentes primarias, asimismo, hará posible la producción de nuevos conocimientos y formas de explicar la historia del fútbol que, en última instancia,
enriquecerán y acrecentarán los debates historiográficos del fútbol \emph{soccer}, muchos de los
cuales ---infortunadamente--- dentro del ámbito académico mexicano poco se conocen, debido
a que el análisis del fútbol y la construcción de su campo de estudio todavía se encuentran en
su etapa formativa, por lo que los paradigmas y las tendencias metodológicas, así como los
principales debates que de alguna manera le dieron forma a esta subdisciplina, han pasado
desapercibidos para nosotros, por lo que considero conveniente hacer un breve recuento para
comprender cómo ha sido el desarrollo de los estudios históricos del fútbol.

\section*{\mdseries\large\textsc{El fútbol en la historia: un recuento}}
\addcontentsline{toc}{section}{El fútbol en la historia: un recuento}

\noindent Aunque el campo de la historia del fútbol es relativamente reciente (a lo mucho alcanza un
rango de 50 años), ha evolucionado rápidamente, pues durante ese espacio de tiempo ha
experimentado muchos y significativos cambios, ya que se han reevaluado y modificado sus
paradigmas, metodologías, enfoques e incluso el perfil de los historiadores del fútbol, lo cual
ha ampliado y acrecentado el rango de variables, enfoques y metodologías empleadas,
mejorando con ello la calidad de los trabajos.\footnote{Contrario a lo que se pudiera pensar, la Historia no es una ciencia estática, sino una disciplina en constante movimiento que de manera recurrente somete a revisión y crítica las formas en que se estudia el pasado, con el objetivo de reformular y replantear sus paradigmas, metodologías y objetos de estudio. Fue en uno de esos momentos de revisión y replanteamiento que se conoce como \emph{linguistic turn}, que el fútbol comenzó a ser historiado. Dosse, <<La historia>>, pp. 17-54. Burke, <<Obertura>>, pp. 11-37. Burke, \emph{La Revolución}, p. 11, 12. Florescano, \emph{Historia de las historias}, pp. 435-438. Holt, ``Historians'', pp. 1-33. Bass, ``State of the field'', pp. 148-172. Ruck, ``The field of Sport History'', pp. 192-194.}

Es decir, de las crónicas de corte periodístico (textos empíricos y anecdóticos) que
principalmente se enfocan en relatar cronológicamente el surgimiento del fútbol, así como
las hazañas más memorables de los héroes más representativos de los clubes, se dio paso a los textos académicos (narrativas analíticas) que más que relatar, explicaban el origen y
desarrollo del fútbol a partir de la aplicación de categorías y modelos teóricos diversos.\footnote{Debido a que los primeros estudios históricos del deporte (incluidos los del fútbol) fueron desarrollados por personajes no formados como historiadores (periodistas, exdeportistas, profesores de educación física) sus trabajos muestran una estructura descriptiva-narrativa, un modelo unidimensional que de manera secuencial y cronológica ordena la información empírica. Esta forma de trabajar la historia se conoce como reconstruccionista o empirista y consiste en realizar una recopilación y clasificación de la información, además de una meticulosa descripción los hechos sin modificarlos para que éstos <<hablen por sí mismos>>. En un segundo momento, personajes formados como historiadores o como sociólogos, comenzaron a analizar a los deportes, buscando entender las causas que les dieron origen, además de explicar sus contribuciones para el desarrollo de algunas instituciones o grupos que los promovían y practicaban. Esta forma de historiar se conoce como construccionista y aplica teorías como el estructuralismo, funcionalismo, análisis figuracional, marxismo, modernización y categorías como clase, raza, género y nacionalismo que son utilizadas como marcos interpretativos para contextualizar y organizar sus evidencias y para construir sus explicaciones. Morrow, ``Canadian sport history'', pp. 67-79. Phillips, ``Deconstructing sport history'', pp. 327-343. Struna, ``Social history and sport'', pp. 187-203. Booth, ``Theory'', pp. 12-34. Booth, \emph{The field}, pp. 23-194.}

Fue en la década de 1960 cuando el futbol comenzó a ser historiado.\footnote{Desde el siglo \textsc{xix} existen trabajos escritos por cronistas (Joseph Strutt, William Hone), periodistas (John Cartwright) y directivos del fútbol (Montague Shearman y Charles Alcok) y que relatan que este deporte ha formado parte de la historia del Reino Unido desde el siglo \textsc{xiii}, sin embargo, fue hasta la década de 1960 que toman relevancia como fuentes primarias para la historia del fútbol. Collins, ``Early football'', pp. 1127- 1142. Curry, ``The origins of football debate'', pp. 2158-2163. Dunning, Curry, ``Public schools'', pp. 31-52.} Periodistas
como Charles Sutcliffe, Fred Hargreaves, Francis Magoun, Morris Marples, Geoffrey Green
y Percy Young (entre otros), fueron los primeros en relatar la historia del fútbol en el Reino
Unido, enfocándose en describir que el origen de este deporte era respetable, ya que provenía
de las \emph{Public Schools}, uno de los círculos más elitistas de la sociedad británica.\footnote{Collins, ``Early football'', pp. 1127-1142. Young, \emph{A history.} Green, \emph{The history of football association.}}

Posteriormente, el fútbol atraería a los académicos recién egresados de los posgrados
de Historia y Sociología y a los historiadores formados en el campo de la historia social.\footnote{Bass, ``State of the field'', pp. Pope, Nauright, ``Introduction'', pp. 1-12.} El
sociólogo Eric Dunning en 1961 desarrolló el primer texto académico del fútbol. Dos años
después, publicó en la revista \emph{History Today} la \emph{Status Rivalry Hypothesis}, un planteamiento
que se volvió un referente para otros académicos y que se establecería como el cisma
historiográfico por antonomasia. La \emph{Status Rivalry} reafirma que el fútbol \emph{soccer} tiene un linaje respetable, pues luego de que el fútbol tradicional o \emph{folk football} cayera en desuso por
las prohibiciones de las autoridades, fue recuperado y regulado por los alumnos de las \emph{Public
Schools} (Eton, Harrow, Westminster, Rugby, entre otras), las cuales comenzaron a rivalizar
entre sí tratando de imponer su estilo de juego sobre los otros.\footnote{Académicos como James Walvin, Tony Mason, David Russell (entre otros) comenzarían a historiar el fútbol siguiendo las ideas de Eric Dunning. Collins, ``Early football'', pp. 1127-1142. Dunning, Curry, \emph{Association football}, p. 1, 2. Walvin, \emph{The People's game}. Holt, ``Historians'', pp. 1-33.}

El punto culminante de esta rivalidad se suscitó en 1863, cuando los partidarios de
los principales estilos de fútbol (Eton y Rugby), se reunieron en Londres con el objetivo de
desarrollar un reglamento unificado que permitiera la práctica del fútbol a nivel nacional. De
las seis reuniones celebradas surgió la \emph{Football Association}, sin embargo, no fue posible
desarrollar un estilo de juego único, pues los partidarios del \emph{rugby} se retiraron luego de que
se excluyera tomar el balón con las manos y en su lugar se priorizara patearlo, aspecto que
poco a poco se refinaría hasta convertirse en el moderno fútbol \emph{soccer}.\footnote{Previo a la formación de la \emph{Football Association}, en la Universidad de Cambridge los exalumnos de Eton, Rugby y demás \emph{Public Schools}, realizaron varios intentos por desarrollar un reglamento común (1846, 1856, 1858), que permitiera una mayor interacción entre los diferentes grupos, sin embargo, ningún intento fructificó debido a que los partidarios del estilo aristocrático de Eton, no estaban dispuestos a modificar su estilo de juego por el de los burgueses de Rugby, al que consideraban falto de jerarquía y nobleza y ante la imposibilidad de lograr algún acuerdo, cada grupo optó por seguir practicando su estilo de juego separado de los otros. Harvey, ``An epoch'', pp. 53-87. Curry, Dunning, \emph{Association football}, pp. 69-77. Dunning, Curry, ``Public Schools'', pp. 46-48. Harvey, \emph{Football}, pp. 33-44}

La tesis que plantea que el fútbol \emph{soccer} se configuró y se difundió por influencia de
las \emph{Public Schools} inglesas, ha sido el paradigma reinante en la historia del fútbol por al
menos cuatro décadas, sin embargo, en los últimos quince años ha sido puesta en duda por
un grupo de historiadores conocidos como los revisionistas y encabezados por Adrian
Harvey, John Goulstone y Peter Swain, quienes han abierto un nuevo debate de los orígenes
del fútbol en el Reino Unido, al realizar una revisión crítica del paradigma dominante, así como una precisión a sus datos, planteamientos, teorías y resultados, que han dado lugar a
una nueva versión del desarrollo histórico del fútbol \emph{soccer} en el Reino Unido.\footnote{En 1974, John Goulstone fue el primer historiador que puso en entredicho la \emph{Status Rivalry Hypothesis} de Eric Dunning, señalando que fuera de la influencia de las \emph{Public Schools} se desarrolló otra subcultura del fútbol. Sin embargo, por años su trabajo paso desapercibido para otros historiadores, porque no se publicó como libro sino en formato de panfleto que carecía de la estructura de un texto académico (marco teórico y referencias) y bajo este estilo se depositó en la \emph{British Museum Library}, lo cual contribuyó a que se mantuviera olvidado. Harvey, ``The emergence'', pp. 2154-2163. Mangan, ``Missing men'', pp. 170-188. Swain, Harvey, ``On Bosworth'', pp. 1425-1445.}

\section*{\mdseries\large\textsc{El fútbol y el revisionismo historiográfico: \textit{the origins of football debate}}}
\addcontentsline{toc}{section}{El fútbol y el revisionismo historiográfico: \textit{the origins of football debate}}

\noindent Contrario a lo que establece el paradigma dominante, los revisionistas rechazan que a
principios del siglo \textsc{xix} el \emph{Folk Football} (fútbol rural y tradicional) hubiera desaparecido
por completo, sino que lejos de declinar, el fútbol seguía en boga en varias regiones del Reino
Unido como Yorkshire, donde siguió practicándose con mucho auge, no sólo bajo la forma
rural y tradicional sino también en formatos más regulados y organizados.\footnote{Curry, Dunning, ``The problem with revisionism'', pp. 429-445. Harvey, ``The emergence'', pp. 2154-2163. Kitching, ``Old football'', pp. 1733-1749. Swain, ``The origins of football debate'', pp. 519-543.}

De igual forma, los revisionistas rechazan que el fútbol \emph{soccer} haya sido configurado
y difundido por las \emph{Public Schools}, pues se ha documentado la existencia de una subcultura
futbolística independiente en la ciudad de Sheffield (región de Yorkshire).\footnote{El tipo de fútbol practicado en Sheffield no era del rural tradicional o \emph{folk football}, es decir, este estilo de fútbol no era masivo, sino que celebraba encuentros con equipos de diez o 20 jugadores por bando y se llevaba a cabo en campos con áreas de juego delimitadas en lugar de las calles de la ciudad. Estos encuentros se celebraban semana a semana y se organizaban a partir de una apuesta, también había menos violencia porque la rudeza intencional estaba prohibida y era sancionada. Curry, Dunning, ``The problem with the revisionism'', pp. 429-445. Kitching, ``Old football'', pp. 1733-1749. Swain, ``The origins of football debate'', pp. 299-317. Harvey, \emph{Football}, pp. 1-91.} Según Harvey,
no existe evidencia que indique que el fútbol practicado en Sheffield se derive de las \emph{Public
Schools}, sino que fue desarrollado de forma independiente, a partir de las ideas que
circulaban en la localidad y fuera del contexto de la elite.\footnote{Desde 1855 existió en el área de Sheffield una extensa y homogénea cultura deportiva que de forma recurrente practicaba diversos deportes, en especial el fútbol. Dos años después, se fundó el Sheffield \textsc{fc}, a la fecha el club de fútbol más antiguo del mundo. Harvey, ``The emergence'', pp. 2154-2163. Swain, Harvey, ``On Bosworth'', pp. 1425-1445. Harvey, \emph{Football}, pp. 93-231.}

Asimismo, Harvey establece que el estilo de fútbol practicado en Sheffield fue el más
importante del Reino Unido y el Mundo en la década de 1850 a 1860 y afirma que los
reglamentos desarrollados en Sheffield son los antecedentes que dieron origen a las reglas
del fútbol \emph{soccer} actual.\footnote{Collins, ``Early football'', pp. 1127-1142. Kitching, ``Old football'', pp. 1733-1749. Harvey, ``An epoch in the annals'', pp. 53-87.} Para los revisionistas no existe ninguna \emph{Status Rivalry}, pues en su consideración, es una teoría carente de sustento, porque previo a 1859 hubo poco contacto entre Eton y Rugby y ninguna evidencia de rivalidad entre ellas.\footnote{Harvey, ``The emergence'', pp. 2154-2163.}

De igual forma, los revisionistas señalan que el contacto entre las \emph{Public Schools} y el
resto de la sociedad británica fue mínimo, por tanto, las reglas y los estilos de juego de la
elite tuvieron poco impacto en el desarrollo del fútbol \emph{soccer}. Para los revisionistas, se ha
exagerado el rol que jugaron las \emph{Public Schools} en la configuración del fútbol \emph{soccer},
mientras que el fútbol practicado en Sheffield en la década de 1850-60, aún no tiene el
reconocimiento que merece como el verdadero el precursor del moderno fútbol \emph{soccer}.\footnote{Harvey, ``The emergence'', pp. 2154-2163. Collins, ``Early football'', pp. 1127-1142. Harvey, ``An epoch in the annals'', pp. 53-87.}

Al establecer que Sheffield es el precursor del fútbol \emph{soccer}, los revisionistas
pretenden imponer un nuevo paradigma, señalando que en Sheffield se realizaron mayores
contribuciones en la configuración del fútbol \emph{soccer}.\footnote{Sheffield aportó varias innovaciones al desarrollo del moderno soccer; inventó los tiros penales, también aportó los tiros de esquina, los tiros libres y los travesaños de las porterías. Swain, Harvey, ``On bosworth'', pp. 1425-1445. Harvey, \emph{Football}, p. 123.} Asimismo, atribuyen la sobrevivencia y difusión de las reglas y estilo de juego de la \emph{Football Association}, no a las \emph{Public Schools}, sino a los practicantes del estilo de fútbol de Sheffield que en su mayoría eran miembros de
la clase media y baja (profesores, abogados, dependientes y obreros).\footnote{Según los revisionistas, en 1867 los practicantes del fútbol en Sheffield evitaron la disolución de la \emph{Football Association} (\textsc{fa}) luego de que perdiera más de la mitad de los equipos que la habían fundado. En Sheffield se organizaron varios partidos para promocionar el estilo de juego y los reglamentos de la \textsc{fa}, también, los cerca de 1200 practicantes del estilo de fútbol de Sheffield se afiliaron a la \textsc{fa} y finalmente, en 1877, el área de Sheffield abandonó sus reglas para fusionarse con la \textsc{fa} y adoptó sus reglamentos y estilo de juego que eventualmente daría lugar al moderno fútbol \emph{soccer}. Collins, ``Early football'', pp. 1127-1142. Swain, Harvey, ``On Bosworth'', pp. 1425-1445. Harvey, ``An epoch in the annals'', pp. 53-87. Harvey, \emph{Football}, pp. 123-211.}

La respuesta por parte de la corriente ortodoxa no se hizo esperar, señalando que, si
bien los revisionistas han presentado nuevas e irrefutables evidencias que detallan la
existencia de la práctica del fútbol en el área de Sheffield previo a la formación de la \emph{Football
Association} en 1863, dicha evidencia no sustenta que Sheffield sea el precursor del moderno
fútbol \emph{soccer}, debido a que declina en cantidad y calidad luego de 1860.\footnote{Lewis, ``Innovation not invention'', pp. 475-488.}

A decir de Curry y Dunning, los planteamientos de los revisionistas tienen algunas
inconsistencias que hacen insostenible que el fútbol de Sheffield sea el precursor del moderno
fútbol \emph{soccer}. Por ejemplo, es incorrecto que el fútbol en Sheffield careciera de influencia de
las \emph{Public Schools}, pues se ha demostrado la inclusión del \emph{Rouge} (una forma de puntaje)
proveniente del estilo de juego de Eton.\footnote{Curry, ``Playing for money'', pp. 336-355. Curry, Dunning, \emph{Association Football}, pp. 92-114. Collins, ``Early football'', pp. 1127-1142.}

Los ortodoxos tampoco reconocen que Sheffield evitara la disolución de la \emph{Football
Association} (\textsc{fa}) al fusionarse con ella, sino que consideran que el área de Sheffield se sujetó
a las reglas y estilo de juego de la \textsc{fa}, porque la influencia del fútbol de Sheffield fue de
alcance local, por tanto, su contribución para el desarrollo y configuración del moderno fútbol
\emph{soccer} es también mínima.\footnote{Curry, Dunning, \emph{Association Football}, pp. 178-180. Collins, ``Early football'', pp. 1127-1142.} Asimismo es insostenible la tesis que señala que la difusión del estilo de juego de la \emph{Football Association} se deba a los miembros de las clases media y baja, ya que los revisionistas se basaban en información <<escasa y dispersa>>.\footnote{A decir de los ortodoxos, los revisionistas hacen mención de un total de 59 partidos en un periodo de 29 años (en promedio dos por año). Además, muchos de estos encuentros sólo fueron retos ya que no hay pruebas que confirmen que se hayan llevado a cabo. Con tan mínima evidencia difícilmente se puede acreditar al fútbol de Sheffield como el precursor del moderno fútbol \emph{soccer.} Curry, Dunning, \emph{Association Football}, pp. 155-160. Collins, ``Early football'', pp. 1127-1142. Curry, Dunning, ``The problem with the revisionism'', pp. 429-445.}

Por supuesto que el debate no ha finalizado, pues ambos grupos siguen presentando
nuevos argumentos y evidencias.\footnote{El debate se ha vuelto interdisciplinar, ya que los revisionistas (historiadores) acusan a los ortodoxos (sociólogos) de forzar y ajustar los hechos para <<validar la teoría>> (en este caso el proceso de civilización de Elias), mientras que los ortodoxos, señalan que la metodología de los revisionistas carece por completo de una teoría y sólo rastrean el estilo de fútbol más parecido al moderno. Holt, ``History'', pp.1-33. Collins, ``Early football'', pp. 1127-1142.} Por ejemplo, hoy en día, ambas corrientes señalan que la historia del fútbol es más compleja de como se había pensado previamente y que los hallazgos
de los revisionistas hacen considerar que el fútbol en sus dos principales vertientes (\emph{soccer} y \emph{rugby}) fue <<coproducido>> por las influencias de distinto estrato social (la elite, la clase media y la clase baja) y por las ideas circulantes tanto en Londres como en Yorkshire, aunque claro, cada corriente le atribuye mayor importancia e influencia a diferente grupo y región: los ortodoxos a las elites de las \emph{Public Schools} y al área de Londres; los revisionistas al área de Sheffield y a las clases media y baja.\footnote{Curry, Dunning, ``The problem with revisionism'', pp. 429-445. Curry, Dunning, \emph{Association football}, pp. 175-190. Harvey, ``The emergence'', pp. 2154-2163.}

Aunque hoy en día se tienen más conocimientos acerca del desarrollo histórico del
fútbol, no es posible precisar dónde surgió el moderno fútbol \emph{soccer}.\footnote{Gracias a la digitalización de periódicos y archivos judiciales del siglo \textsc{xix}, nuevos datos están saliendo a la luz, los cuales han permitido producir nuevas explicaciones del surgimiento del fútbol soccer que mantienen en boga el debate historiográfico. Swain, ``The origins of football debate'', pp. 2212-2229. Swain, ``The origins of football debate: football and cultural'', pp. 631-649. Hay, ``A tale of two footballs'', pp. 952-969.} Si en el Reino Unido
donde el estudio del fútbol ha sido sumamente productivo, no se puede determinar con
certeza dónde surgió el fútbol \emph{soccer} ¿Por qué en México, donde el estudio académico del fútbol aún es incipiente y donde gran cantidad de archivos y fuentes de información aún no
han sido agotados, se considera que el tema referente al surgimiento del fútbol \emph{soccer} es un
tema ya estudiado y por todos conocido, por tanto, un tema ya cerrado y concluido? Esta
cuestión, así como otras derivadas, serán objeto de discusión en los siguientes apartados.

\section*{\mdseries\large\textsc{El fútbol en la historiografía mexicana: un análisis}}
\addcontentsline{toc}{section}{El fútbol en la historiografía mexicana: un análisis}

\noindent Mientras en el Reino Unido el fútbol ha sido sujeto de análisis por al menos medio siglo, en
México se mantiene ignorado, porque la Historia ---académicamente hablando--- le ha prestado
poca atención al estudio de este deporte, a pesar de que se practica en el país desde fines del
siglo \textsc{xix}.\footnote{Angelotti, \emph{Chivas y tuzos}, p. 86. Angelotti, <<El estudio del fútbol>>, pp. 211-222. Ramírez, <<Lineamientos>>,
pp. 153-181. Macías, <<El fútbol y el Bajío>>, pp. 1-16.} En efecto, aunque en los últimos años la producción historiográfica ha
aumentado, aún no se puede hablar de la existencia de un campo de estudio plenamente
constituido, pues difícilmente se imparten cursos de historia del deporte en las universidades,
los congresos y conferencias son esporádicos y es complicado que trabajos con temática
deportiva sean publicados en las principales revistas académicas.\footnote{La revista \emph{Historia Mexicana} (fundada en 1951) a la fecha sólo ha publicado dos artículos con temática deportiva: Beezley, <<El estilo porfiriano>> y Rodríguez Kuri, <<Ganar la sede>>. De igual forma, la Revista \emph{Sociológica} en 25 años de existencia únicamente ha publicado un texto referente a los deportes. Ramírez, <<Lineamientos>>, pp. 153-181. Alberro, <<El primer medio siglo>>, pp. 643-653. Meneses, Ávalos, <<La investigación del fútbol>>, pp. 33-64.}

En parte, el rezago que guardan los estudios históricos del fútbol en México es
atribuible a la apatía y desinterés de los académicos, pero también se debe a cómo se ha
desarrollado el ámbito académico mexicano, ya que, desde la profesionalización de la
Historia (circa 1940), se dio prioridad a los asuntos políticos y económicos y se desdeñaron los
aspectos sociales y culturales que se consideraban como no relevantes, donde se engloba a las <<asociaciones de damas caritativas, clubes campestres, cafés, academias científicas y literarias, clubes de leones y otras especies de la misma índole y sociedades de charros y de tantos juegos de pies y de músculos que se agrupan en el rótulo de los deportes>>\footnote{González, \emph{El oficio de historiar}, pp. 48-174. Sánchez, <<Hacia una historia>>, pp. 25-45.}

Hasta antes del primer cuarto del siglo \textsc{xx}, el ejercicio histórico estuvo en manos de
médicos, juristas, políticos, militares, párrocos, literatos y demás letrados que por mero gusto
o por interés político, decidieron tomar la pluma y escribir su versión de la historia de
México.\footnote{Los temas más abordados eran la época prehispánica, la colonia y la Independencia. González, \emph{El oficio de historiar}, pp. 46-48. González, <<Silvio Zavala>>, pp. 7-19. Matute, <<Orígenes del revisionismo>>, pp. 29-48. Vázquez, <<Cincuenta y tres años>>, pp. 709-718.} Fue hasta la década de 1940, cuando se inicia con la profesionalización de la Historia gracias a la consolidación del Estado Mexicano que permitió la fundación de centros e instituciones como el \textsc{inah} o el Colegio de México, espacios científicos donde se
formarían las primeras camadas de historiadores académicos.\footnote{González, <<Silvio Zavala>>, pp. 7-19. Vázquez, <<Historia Mexicana>>, pp. 11-23. Anaya, <<La construcción de la memoria>>, pp. 525-536. Zermeño, <<La historiografía en México>>, pp. 1695-1742.}

Sin embargo, el fútbol no figuró como tema de estudio para la Historia porque para
al menos tres generaciones de historiadores, el principal tema del siglo \textsc{xx} (por su
trascendencia política, económica y social) fue la Revolución Mexicana.\footnote{A decir de Guillermo Zermeño, <<La revisión de la Revolución Mexicana fue uno de los campos de estudio preferidos de la nueva generación de historiadores>> nacionales y extranjeros. En un principio, el revisionismo no fue historiográfico sino político. Las siguientes generaciones de revisionistas de la Revolución de 1960 y 1970, comenzarían a emplear nuevos enfoques (regional) que pretendían desmitificar y superar la idea de la Revolución como movimiento nacional, agrario y popular que confrontó en gran escala al gobierno dictador y a los grupos terratenientes. Zermeño, <<La historiografía en México>>, pp. 1695-1742. Matute, <<Orígenes del revisionismo>>, pp. 29-48. Knight, <<Interpretaciones>>, pp. 23-43.} Asimismo, otros
historiadores fueron influenciados por estilos historiográficos extranjeros ---principalmente
por la \emph{Escuela de los Annales}--- y decidieron trabajar temáticas de orden socioeconómico.\footnote{En 1969, Enrique Florescano publicó \emph{Precios del maíz y crisis agrícolas}, uno de los primeros estudios de historia serial en México. Pablo González Casanova, por su parte, trabajó la historia del movimiento obrero en México con una perspectiva marxista. Zermeño, <<La historiografía en México>>, pp. 1695-1742.}

Finalmente, el fútbol no fue objeto de estudio para la Historia por ser una actividad contemporánea a la profesionalización del gremio. Hasta antes de la década de 1980 se pensaba que el tiempo presente no era historiable, así que todas las actividades cuyo desarrollo fuera visible, no se consideraban como un objeto de estudio para los historiadores contemporáneos, sino un campo de competencia para los cronistas y reporteros.\footnote{González, \emph{El oficio de historiar}, pp. 63-160. Matute, <<Estudios de Historia Moderna>>, pp. 779-789.}

Los estudios históricos del fútbol mexicano se pueden catalogar en dos rubros
básicos: los realizados por aficionados (periodistas, exfutbolistas y literatos)\footnote{Escritores renombrados como Juan Villoro o Eduardo Galeano, han escrito novelas, ensayos, poemas y cuentos donde exponen lo que para ellos representa el fútbol y lo que éste genera en las sociedades. Galeano, \emph{El fútbol a sol}. Villoro, \emph{Dios es redondo}. Angelotti, <<El estudio>>, p. 215. Ovalle, \emph{Historia del fútbol}, pp. 22-25.} y los
realizados por historiadores académicos. Los primeros son el rubro más prolífico y el más
antiguo, pues desde de 1960 se publicaron los primeros trabajos sobre el particular, como \emph{El
libro de oro del fútbol mexicano}, de Juan Cit y Mulet, un texto considerado como
indispensable para entender el surgimiento y la trayectoria de este deporte en México.\footnote{El gran mérito de Juan Cit y Mulet fue el de descubrir el potencial del fútbol para ser historiado en una época donde imperaba el estudio histórico de los asuntos políticos y económicos. Cit, \emph{El libro de oro}, pp. 17-42.}

Este tipo de trabajos describen las hazañas y los hechos deportivos más
trascendentales de un futbolista o un equipo. Las crónicas son, ante todo, trabajos carentes
de rigor académico, llenos de inconsistencias y finalmente, por no contar con suficientes
fuentes de información, contienen muchas lagunas y vacíos históricos que han contribuido a
incrementar las ficciones y la vigencia de los mitos del fútbol mexicano, muchos de los cuales
se siguen considerando como verídicos.\footnote{Las crónicas son narrativas escritas con lenguaje informal y coloquial que entretienen e informan <<recontando>> las anécdotas y los hechos del pasado inmediato que, a juicio del cronista, son los sucesos más emblemáticos para una comunidad y los más dignos de ser recordados. Las crónicas suelen entremezclar la ficción con la realidad y están llenas de opiniones personales y juicios de valor que rara vez explican o comprueban lo que dicen, pues las crónicas, tal y como dice Juan Villoro, son como los carteros, no escriben las noticias, sólo se encargan de entregarlas. Por otra parte, según David Wood muchos de los textos catalogados como crónicas, fueron en un principio noticias publicadas en los diarios que posteriormente se publicaron bajo
el formato de colecciones. Wood, ``Playing by the book'', pp. 27-41. González, \emph{El oficio de historiar}, pp. 99-
104. Espinoza, <<La vida privada de los goles>>, pp. 86-90.}

En el rubro de las crónicas deportivas se incluyen los trabajos de corte biográfico y
las historias de clubes y equipos. Los textos de carácter vivencial se enfocan en resaltar las
cualidades de los biografiados e ignorar sus acciones poco éticas, pues el objetivo principal
de una biografía es el de rendir homenaje y convertir en modelo a seguir al personaje del cual
se escribe su historia. Similarmente, las historias de clubes y equipos también son trabajos a
modo, donde principalmente se habla de los orígenes del club, de sus triunfos memorables,
sus goleadores históricos y sus dirigentes. Sin embargo, por la falta de crítica y análisis su
trascendencia es mínima, aunque funcionan comercialmente, ya que son textos plagados de
fotos, datos y anécdotas poco conocidas del mundo del fútbol y sus actores.\footnote{Los textos biográficos y las historias de clubes, son trabajos cuyo fin último es el de ganar dinero privilegiando las anécdotas y en detrimento del análisis y rigor académico. Textos como la biografía de Jorge Campos, los libros de Carlos Calderón sobre el Club Pachuca y los publicados por Editorial Clío, pertenecen a esta categoría. Ramírez, <<Lineamientos>>, p. 155. Angelotti, <<El estudio>>, p. 219. Ovalle, \emph{Historia del fútbol}, pp. 26-33.}

En lo referente a los textos realizados por historiadores académicos, como ya se ha
mencionado, éstos son escasos, aunque en número creciente.\footnote{En México, los primeros trabajos académicos sobre historia del deporte fueron realizados por William Beezley en 1983. Beezley señala que los deportes surgen en México durante el periodo porfirista como derivación del proceso de modernización. Según Beezley, la modernización impulsó la economía y junto con la estabilidad política generaron una sensación de progreso que hizo que la sociedad mexicana adoptara las prácticas y los entretenimientos extranjeros, entre ellos los deportes. Beezley al ser uno de los primeros en historiar los deportes en México, se ha convertido en un referente para otros académicos que han hecho suyos algunos de sus planteamientos. Beezley, <<El estilo porfiriano>>, pp. 265-284. Beezley, \emph{Judas}, pp. 9-55.} Sin embargo, una gran
cantidad de esos trabajos son tesis de grado (licenciatura, maestría y doctorado) que no se
publican, por lo que tener acceso a ellas resulta difícil y por tanto sus aportes tienen poca
difusión y resonancia.\footnote{Las tesis sobre el fútbol podrían alcanzar el centenar. Angelotti, <<El estudio>>, pp. 215, 219.} En cuanto a los libros publicados que analizan la historia del fútbol en México, son pocos los trabajos que han visto la luz, de los cuales tres son los que más se
destacan y los que más se toman como referencia.\footnote{Según Luis González, los trabajos cuyos aportes son intrascendentes ya que sólo repiten los puntos de vista de otros investigadores no es necesario reseñarlos, ni analizarlos a profundidad, sino que se pueden <<dejar de lado sin gran inconveniente.>> Por esa razón, la revisión bibliográfica aquí presentada sólo hace referencia a los tres textos que a la fecha más aportaciones han realizado al estudio histórico y social del fútbol en México. González, \emph{El oficio de historiar}, p. 193.}

Uno de esos trabajos es el de Fernando Huerta titulado \emph{El juego del hombre. Deporte
y masculinidad entre obreros}, es un estudio antropológico enfocado en explicar cómo el
fútbol y el béisbol se desarrollaron como espacios donde los obreros de la Volkswagen de
Puebla refuerzan su masculinidad por medio de la práctica deportiva, la cual se establece
como un juego ritual con un amplio simbolismo de la clase obrera.\footnote{Huerta Rojas, \emph{El juego del hombre}, pp. 13-274.}

Otro de los textos destacados es el de Andrés Fábregas \emph{Lo sagrado del rebaño. El
fútbol como integrador de identidades}, donde explica cómo el fútbol se ha convertido en un
elemento aglutinador de los diversos colectivos de una sociedad y cómo el fútbol permite a
los individuos expresar a través de los equipos de fútbol, diferentes aspectos de la cultura y toma como ejemplo a las <<chivas rayadas>>, equipo que se considera como el verdadero representante del fútbol mexicano por ser el único que alinea futbolistas nacionales.\footnote{Fábregas, \emph{Lo sagrado del Rebaño}, pp. 10-100.}

Similarmente, Gabriel Angelotti en su libro \emph{Chivas y tuzos. Íconos de México,
identidades colectivas y capitalismo de compadres en el fútbol nacional}, explora cómo se
han entretejido redes sociales, políticas y comerciales en torno a los equipos del fútbol
mexicano. Angelotti establece que los clubes de fútbol en la actualidad se han consolidado
como empresas comerciales gracias a que tienen relación cercana con los gobiernos locales, además, reiteradamente han hecho uso de la historia para hacer que los aficionados concurran a los estadios y compren todo tipo de artículos con los colores del equipo.\footnote{Angelotti, \emph{Chivas y tuzos}, pp. 173-370.}

Todos estos textos tienen en común que analizan el fútbol a partir de la construcción
de identidades, pero, además, comparten un mismo punto de partida, es decir, toman como
referencia a las crónicas que abordan el tema del surgimiento del fútbol \emph{soccer} en México, a
pesar de que dichas crónicas tienen más rupturas que continuidades y que no cuentan con las
suficientes evidencias para confirmar la veracidad de sus postulados. Historiográficamente
hablando, esto representa un gran problema, pues a excepción de Gabriel Angelotti, ningún
académico ha cuestionado los planteamientos de las crónicas en más de medio siglo, por el
contrario, se siguen considerando fuentes indispensables para conocer el origen del fútbol en
México y por reiteración, se han convertido en los cismas historiográficos de donde parten
los estudios académicos.\footnote{Luis Carlos Ovalle, Andrés Fábregas, Richard V. McGehee (entre otros), han aceptado sin objetar los planteamientos que pretenden explicar el surgimiento del fútbol sin ofrecer mayores pruebas para sustentar sus dichos, siendo que el oficio de historiador, exige un compromiso con la verdad, por lo que antes de aceptar cualquier postulado, se hace necesario reunir datos, hacer una crítica de fuentes y análisis hermenéutico a sus testimonios a fin de llenar los vacíos existentes, evitar los anacronismos, las falsas interpretaciones y principalmente las deformaciones dolosas. Ovalle, \emph{Historia del fútbol}, pp. 40-57. McGehee, ``Mexico'', p. 490. Angelotti, \emph{La dinámica del fútbol}, pp. 27-44. Angelotti, <<El origen del fútbol>>, pp. 1-23. González, \emph{El oficio de historiar}, pp. 50-103. Fábregas, \emph{Lo sagrado del rebaño}, pp. 1-9.}

Las crónicas han tratado de ubicar en qué lugar surgió el fútbol \emph{soccer} con el único
objetivo de establecer a sus regiones y ciudades como la cuna del fútbol mexicano (el lugar
dónde por primera vez se jugó un partido de fútbol con apego a las reglas y dónde
complementariamente se fundó un club).\footnote{Marc Bloch señala que, al indagar en los orígenes de una actividad o práctica, se suele caer en el error de hablar de fechas en lugar de las causas (el cómo y por qué) y en la historia del fútbol mexicano esta ha sido una constante. Bloch, \emph{Apología para la historia}, p. 59.} La búsqueda de los orígenes del fútbol mexicano
se ha convertido en una carrera donde varios grupos pretenden obtener para sus ciudades el privilegio de ser el sitio que vio nacer al fútbol mexicano, sin embargo, más que ofrecer certezas, sólo han construido mitos en torno al surgimiento del fútbol mexicano que lejos de despejar las dudas sobre sus orígenes, las han incrementado.\footnote{Los cronistas, ---dice Luis González--- son por lo general inexpertos como investigadores, pero cuentan con aptitudes para la narrativa y dentro del ámbito del fútbol, existen gran cantidad de casos. González, <<Silvio Zavala>>, pp. 7-19.}

\section*{\mdseries\large\textsc{El surgimiento del fútbol mexicano: un análisis revisionista a sus orígenes}}
\addcontentsline{toc}{section}{El surgimiento del fútbol mexicano: un análisis revisionista a sus orígenes}

\noindent ¿En qué lugar de la República mexicana surgió el fútbol? Pareciera que la respuesta a esta
interrogante es por todos conocida y que con certeza se sabe dónde y cuándo rodó por primera
vez un balón de fútbol en México, sin embargo, cuatro ciudades (Pachuca, Real del Monte,
Orizaba y la Ciudad de México) afirman ser la cuna del fútbol mexicano, aspecto que ha
generado una controversia que a la fecha sigue vigente. ¿Por qué cuatro ciudades dicen ser
la cuna del fútbol mexicano? ¿Por qué hasta ahora no ha sido posible discernir en qué lugar
de México surgió el fútbol \emph{soccer}? La respuesta a estas interrogantes tiene que ver con dos
aspectos: 1) la forma en cómo se han construido los argumentos, 2) los objetivos que
persiguen. Sobre el primer punto, se observa una escasa y deficiente búsqueda de
información, nula crítica de fuentes, revisión y comprobación de datos.

También se observa una total carencia de análisis y afirmaciones probadas por
acumulación de anécdotas y testimonios orales. Asimismo, donde no se encuentra
información, se llenan esos vacíos con relatos míticos y finalmente, hay una distorsión de
resultados, ya que pretenden generalizar los alcances de sus conclusiones, pues a pesar de
que se indaga en una localidad, sus afirmaciones se extrapolan (como verdades irrefutables
y universales) a otras áreas y regiones donde el fútbol todavía no se ha historiado. Sobre el segundo punto, pareciera que se indaga en la historia del fútbol mexicano, no para desarrollar
nuevos conocimientos, sino que el motivo que los impulsa es el de construir un legado
cultural (dolosamente distorsionado) que justifique la apropiación y usufructo de la historia
del fútbol, tal cual lo hace el Club Pachuca en la actualidad.\footnote{Con el fútbol se pretende realizar lo que se hacía con las reliquias de los santos; se busca convertir una región o ciudad en un lugar privilegiado y a la vez en custodio de un bien, un objeto o pasado glorioso que genera un sentimiento de pertenencia y orgullo y donde además se fundan sitios de culto y peregrinaje. Se pretende que la ciudad de Pachuca se distinga de otras plazas futboleras no sólo por los logros deportivos que ha alcanzado el equipo de la localidad (el \emph{Club Pachuca}), sino también, por ser la cuna del fútbol mexicano, que, dicho sea de paso, ya cuenta con un recinto de culto y peregrinaje: el \emph{Salón de la fama del fútbol}. El artífice de toda esta propaganda ha sido el \emph{Club Pachuca}, que sistemáticamente ha utilizado la historia del fútbol mexicano como campaña publicitaria para generar un sentido de pertenencia en torno a la ciudad y al equipo, para convertirlos en referentes identitarios de los pachuqueños y así justificar la trascendencia histórica y la <<existencia institucional>> del \emph{Club Pachuca}. Aguilar, <<Entre la verdad y la mentira>>, pp. 13-32. Angelotti, <<El origen del fútbol>>, pp. 1-23.}

Si todas las versiones están mal elaboradas y además son tendenciosas ¿Cómo
dilucidar dónde surgió el fútbol \emph{soccer}? ¿Cómo saber cuál de ellas es la versión verdadera?
En primer lugar, lejos estamos de poder ubicar con certeza dónde se localiza la cuna del
fútbol mexicano, pues a la fecha sólo se ha consultado la información de algunas ciudades y
estados, quedando una gran cantidad de fuentes sin revisar, por tanto, a medida que nueva
información aparezca, se modificará sustancialmente la historia del fútbol mexicano no sólo
en lo referente a sus orígenes, sino también a sus causas y consecuencias.

En segundo lugar, para dar respuesta a estas interrogantes es necesario realizar un
ejercicio revisionista utilizando el paradigma indiciario para conocer los verdaderos alcances
y relevancia de cada versión. El avance de la Historia se suscita cuando nuevos datos son
descubiertos, pero también cuando los hechos que se narran son comprobados y en ese
sentido, se pretende comprobar la veracidad de los hechos, datos, fechas y planteamientos,
analizando sus detalles por medio del paradigma indiciario y contrastando las fuentes de cada versión, con otros datos y evidencias, que en última instancia nos permitan establecer en su justa dimensión, los aportes de cada una de las versiones.\footnote{Uno de los más recientes modelos de análisis utilizados por los historiadores es el método hermenéutico, un dispositivo de análisis de textos, necesario para <<comprender un texto en su contexto>>, pues se debe tener en cuenta que todo conocimiento es construido, por tanto, la tarea de la hermenéutica es explicar cómo se ha producido la construcción de este conocimiento y sus interpretaciones derivadas. Este modo de trabajar la Historia se conoce como deconstruccionista. Moreno, <<La investigación empírica>>, pp. 72-87. Booth, ``Theory'', pp. 12-34.}

En 1960 Juan Cit y Mulet publicó \emph{El libro de oro del fútbol mexicano} y donde
establecía que en el año de 1900 se fundó el \emph{Pachuca Athletic Club}, el primer club del fútbol
mexicano.\footnote{Cit y Mulet señala que <<Técnicos y mineros ingleses del grupo de la Cía. Real del Monte de Pachuca, en la capital del Estado de Hidalgo, fueron quienes allá por el año de 1900, formaron el primer equipo de fútbol soccer de la República mexicana>>. Sin embargo, no ofrece mayores datos acerca de cómo fue que el fútbol comenzó a practicarse ni qué motivos impulsaron a los británicos para practicar el fútbol \emph{soccer} en México. Cit, \emph{El libro de oro del fútbol}, pp. 9-16.} A partir de la publicación del libro de Cit y Mulet, Pachuca comenzó a ser
considerada como la cuna del fútbol mexicano, ya que los que han escrito después de él han
repetido constantemente sus dichos.\footnote{El comentarista deportivo Heriberto Murrieta, señaló en una entrevista que, <<históricamente le asisten más datos a la ciudad de Pachuca como cuna del fútbol>>, porque según había leído, fue en la zona minera del Estado de Hidalgo donde comenzó a practicarse el fútbol, <<Por eso dicen que esa es la cuna del fútbol mexicano>>. Al igual que Murrieta muchos comentaristas deportivos y también muchos académicos, siguen esa lógica y aceptan sin menor objeción que el fútbol \emph{soccer} surgió en la ciudad de Pachuca. Barrón, <<México>>, pp. 93-100.} Sin embargo, no todos han aceptado las afirmaciones
de Cit y Mulet y sus seguidores, por lo que también han surgido otras versiones que han dado
origen a una controversia respecto al surgimiento del fútbol mexicano.

La primera de las versiones que comenzó cuestionar la hegemonía de Pachuca como
la cuna del fútbol fue la ciudad de Real del Monte (una población minera localizada a 15 km de Pachuca) y donde se afirma que allí ---verdaderamente--- se jugó al fútbol en México antes de emigrar a Pachuca.\footnote{Diversas voces consideran que fue Real del Monte, el lugar dónde por primera vez se jugó al fútbol \emph{soccer} y dónde se formó el primer club de fútbol con toda formalidad, porque fue el primer sitio donde hubo un asentamiento británico, aunque posteriormente el equipo se trasladó a la ciudad de Pachuca. Sin embargo, no hay documentos o pruebas que sustenten ninguno de estos relatos. Angelotti, \emph{Chivas y tuzos}, p. 243. Calderón, <<Orígenes (\textsc{iii})>>, pp. 1-8. Ricardo Olivares, <<Real del Monte... origen del fútbol mexicano>> \emph{La Prensa}, 13 de junio 2010, \url{http://www.oem.com.mx/laprensa/notas/n1670824.htm}} También se ha mencionado que el fútbol ya se practicaba en Real del Monte antes de 1900.\footnote{El exminero Juan Moreno señala que en la mina \emph{Dolores} se disputó <<el primer partido de fútbol en nuestro país>>, en una fecha no determinada, pero sí muy anterior a 1900, año de la fundación del \emph{Pachuca Athletic Club}. Este relato le fue transmitido de forma oral por su padre Eulogio Moreno, quien desde 1939 se desempeñaba como minero. J. C. Vargas, <<Entre minas y fútbol>>, \emph{Excélsior}, 13 de agosto 2013, \url{http://www.excelsior.com.mx/adrenalina/2013/08/13/913392}} Por otra parte, la falta de evidencias no ha sido una limitante para señalar que Real del Monte es el lugar de origen del fútbol mexicano,
por el contrario, tal y como señala Gabriel Angelotti, la falta de pruebas ha permitido remontar el origen del fútbol a fechas muy tempranas. Por ejemplo, el cronista Enciso Vargas,
dice contar con información hemerográfica del año de 1889, donde se menciona que el fútbol \emph{soccer} comenzó a ser practicado en Real del Monte <<desde mediados del siglo \textsc{xix}>>.\footnote{A la fecha el cronista Enciso Vargas no ha presentado la información que dice poseer. Angelotti, \emph{Chivas y tuzos}, p. 248.}

La falta de crítica y de revisión a la historia del fútbol mexicano, ha propiciado que
relatos fuera de toda lógica se mantengan vigentes, a pesar de que en ellos se observa un total
desconocimiento de la historia del fútbol en general, pues se debe tener presente que la
\emph{Football Association} (\textsc{fa}), el organismo de donde surge el fútbol \emph{soccer}, se fundó hasta 1863 en la \emph{Freemason's Tavern} de Londres. Además, tal y como señala Gavin Kitching, el fútbol
\emph{soccer} no tuvo un impacto inmediato en la sociedad británica, sino que fue necesario por lo
menos una década para que se difundiera por todo el Reino Unido y se estableciera, junto
con el \emph{rugby}, como uno de los estilos de fútbol preponderantes, por lo que la versión que
señala que el fútbol \emph{soccer} llegó a Real del Monte a mediados del siglo \textsc{xix}, tiene un desfase
de por lo menos veintitrés años.\footnote{También se debe tener en cuenta que la configuración del fútbol \emph{soccer} fue un proceso paulatino, por ejemplo, la aparición del portero se suscitó en 1871, en 1873 se implementa el tiro de esquina, en 1874 se adjunta el travesaño a las porterías y en 1890 surge el penalti. Por tanto, si se hubiera disputado un partido de fútbol en Real del Monte en 1850, sería de un estilo muy diferente al \emph{soccer}. Kitching, ``Old football'', pp. 1736-1738. Ovalle, \emph{Historia del fútbol}, p. 12.}

Por otra parte, la ciudad de Orizaba también pretende ser considerada como la cuna
del fútbol mexicano, señalando que fue a través del puerto de Veracruz que el fútbol ingresó
a nuestro país, por lo que se supone que los primeros partidos se disputaron en tierras
veracruzanas (entre ellas Orizaba) y no en la zona minera del estado de Hidalgo. Los británicos arribaron a Orizaba para trabajar en la fábrica \emph{Santa Gertrudis}, donde se organizó un club deportivo, el \emph{Orizaba Athletic Club} y cuya fecha de fundación ha sido el principal motivo de debate, pues algunos consideran que el fútbol comenzó a practicarse desde el establecimiento del club en {1898}.\footnote{Algunos consideran que el Orizaba Athletic Club es el más antiguo del fútbol mexicano, porque en sus uniformes el escudo tenía bordado el año de 1898. También se señala que el campeonato ganado por el \emph{Orizaba Athletic Club} en 1902, es indicativo de que el fútbol ya se practicaba en esta población desde fechas tempranas. Ovalle, \emph{Historia del fútbol}, pp. 56-66.} Pero también se menciona que el fútbol comenzaría a
practicarse algunos años después de la fundación del club (hasta 1901).\footnote{Juan Cit y Mulet menciona que Duncan Macomish, trabajador de la fábrica \emph{Santa Gertrudis}, tuvo la idea de formar un equipo de fútbol que se denominaría como \emph{Orizaba Athletic Club} y que esto ocurrió a fines de 1901 y principios de 1902. Carlos Calderón por su parte señala que en 1898 el \emph{Orizaba Athletic Club} no practicaba el fútbol, sino el \emph{cricket} y el béisbol. Cit, \emph{El libro de oro}, p. 36. Calderón, <<Orígenes (\textsc{ii})>>, pp. 1-5.}

Tratando de finalizar con la controversia, en el año 2009 el presidente del actual
equipo \emph{Albinegros de Orizaba}, Fidel Kuri Grajales, en conferencia de prensa <<defendió a Orizaba como la auténtica cuna del fútbol e indicó que el \emph{Orizaba Athletic Club} se fundó en 1898, dos años antes que el equipo hidalguense (\emph{Pachuca Athletic Club}), al que le apostó un millón de pesos, para quien diga la contrario.\footnote{Emilio González, <<Fidel Kuri defiende a Orizaba como la cuna del fútbol>>, \emph{El Sol de Orizaba}, 4 de marzo 2009, \url{http://www.oem.com.mx/esto/notas/n1070593.htm}}

El señor Kuri Grajales agregó que el \emph{Orizaba Athletic Club} desde 1898 tenía tanto
equipo de \emph{cricket} como de fútbol, afirmación que fue avalada por varios cronistas e
historiadores deportivos (no se menciona quiénes eran) y por último el señor Kuri Grajales y
demás acompañantes, visitaron los terrenos de la antigua fábrica \emph{Santa Gertrudis} donde se presume se efectuó el primer partido de fútbol y tras esta visita se concluyó que Orizaba era sin lugar a dudas <<la Cuna del fútbol y no Pachuca>>\footnote{Emilio González, <<Fidel Kuri defiende a Orizaba como la cuna del fútbol>>, \emph{El Sol de Orizaba}, 4 de marzo 2009, \url{http://www.oem.com.mx/esto/notas/n1070593.htm}}

Aunque el presidente del actual \emph{Club Orizaba} pretendió finalizar con las
controversias, todo se mantuvo igual, ya que no se presentó ningún documento que
confirmara la veracidad de las afirmaciones (al menos no se menciona ninguno), sólo se
presentaron (se reiteraron) algunos testimonios orales ya conocidos. Además, los alcances de
la versión de Orizaba sólo pueden remontarse hasta la fundación de la fábrica \emph{Santa Gertrudis}
(1897) que fue la fuente que dio origen al \emph{Orizaba Athletic Club}.\footnote{García, <<Santa Gertrudis>>, pp. 207-225. Ribera, <<Moviendo telares>>, pp. 1-33.}

En la actualidad, tanto la versión de Orizaba, así como la de Real del Monte, dejan de
tener relevancia como posibles cunas del fútbol mexicano, pues gracias a la digitalización de
archivos, se han localizado nuevos datos y evidencias que indican que, en otras poblaciones
del país (como la ciudad de México), el fútbol ya se practicaba desde fechas previas a las
propuestas por Orizaba y Real del Monte.

La aparición de nueva información ha permitido precisar con mayor certeza desde
cuándo comenzó a practicarse el fútbol \emph{soccer} en la ciudad de México y principalmente cómo
fue que este deporte comenzó a practicarse en la capital del país. La ciudad de México por
ser el centro político y económico del país, fue una de las ciudades donde se comenzaron a
introducir y practicar los deportes, ya que fue el lugar donde muchos extranjeros
establecieron su lugar de residencia y las oficinas de sus negocios.\footnote{En 1882 se celebró uno de los primeros partidos de béisbol y en 1892 el primer concurso atlético. ``Baseball'', \emph{Two Republics}, 6 de agosto de 1882, p. 3. ``The program crystalizes for its celebration'', \emph{Daily AngloAmerican}, 22 de Junio 1892.}

Respecto al surgimiento del fútbol en la ciudad de México existen diversas versiones,
por ejemplo, Robert Blackmore (uno de los pioneros del fútbol mexicano), establece que el
fútbol \emph{soccer} comenzó a ser practicado en el \emph{Reforma Athletic Club}.\footnote{Según Luis Celay presidente del \emph{Club Reforma}, el primer partido de fútbol celebrado en la capital del país se disputó en el \emph{Reforma Athletic Club} en 1901, por lo que el \emph{Club Reforma} es la cuna del fútbol en la ciudad de México. Robert Blackmore, Revista \emph{Récord}, 1942, citado en Angelotti, \emph{La dinámica del fútbol}, p. 27. María del Refugio Melchor, <<Club Reforma, cuna del fútbol en la ciudad de México>>, \emph{El Financiero}, 19 de mayo 2014, \url{http://www.elfinanciero.com.mx/after-office/club-reforma-cuna-del-futbol-en-la-ciudad-de-mexico.html}} Por otra parte, en el
libro de Carlos Ramírez se indica que desde 1897, los alumnos de los colegios maristas y
jesuitas comenzaron a practicar el fútbol \emph{soccer} en la ciudad de México.\footnote{Ramírez, ¿Cuál es la historia?, p. 11.}

En ese mismo sentido, Carlos Calderón menciona (sin precisar fechas) que algunos
de los primeros balones de fútbol que arribaron al país, se distribuyeron en varios colegios ingleses de la ciudad de México, donde <<se intentó con poco éxito organizar algunos encuentros porque eran escasos los interesados. Cardozo también cita que en la ciudad de México en el año de 1896 había entre 30 y 40 personas que practicaban el fútbol, sin embargo,
Calderón no considera que la ciudad de México sea la cuna del fútbol mexicano, porque a su juicio, estas personas  solo <<cascareaban>>, es decir, practicaban el fútbol sin mucho apego a las reglas y más con sentido lúdico que competitivo.\footnote{Calderón, \emph{Pachuca la cuna del fútbol}, p. 13.}

Sin embargo, la información encontrada recientemente muestra que el fútbol se
comenzó a practicar en la ciudad de México desde 1892, cuando se organizó un partido con
motivo de la inauguración del \emph{Mexican Athletic Club}. La nota menciona que Mr. McAusland
había formado un equipo que había estado practicando diariamente y que se enfrentaría con el equipo del \emph{Mexican Athletic Club} el próximo 2 de octubre y que ese encuentro sería el
primer partido celebrado en la ciudad jugado entre dos diferentes clubes.\footnote{``Football'', \emph{Two Republics}, 21 de septiembre 1892, p. 4.}

En un principio, la práctica del fútbol en la ciudad de México fue irregular en función
de que la comunidad británica radicada en la capital no era tan numerosa y al no existir los
suficientes individuos para conformar equipos, el fútbol no podía despuntar, por esa causa,
el \emph{Mexican Sportman} buscaba conocer cuántos futbolistas había en la ciudad y qué estilo
practicaban (si el \emph{rugby}, el \emph{soccer} o ambas). A pesar de que el fútbol no se había consolidado,
el \emph{Mexican Sportman} recalcaba que en la ciudad había un significativo número de devotos
practicantes y un caso representativo eran los escolares ingleses que en ocasiones formaban
algún equipo que desafortunadamente se desbandaba porque no había una competencia que
los mantuviera unidos.\footnote{``Football'', \emph{Mexican Sportman}, 10 de octubre 1896, p. 2, 3.}

Posteriormente en esa misma revista se publicó una carta del señor Geo McLellan
donde informaba que junto con Louis Lubens y Mr. Mohler pensaban formar un club de
fútbol. Se menciona que todos estos personajes practicaban indistintamente el \emph{rugby} y el
\emph{soccer}, pero consideraban que el \emph{soccer} era el estilo que podría tener más éxito, porque se
adaptaba a las condiciones físicas y a la idiosincrasia de la sociedad mexicana.\footnote{``Football'', \emph{Mexican Sportman}, 24 de octubre 1896, p. 3.}

Sin embargo, los planes de fundar este club no se concretaron, porque la práctica
futbolística comenzaría a ser estigmatizada como bárbara y violenta, por causa de la
celebración del primer partido de fútbol americano, evento que generó en la prensa nacional y en la sociedad mexicana, una percepción negativa de toda actividad con el membrete de
fútbol, aspecto que demoró la consolidación del fútbol \emph{soccer} en la ciudad de México.\footnote{En 1896, ni la prensa, ni la sociedad mexicana estaban muy compenetrados con los diversos tipos de fútbol, así que recurrentemente se confundía el \emph{soccer} con el americano o el \emph{rugby}. Por ejemplo, el \emph{Mexican Herald} sin considerar que se trataba de estilos diferentes, publicó que <<el fútbol en los Estados Unidos era uno de los deportes de moda y en Inglaterra era el segundo después del \emph{cricket}>>. ``The football games'', \emph{Mexican Herald}, 17 de diciembre 1896, p. 8.}

A fines de 1896 y principios de 1897 los equipos de las Universidades de Texas y
Missouri disputaron tres partidos de fútbol americano, un deporte jamás visto en México, por
lo que un estimado de dos mil personas se hicieron presentes en el hipódromo de Indianilla,
los estadounidenses para revivir escenas que les eran familiares, mientras que los mexicanos
por la curiosidad de presenciar algo nuevo y ver de qué se trataba.\footnote{``Crescent victorious'', \emph{Mexican Herald}, 28 de diciembre 1897, p. 1.}

En consideración de los estadounidenses, el primer partido de fútbol americano fue
un completo éxito y tuvo efectos positivos, ya que motivó a los empleados de los ferrocarriles
a que formaran equipos.\footnote{``Football in Mexico'', \emph{Mexican Herald}, 28 de diciembre 1896, p. 4. ``Football yesterday'', \emph{Mexican Herald}, 30 de diciembre 1896, p. 1.} En cambio, para la sociedad mexicana, el fútbol americano era
brutal y violento, más que los toros y el boxeo, porque prácticamente el juego se transformaba
en un campo de batalla donde los jugadores saltaban uno sobre otro, formando <<un cuadro desagradable>> y donde se hacía necesario contar con la presencia de médicos y ambulancias.\footnote{``An innovation'', \emph{Mexican Herald}, 3 de diciembre 1896, p. 5. ``Football'', \emph{Two Republics}, 27 de diciembre 1896, p. 4. ``The football game'', \emph{Mexican Herald}, 30 de diciembre 1896, p. 5.} En síntesis, el fútbol americano no fue considerado apto para la sociedad mexicana, porque no concordaba ni con la complexión física ni con el temperamento de los mexicanos.\footnote{Beezley, \emph{Judas}, pp. 52-57. <<Estudiantes Americanos>>, \emph{El Mundo}, 29 de diciembre 1896, p. 1.}

Aunque en más de una ocasión se mencionó que la rudeza del fútbol americano no
era intencional, la sociedad mexicana consideraba que el fútbol (en todos sus estilos, incluido
el \emph{soccer}) era algo violento, por tanto, los subsecuentes intentos de formar algún equipo no
pudieron cristalizarse. Sin embargo, los ánimos por desarrollar el \emph{soccer} no desvanecieron
pues en la ciudad había varios futbolistas que se encontraban ansiosos por formar un club.\footnote{A fines de julio de 1897 los señores R. H. Gill, Louis Loubens y Fred Stein intentaron fundar un equipo y además buscaban organizar partidos con equipos de otros estados. ``Passing day'', \emph{Mexican Herald}, 2 de enero 1897, p. 1. ``Football'', \emph{Mexican Sportman}, 2 de enero 1897, p. 8. ``Football'', \emph{Mexican Sportman}, 30 enero 1897, p. 3. ``Difference in brutal sports'', \emph{Two Republics}, 16 de febrero 1897, p. 4. ``City briefs'', \emph{Mexican Herald}, 29 de julio 1897, p. 5. ``New football team'', \emph{Two Republics}, 29 de julio 1897, p. 8.}
Ya para 1900 la práctica del fútbol se vuelve más consistente y organizada, un ejemplo de ello fue el equipo <<Los gringos>> integrado por escolares y cuyo director era F. W. Bone, su manager H. Dewey y capitán A. W. Malcomsom\footnote{El club \emph{Gringos}, estaba integrado por escolares que estaban de vacaciones en la ciudad de México. Sus prácticas las realizaban en el Paseo de la Reforma y también se menciona que el fútbol se estaba popularizando, ya que en breve se formaría otro equipo que competiría contra los \emph{Gringos}. ``Football team'' \emph{Mexican Herald}, 22 de julio 1900, p. 16. ``Another football team to be organized'', \emph{Mexican Herald}, 22 de Agosto 1900, p. 8.}

El fútbol en la ciudad de México se consolidó en 1901 con el surgimiento de los
equipos \emph{British Club} y \emph{Reforma Athletic Club}, que tuvieron una serie de duelos que marcarían
un parteaguas en la historia del fútbol, ya que de esta serie de partidos se organizaría la
primera liga de fútbol \emph{soccer} del país. Según la prensa, el primer partido disputado entre el \emph{British Club} y el \emph{Club Reforma} tuvo un buen inicio, pues terminó <<sin serios inconvenientes>> y sin << huesos rotos>> porque el estilo de fútbol que se practicó fue el futbol \emph{soccer} uno de los más emocionantes para los espectadores y menos peligroso para los practicantes.\footnote{El partido terminó un gol a cero en favor del \emph{Club Reforma}, aunque algunos pensaron que fue empate, esto porque un tiro del \emph{British Club} pasó por encima del travesaño, pero las reglas del \emph{soccer} indicaban que debía pasar por debajo. Al final del partido el \emph{British Club} pidió la revancha, pero bajo el formato del \emph{rugby}. ``Sunday’s sports'', \emph{Mexican Herald}, 3 de junio 1901, p. 2.}

Aunque se solicitó jugar la revancha bajo el estilo del \emph{rugby}, los siguientes encuentros
que el \emph{British Club} y el \emph{Club Reforma} disputaron fueron en el estilo del \emph{soccer}, porque a los ojos de la prensa y de la sociedad mexicana era menos bárbaro y violento y porque requería once participantes, en lugar de los quince requeridos en el \emph{rugby}.\footnote{De forma reiterada se estuvo informando que el estilo que se iba a practicar era el \emph{soccer}. ``Sports look up'', \emph{Two Republics}, 22 de noviembre 1901, p. 12. ``Football Sunday'', \emph{Mexican Herald}, 12 de diciembre 1901, p. 2. ``Football games'', \emph{Mexican Herald}, 12 de diciembre 1901, p. 8. ``Football'', \emph{Mexican Herald}, 22 de diciembre 1901, p. 12. ``Sunday football'', \emph{Mexican Herald}, 16 de enero 1902, p. 5. ``The Reformas win'', \emph{Mexican Herald}, 20 de enero 1902, p. 7. ``Football teams organized'', \emph{Mexican Herald}, 09 de octubre 1902, p. 5.} La revisión realizada a la historia del fútbol de la ciudad de México nos muestra que en 1901, el fútbol \emph{soccer} comenzaba a tener gran auge, sin embargo, su práctica se inició en el año de 1892, por lo
tanto, la versión de la ciudad de México antecede a las de Pachuca, Real del Monte y Orizaba.

\section*{\mdseries\large\textsc{Pachuca la cuna del fútbol mexicano ¿Una historia distorsionada?}}
\addcontentsline{toc}{section}{Pachuca la cuna del fútbol mexicano ¿Una historia distorsionada?}

\noindent Luego de revisar las versiones de Real del Monte, Orizaba y la ciudad de México, pasamos
a revisar la versión de Pachuca, por muchos considerada como la principal de todas las
versiones, no porque ofrezca mayores certezas que las demás, sino que ha sido la más
difundida, ya que, de forma sistemática y a través de diversos medios se ha intentado
establecer a Pachuca (por reiteración) como la cuna del fútbol mexicano.\footnote{Páginas de \emph{internet}, revistas, programas de radio y televisión han sido utilizados para señalar que Pachuca es
la cuna del fútbol mexicano. Angelotti, <<El origen del fútbol>>, p. 8,9.}

En efecto, sin escatimar gastos, la actual administración del \emph{Club Pachuca} ha
pretendido imponer la versión de Pachuca sobre cualquier otro relato y narrativa,
principalmente por medio de la publicación de libros y artículos, donde con lujo de detalle
se narran los orígenes del fútbol mexicano a partir de la historia del \emph{Club Pachuca}.\footnote{Según Gabriel Angelotti, la frase <<Pachuca cuna del fútbol>> es más un \emph{slogan} comercial que una realidad histórica fehacientemente comprobada, pero a fuerza de reiterarla, se ha convertido en una <<verdad>> que acalla a sus detractores. La versión de Pachuca es un relato mítico (con mucho de ficción y poco de verdad) que se busca convertir en hecho histórico pues, aunque no se tienen evidencias claras de su autenticidad, se hace pasar como verídico por medio de la impostura, que, a decir de Marc Bloch, era el veneno más virulento que contamina los testimonios. Angelotti, \emph{La dinámica del fútbol en México}, p. 28. San Miguel, <<Mito e historia>>, pp. 133-156. Bloch, \emph{Apología para la historia}, p. 105.} Estos
trabajos pretenden ajustarse a los cánones académicos, sin embargo, sus argumentaciones son parciales y selectivas porque tendenciosamente y a conveniencia se eligieron algunos hechos <<mientras que otros se olvidaron u omitieron>>\footnote{Angelotti, \emph{Chivas y tuzos}, p. 242.}

Lo que se observa es que la búsqueda de los orígenes del fútbol mexicano no ha sido
desinteresada, pues intencionalmente se ha forzado a las fuentes para establecer a Pachuca
como el (único) lugar de origen del fútbol mexicano, ya que se narran y describen de forma
secuenciada diversos hechos y donde se pretende evidenciar las acciones de ciertos
personajes, para exhibir que, en Pachuca, desde un principio y en todo momento, se tuvo la
intención de implantar el fútbol \emph{soccer}. Sin embargo, esta forma de proceder contiene
múltiples errores metodológicos, pues se confunde concatenación de hechos con su
explicación, de igual forma, resulta erróneo recurrir a las intenciones para explicar las acciones de los personajes, ya que <<No todos los acontecimientos en los que interviene el hombre han sido planeados. Muchas veces los planes son modificados por las circunstancias (y) en otras circunstancias es imposible dar con el diseño original de las acciones>>.\footnote{Tanto Marc Bloch como Luis Gonzáles señalan que el sentido de la Historia no es presentar explicaciones teleológicas (dar a conocer los propósitos de los personajes y su relación con los hechos), sino el de realizar investigaciones causales que expliquen cómo y por qué se suscitan los hechos. Bloch, \emph{Apología para la historia}, p. 62. González, \emph{El oficio de historiar}, p. 52.}

Por ejemplo, Carlos Calderón señala que <<El lugar elegido ---azar o destino--- para ser
la cuna del fútbol en México fue ni más ni menos que una hermosa ciudad enclavada en un estado minero por tradición: el Estado de Hidalgo. La ciudad conocida como la \emph{Bella Airosa}, daría fruto al primer equipo constituido oficialmente: Pachuca.\footnote{Calderón, \emph{Pachuca la cuna del fútbol}, p. 13.} Calderón pretende explicar el origen del fútbol mexicano a partir de un planteamiento determinista, pues pareciera que sólo hizo falta que los mineros ingleses arribaran a Pachuca para que el fútbol surgiera y comenzara a practicarse, sin embargo, no considera que los británicos arribaron a distintos lugares del país y se emplearon en diversas actividades, ni tampoco explica qué representaba el fútbol para los británicos y cómo y por qué era practicado.

En efecto, no todos los británicos que migraron a México eran ingleses y no todos se
asentaron en Hidalgo y se dedicaron a la minería, sino que también arribaron galeses,
irlandeses y escoceses que se desempeñaron como empresarios, comerciantes, banqueros,
empleados, obreros, ingenieros, administradores, contratistas, diplomáticos, técnicos
industriales, textileros, ferrocarrileros, agentes comerciales, viajeros, profesores, médicos,
artistas (músicos y bailarines), petroleros, misioneros, químicos y marinos.\footnote{Alatriste, <<Aspectos económicos>>, pp. 101-148.}

Es decir, los expatriados británicos (al igual que los estadounidenses) no eran un
grupo homogéneo, sino una <<diáspora>> ya que no eran un grupo uniforme, pues entre ellos había diversidad de clases, razas, oficios e incluso nacionalidades.\footnote{Schell, \emph{Integral}, pp. IX-18.} Lo diverso de la comunidad británica asentada en México, hizo necesaria su vinculación a instituciones como escuelas, iglesias, clubes y prácticas (como las fiestas cívicas y los deportes), para reforzar su identidad nacional, socializar entre sí y establecer lazos de unión directas (amistad, matrimonios) e indirectas (comerciales).

Los británicos encontraron en el fútbol una forma de socializar y de reforzar su
identidad, pues el fútbol en palabras de Eric Dunning, ha sido un espacio que permite manifestar y exhibir <<sentimientos colectivos>> de un grupo que se identifica con la ciudad o la nación donde habitan o <<con un subgrupo concreto, como una clase social o la etnia>>. Los extranjeros que arribaron a México necesitaron encontrar y establecer nuevos foros sociales que les permitieran reforzar su identidad y mantenerse unidos e integrados, pues según Dunning, los individuos que migran o se dispersan, pueden caer en la <<soledad de la multitud>>, un aspecto que hace necesario que los individuos busquen un  nuevo espacio social donde relacionarse y socializar con personas con intereses afines a los suyos.\footnote{Richard Holt señala que, en el Reino Unido durante el siglo \textsc{xix}, a medida que crecían las ciudades se volvían espacios impersonales, principalmente, para los individuos que migraban del campo, por tanto, los nuevos habitantes de las ciudades requirieron forjarse una nueva identidad y establecer nuevos foros donde expresarla y reforzarla. Pérez-Rayón, <<Sociología>>, 1993, pp. 3-6. Holt, \emph{Sport}, p. 167. Dunning, \emph{Fenómeno}, 1999, p. 15.}

En México, para muchos británicos el fútbol fue ese espacio que permitió reforzar las identidades (de clase, raza, género, pero principalmente nacionales) integrando a los individuos que en otros ámbitos estarían fragmentados o enfrentados.\footnote{Andrés Fábregas señala que el fútbol cumple las mismas funciones integradoras que la religión, la plaza pública o la política. Fábregas, <<El fútbol en Chiapas>>, pp. 145-161.} El fútbol ofrece la oportunidad de romper con las barreras sociales y satisfacer la necesidad emocional de camadería y compañía, lo cual ha contribuido con la creación de un sentido de comunidad, aspecto vital para el establecimiento
y mantenimiento del imperio británico y de la \emph{Commonwealth}.\footnote{Tranter, \emph{Sport, economy and society in Britain}, pp. 37-56.}

El fútbol tiene la capacidad de hacer visibles a las minorías, formando <<comunidades imaginadas>>\footnote{Las comunidades imaginadas, según Benedict Anderson, están integradas por personas que sienten que comparten ciertos vínculos e intereses con otras que cohabitan en el mismo territorio, a pesar de no conocerse y no poder reunirse. Anderson, \emph{Comunidades imaginadas}, pp. 17-30. Ward, ``Sport and national identity'', pp. 518-531. Jarvie, Reid, ``Sport, nationalism and culture'', pp. 97-124.} de varios niveles (local, regional, nacional) que se expresan a través del fútbol por medio de diversos símbolos (banderas, escudos, himnos) y de sus expresiones físicas (estilos de juego, estereotipos). Es decir, el fútbol despierta los sentimientos de las personas por su clase, raza o nación y los aglomera alrededor de símbolos comunes, logrando que los individuos se identifiquen con los valores asociados a dichos símbolos, lo que finalmente produce que actúen como un mismo grupo, al menos por el tiempo que dura un partido.\footnote{Por ejemplo, la ceremonia de los himnos previo a un partido de fútbol, es una forma de poner en funcionamiento y de volver tangibles los sentimientos nacionalistas de los individuos por su patria. Tuck, Maguire, ``Making sense of global Patriotic Games'', pp. 26-54.}

En Escocia, por ejemplo, el nacionalismo se hace más evidente cuando los escoceses juegan un partido de fútbol contra el enemigo histórico: Inglaterra.\footnote{Bairner, \emph{Sport, nationalism, and globalization}, p. 65.} Para los escoceses, el fútbol forma parte de su destino, por esa razón lo han utilizado para <<hacer frente a la dominación política y económica de Inglaterra>> buscando con ello vencerlos en su propio juego, ya que los triunfos deportivos han sido interpretados como signos de progreso, pero,
principalmente como emblemas de emancipación, porque han sido elementos que han permitido a los escoceses <<liberarse>> simbólicamente del dominio y control inglés.\footnote{Según Elias y Dunning, el fútbol tiene la capacidad de reflejar los aspectos de la vida diaria, pero también permite evadirse de ella. El fútbol ofrece la oportunidad de caer y levantarse, por esa razón es tan exitoso y atractivo, pues en un partido, un equipo puede ir perdiendo y, sin embargo, al término del encuentro resultar vencedor. Esto produce una catarsis para los individuos que pocas veces tienen en su vida la oportunidad de saborear el triunfo y sentirse orgullosos de quiénes son, un aspecto de importancia crítica, porque la dinámica social puede ser revertida y los dominados pueden (por un momento y sólo dentro del ámbito del fútbol) volverse los dominadores. Elias, Dunning, \emph{Deporte y ocio}. pp. 154-159. Tuck, Maguire, ``Making sense of global Patriotic Games'', p. 30. Findlay, ``It’s a Dutch invention'', pp. 261-273.}

En el Reino Unido, en 1872 se celebró el primer partido entre Escocia e Inglaterra y con ello se estableció una rivalidad de carácter <<independiente>> que no sólo se disputaba en el Reino Unido, sino que también se ha disputado en todos los lugares donde se estableció una comunidad británica, como en México, donde escoceses e ingleses socializaron entre sí mediante el fútbol y la tensión que generaban las diferencias existentes entre ambos, principalmente las relaciones de poder y la identidad nacional.\footnote{La interdependencia según Dunning es un concepto de poder y de conflicto y se refiere a la interacción que los individuos construyen entre sí y donde socializan e intercambian bienes, información y sentimientos. En el caso del fútbol, la interdependencia es un factor indispensable para su desarrollo, porque en todo momento las acciones de un grupo afectan al otro. Dunning, \emph{Fenómeno}, 1999, pp. 28-31. Findlay, ``It’s a Dutch invention'', p. 263. Rookwood, Buckley, ``The significance of the olympic soccer'', pp. 6-15. Leese, ``Illustrating the Auld Enemies'', pp. 183-199.}

En México, el primer partido entre escoceses e ingleses (denominado como encuentro
internacional) se celebró en 1902 y desde ese momento fue considerado como el partido más
importante de todos, ya que estos encuentros eran verdaderas <<pruebas de fuerza>> donde se disputaban la hegemonía deportiva y los honores patrióticos, por esa causa, los <<amantes del fútbol viajaban desde todos los rincones del país>> para presenciar la lucha entre dos naciones que, aunque formaban parte del mismo Reino, eran antagonistas entre sí por razones históricas, políticas, culturales, económicas y por supuesto, también deportivas.\footnote{``Scotia vs Albion'', \emph{Mexican Herald}, 8 de febrero 1902, p. 8. ``Lively foot'', \emph{Mexican Herald}, 10 de febrero 1902, p. 2. ``Other team named'', \emph{Mexican Herald}, 11 de enero 1906, p. 5. ``England vs Scotland'', \emph{Mexican Herald}, 14 de enero 1906, p. 5.}

El encuentro internacional, era un evento que reunía a la mayoría de la comunidad británica asentada en la ciudad de México y a los residentes en otras ciudades del país, pues en estos partidos participaban los mejores futbolistas radicados en territorio mexicano.\footnote{Desde su primera edición, el encuentro internacional, se estableció como el partido más importante de todos por la rivalidad nacional existente entre escoceses e ingleses. Tanto escoceses como ingleses, conformaban un comité para buscar y elegir a sus jugadores y se menciona que algunos de los jugadores debían viajar más de trescientas millas para tener el privilegio de jugar por su patria. Una vez conformados los equipos, el comité de un equipo le enviaba por escrito un reto formal al otro, especificando la hora, lugar y día del partido. Ambos equipos en sus uniformes portaban colores o símbolos nacionales: los ingleses usaban un uniforme en blanco con una rosa de Albion, mientras que los escoceses jugaban con un \emph{short} en color blanco y una camiseta en azul con la característica cruz en diagonal de la bandera en color blanco. Los espectadores por su parte, apostaban fuertemente por sus equipos y solían impulsarlos con música tradicional, los escoceses llevaban gaitas que interpretaban el himno nacional, mientras que los ingleses llevaban una banda militar que tocaba viejas canciones inglesas. ``International football'', \emph{Mexican Herald}, 31 de enero 1902, p. 7. ``Next Sunday’s football'', \emph{Mexican Herald}, 5 de febrero 1902, p. 5. ``Scotia vs Albion'', \emph{Mexican Herald}, 8 de febrero 1902, p. 8. ``Lively foot'', \emph{Mexican Herald}, 10 de febrero 1902, p. 2. ``Crónica general'', \emph{El Mundo}, 19 de enero 1906, p. 3.} Pero, a pesar de la evidente importancia del encuentro internacional, Carlos Calderón no hace ninguna referencia de estos encuentros, porque estos duelos ponen en evidencia que Pachuca no era el único centro futbolístico del país o el más importante, ni tampoco el que propuso fundar la primera liga de fútbol, de hecho, en la capital, los practicantes del fútbol no tenían conocimiento de la existencia del \emph{Pachuca Athletic Club}.\footnote{Carlos Calderón, no sólo considera a Pachuca como la cuna del fútbol, sino también le atribuye al \emph{Pachuca Athletic Club} la formación de la primera liga de fútbol, señalando que a fin de que el fútbol pudiera prosperar, era necesario encontrar otros equipos con quien enfrentarse, situación que en Pachuca no era posible, por tanto, los miembros del \emph{Club Pachuca} se dieron a la tarea de <<buscar por fuera grupos interesados en el fútbol que quisieran formar una liga>>. Fue entonces que William Blamey, el fundador del \emph{Pachuca Athletic Club}, recordó que en la ciudad de México había varios colegios donde se intentaba aleccionar a los estudiantes en la práctica del fútbol, así que se puso en contacto con ellos y con los directivos del \emph{Reforma Athletic Club}, quienes con mucho gusto acogieron la idea de Blamey de formar una liga. Calderón, \emph{Pachuca la cuna del fútbol}, p. 24, 25.}

En efecto, en 1902 en el Hotel Jardín de la ciudad de México, los interesados en
practicar el fútbol se reunieron con el fin de discutir los puntos relativos a la organización de
una serie de partidos para disputarse un campeonato bajo las reglas del fútbol asociación y
que tendría el membrete de \emph{Mexico Amateur Association Football League}. En varias
reuniones posteriores se acordó que los partidos se celebrarían en el \emph{Club Reforma}, que el
torneo consistiría de dos juegos entre cada equipo y que participarían el \emph{Reforma Athletic
Club}, el \emph{British Club}, el \emph{Mexico Cricket Club} y el \emph{Orizaba Athletic Club}, aunque después,
los directivos de la recién organizada liga se enteraron con mucha satisfacción, de que en
Pachuca había un equipo (el \emph{Pachuca Athletic Club}) que deseaba participar en el torneo.\footnote{``Orizaba to enter football league'', \emph{Mexican Herald}, 24 de julio 1902, p. 2. ``Passing day'', \emph{Mexican Herald}, 26 de julio 1902, p. 2. ``Football players meet'', \emph{Mexican Herald}, 31 de julio 1902, p. 2. ``passing day'', \emph{Mexican Herald}, 19 de agosto 1902, p. 5. ``Football contests'', \emph{Mexican Herald}, 20 de septiembre 1902, p. 5. ``Football Schedule'', \emph{Mexican Herald}, 23 de septiembre 1902, p. 2. ``Football prospects good'', \emph{Mexican Herald}, 4 de octubre 1902, p. 8. ``Sports of Sunday'', \emph{Mexican Herald}, 20 de octubre 1902, p. 2.}

Resulta evidente que el desarrollo del fútbol en México fue diferente a como se había
pensado, es decir, no fue obra de un solo individuo ni un solo equipo o club, sino que fue un
proceso de colaboración conjunta donde intervinieron gran cantidad de individuos y grupos
(establecidos en varias partes del país) y que confluyeron de forma interdependiente a partir de la socialización\footnote{La socialización es la tendencia <<para crear redes y organizaciones fuera de la familia>>. Los individuos
recurren a la socialización, para perseguir intereses similares y fortalecerse. Szymanski, ``A theory'', pp. 1-32.} generada en torno a la práctica del fútbol, donde de manera simbólica los valores culturales y las identidades nacionales y de grupo, fueron puestas en disputa, lo que finalmente fue lo que propició el despegue de la primera liga de fútbol en México.

Sin embargo, cabe aclarar que el fútbol no era el deporte más importante del
porfiriato, tal y como Carlos Calderón establece (ese fue el béisbol), esto porque los
estadounidenses tenían mayor presencia en México que los británicos.\footnote{Esparza, \emph{La nacionalización}, pp. 70-117.} En 1897 el capital
estadounidense invertido en México representaba el 64\% de los 320 millones de dólares que
los Estados Unidos habían invertido en América Latina. En contraparte, la inversión del
Reino Unido en nuestro país alcanzó el 16.94\% en 1895, ya que México fue para los
británicos la tercera o cuarta opción para invertir, detrás de Argentina, Brasil y Perú.\footnote{Parra, <<Los orígenes>>, pp. 139-158.}

Según Matthew Brown, Stefan Szymanski y Andrew Zimbalist, la difusión del fútbol
en el mundo se deriva de la expansión comercial del imperio británico la cual, se llevó a cabo
mediante dos modelos; el sistema colonial y el llamado imperialismo informal.\footnote{A. J. Mangan fue uno de los primeros historiadores anglosajones en historiar el fútbol en Sudamérica (analizó el fútbol en Argentina), sin embargo, sólo utilizó fuentes secundarias y sus resultados los extrapoló al resto del continente, incluyendo México, Guatemala y Perú. En estudios más recientes se establece que <<cualitativa y cuantitativamente, la presencia de los británicos fue diferente a lo largo del continente>>, pues cada región y país tiene sus particularidades. En síntesis, los historiadores anglosajones analizan el surgimiento del fútbol en el continente americano a partir de la dependencia económica o el poder imperial, mientras que los latinoamericanos explican que el surgimiento del soccer en América contribuyó a la generación e integración de identidades. Brown, ``British informal'', pp. 169-182. Szymanski, Zimbalist, \emph{National pastime}, p. 54.} En los
países que fueron colonia británica (como la India, Sudáfrica, Australia, Nueva Zelanda), el
fútbol y los deportes en general fueron difundidos por iglesias, escuelas, clubes, centros de
trabajo y la milicia, con el objetivo de establecer el modo de vida de los británicos. El fútbol
fue un vehículo del poder colonial usado para enseñar las reglas del sistema y para civilizar, controlar y establecer el \emph{statu quo} que permitiera transformar las condiciones locales y
volverlas económicamente favorables para los británicos.\footnote{Keech, ``England'', pp. 5-22. Hill, ``Football as code'', pp. 12-28, Dejonghe, \emph{The popularity of football}, pp. 1-12. Njororai, ``Colonial legacy'', pp. 866-882. Cho, ``Introduction'', pp. 579-587.}

Por otra parte, se conoce como imperialismo informal, a la relación que estableció el
Reino Unido con aquellos países que no fueron colonias británicas, como fue el caso de América Latina y se basaba <<en la dominación económica a través del libre comercio, los préstamos y la creación de la infraestructura al interior de los países>>. El imperialismo informal fue una forma de penetrar en países como Argentina, Brasil, Perú o México y a la vez <<evitar un conflicto directo con los Estados Unidos y su Doctrina Monroe>>\footnote{Después de la segunda guerra, al imperialismo informal se le denomina como <<neocolonialismo>>. Pérez, <<Restablecimiento>>, pp. 11-36.}

Mientras que en las colonias el fútbol fue utilizado para implantar el sistema colonial, en los países donde se estableció el imperialismo informal, el fútbol fue utilizado para reforzar las identidades y unificar a los miembros de la comunidad británica.\footnote{El imperialismo informal explica, en parte, la escasa presencia de los británicos en México, pero también deben considerarse otros aspectos que limitaron su migración a nuestro país como la intolerancia religiosa y, principalmente, el rompimiento de las relaciones diplomáticas en 1861. Villegas, <<Los intereses>>, pp. 337-353. Heath, <<Los primeros>>, pp. 77-89. Alanís, <<Los extranjeros>>, pp. 539, 566. Dejonghe, \emph{The popularity of football}, p. 7. Szymanski, Zimbalist, \emph{National pastime}, p. 54.} En nuestro país, los extranjeros recién llegados durante el porfiriato, se sentían extraños por las costumbres, la comida y el modo de vida de los mexicanos y también porque había poco para hacer <<los que vivían en las ciudades  tenían mejor oportunidad de vivir como lo hacían en su país que al parecer ese era el deseo de muchos>>. Para los extranjeros, la práctica deportiva fue una de las formas que les permitieron vencer el aburrimiento y también una forma de <<sentirse unidos y leales a sus países...>> mientras buscaban fortuna en México.\footnote{``Americans in Mexico'', \emph{Mexican Herald}, 4 de diciembre 1902, p. 12. ``Where to go and what to do in Mexico City'', \emph{Mexican Herald}, 4 de septiembre 1904, p. 9.}

Fue el \emph{cricket} y no el fútbol, el primer deporte que los británicos practicaron en
México pues desde 1827 se fundó el \emph{Mexico Union Cricket Club}.\footnote{El \emph{Mexico Union Cricket Club} tuvo una exitosa primera etapa que duró de 1827 a 1870. Posteriormente, se volvió a reorganizar en 1880, logrando mantenerse vigente y prosperando hasta el cambio de siglo. Costeloe, ``To bowl a mexican'', pp. 112-124.} La temprana existencia
de un club dedicado a la práctica del \emph{cricket}, por sí mismo, es indicativo de la importancia
que este deporte tenía para los británicos en México. De hecho, el \emph{cricket} era considerado el
deporte nacional y se practicaba durante la primavera y el verano, las mejores temporadas
para la práctica de los deportes.\footnote{Hasta 1875, los clubes de \emph{cricket} en el Reino Unido fueron las principales organizaciones deportivas. Adrian Harvey señala que el fútbol fue introducido en los clubes de \emph{cricket} para mantener en forma a los jugadores en la temporada invernal. Harvey, \emph{Football}, pp. 23-208. Walton, ``The origins'', pp. 125-140.}

Desde fines del siglo \textsc{xviii}, el \emph{cricket} comenzó a ser considerado como una actividad
que encerraba los principales valores de los británicos (valor, masculinidad, rectitud), pero
aún más significativo que podía transmitirlos, por lo que su papel fue igual de importante en
aquellos lugares donde hubo una comunidad británica tal y como lo era en el Reino Unido y
en sus colonias. Por medio del \emph{cricket} ---nos dice Dominic Malcolm--- se buscaba inculcar
nociones de caballerosidad y de masculinidad. Se pensaba que el \emph{cricket} podía ser usado para civilizar, mostrar lealtad y para estrechar los lazos entre los <<colonos>> y la madre patria.\footnote{Por ejemplo, en Australia el \emph{cricket} simbolizó la unión de la llamada <<raza anglosajona>>. Szymanski, Zimbalist, \emph{National pastime}, p. 54. Malcolm, ``Cricket'', p. 77, 78. Mangan, Hickey, ``Pioneering'', pp. 690-726.}

En México al igual que en el Reino Unido, el interés deportivo de los británicos
estuvo centrado en su deporte nacional (el \emph{cricket}) que antes de 1907 fue más importante que
el fútbol. El fútbol \emph{soccer} comenzaría a practicarse como un complemento, es decir, se estableció como un deporte auxiliar que mantenía activos a los jugadores de \emph{cricket} en los meses de invierno, luego del término de la temporada de \emph{cricket}.\footnote{La prensa menciona que el \emph{cricket} fue el deporte más antiguo que se importó a México y que en Pachuca se practicaba desde 1850. En varios clubes de la capital como el \emph{Reforma Athletic Club}, el \emph{cricket} fue el deporte más importante por más de una década, sin embargo, para el año de 1907, los británicos radicados en la ciudad de México y ciudades aledañas, comenzaron a notar que el \emph{cricket} (su deporte nacional) cada vez tenía menos influencia y era más difícil reunir a los suficientes jugadores, pues eran menos los interesados en practicarlo, por lo que el futuro de este deporte se volvió incierto. ``Resume of local sporting news'', \emph{Mexican Herald}, 18 de octubre 1903, p. 9. ``The development of sport in Mexico growth has been steady if not rapid'', \emph{Mexican Herald}, 09 de octubre 1904, p. 11. ``The cricket season'', \emph{Mexican Herald}, 23 de enero 1907, p. 9. ``Cricket sugestions'', \emph{Mexican Herald}, 6 de febrero 1907, p. 9. ``Are dual champions'', \emph{Mexican Herald}, 1 de marzo 1907, p. 9.
``Reforma-Pachuca cricket match is practically off'', \emph{Mexican Herald}, 26 de mayo 1908, p. 7.}

En efecto, el fútbol, al principio, fue una práctica secundaria con respecto al \emph{cricket}, por ese motivo, se procuraba celebrar los partidos  en <<fecha conveniente>> para no interferir con la temporada del deporte nacional británico (el \emph{cricket}). Por esa razón, la primera temporada de la \emph{Mexico Amateur Association Football League}, iniciaría en octubre y finalizaría antes de enero, pues en enero daba comienzo la temporada de \emph{cricket}.\footnote{``Football players meet'', \emph{Mexican Herald}, 31 de julio 1902, p. 2. ``Football teams organized'', \emph{Mexican Herald}, 9 de octubre 1902, p. 5. ``Football'', \emph{Mexican Herald}, 17 de enero 1902, p. 8.} Pero a
medida que el fútbol se popularizaba, poco a poco, fue desplazando al \emph{cricket}, pues los preparativos para organizar las temporadas de fútbol ya no se realizaban a fines de septiembre, sino en el verano cuando todavía estaba vigente la temporada de \emph{cricket}.\footnote{``Sporting mention'', \emph{Mexican Herald}, 8 de mayo 1903, p. 5. ``Cricket sugestions'', \emph{Mexican Herald}, 6 de febrero 1907, p. 9. ``Opinion divided as to football season'', \emph{Mexican Herald}, 22 de abril 1910, p. 4. ``Cricket for May 15 at Reforma grounds'', \emph{Mexican Herald}, 4 de mayo 1910, p. 4. ``Football league is proposed for winter'', \emph{Mexican Herald}, 16 de abril 1910, p. 4.}

Aunque la evidencia hemerográfica nos muestra que el \emph{cricket} prosperó en México
antes que el fútbol, Carlos Calderón olvida mencionar que el \emph{cricket} era más importante que
el fútbol, con lo cual, se pierden referentes empíricos necesarios para comprender el contexto
histórico que vio nacer la práctica del fútbol en México y, por otra parte, se crea una
distorsión, porque se hace aparentar que el fútbol era el deporte más importante para los británicos, que era la única actividad deportiva que practicaban y que su único objetivo para
venir a México (especialmente a Pachuca) fue el de implantar la práctica de este deporte.\footnote{En \emph{Pachuca la cuna del fútbol}, Calderón no hace mención del \emph{cricket}, sin embargo, en dos artículos posteriores sí lo menciona, aunque no señala la importancia que tenía sobre el fútbol. Calderón, \emph{Pachuca la cuna del fútbol}, pp. 12-25. Calderón, <<Orígenes (\textsc{ii})>>, pp. 1-5. Calderón, <<Orígenes (\textsc{iii})>>, pp. 1-8.}

Carlos Calderón al tratar de explicar el origen del fútbol mexicano hace referencia a
varios hechos que por estar malinterpretados provocan una distorsión (¿involuntaria?) que
hace parecer que el ambiente reinante en el porfiriato era idóneo y propicio para establecer
la práctica del fútbol. Por ejemplo, Calderón señala que los deportes fueron para los
mexicanos una nueva forma de ejercitarse, de divertirse, de aliviar tensiones, pero principalmente de <<matar el tiempo libre que la paz porfiriana les ponía enfrente>>\footnote{Calderón, \emph{Pachuca la cuna del fútbol}, p. 12.}

En primer lugar, es pertinente señalar que, en el caso del fútbol, la sociedad mexicana
no tuvo participación en este deporte sino hasta 1910, es decir, en los primeros equipos y
torneos de fútbol no figuraron futbolistas mexicanos, por lo que la práctica del fútbol fue en
un principio, llevada a cabo en su totalidad por los extranjeros.\footnote{Los primeros mexicanos en jugar al fútbol fueron el Marqués de Guadalupe, Carlos Rincón Gallardo, quien jugó en el \emph{Club Reforma} y los hermanos Parada (Jorge y Agustín), del \emph{British Club}, éstos últimos, aprendieron a jugar mientras estudiaban en Inglaterra. Según la prensa, a pesar de que la liga de fútbol tenía ocho años de celebrarse, salvo las excepciones ya mencionadas, sólo habían jugado futbolistas del imperio británico. Fue hasta 1910 (en el caso del valle de México, pues en Guadalajara inició en 1906 con la fundación del \emph{Club Guadalajara}) cuando por iniciativa de Alfredo B. Cuéllar se funda el \emph{Club México}, un equipo formado enteramente por jugadores mexicanos. ``Football'', \emph{Mexican Herald}, 22 de diciembre 1901, p. 12. ``More diversified sports here than any city this size'', \emph{Mexican Herald}, 6 de octubre 1907, p. 20. J. Sanderson, ``Auspicious opening of the football season'', \emph{Mexican Herald}, 19 de septiembre 1909, p. 25. ``Una invitación a footbolistas'', \emph{El Diario}, 16 de abril 1910, p. 5. ``Football league is now formed'', \emph{Mexican Herald}, 31 de agosto 1910, p. 4. Angelotti, \emph{Chivas y tuzos}, pp. 285-294.} En segundo término,
Calderón confunde e interpreta como sinónimos la pacificación del país (la \emph{pax} porfiriana)
con la disponibilidad de tiempo libre. La llamada paz porfiriana es un término que hace
referencia al cúmulo de acciones emprendidas por el gobierno de Porfirio Díaz y que permitieron poner fin a la inestabilidad política y social que mantenía al país eminentemente rural, disgregada y con gran parte de la economía destruida.\footnote{La paz porfiriana terminó con los constantes golpes de Estado, las intervenciones extranjeras, el bandidaje y los conflictos religiosos. De igual forma, la pacificación del país permitió obtener el reconocimiento de Estados Unidos y el Reino Unido, lo que le daría acceso a préstamos y abrió el paso a las inversiones. Beezley, \emph{Identidad nacional}, p. 104. Guerra, \emph{México}, p. 325. Beezley, ``Estilo porfiriano'', pp. 265-284.}

En lo referente a la disponibilidad de tiempo libre, cabe señalar que ésta era muy
limitada, ya que en el porfiriato la mayoría de los individuos trabajaba de catorce a dieciséis
horas al día.\footnote{En el Reino Unido, desde 1880 se estableció la semana corta de trabajo que obligaba a las fábricas a cerrar los sábados a las dos de la tarde. En México, en contraparte, la jornada laboral de ocho horas se estableció hasta 1917. Herrera, <<Sociedad>>, p. 126, 127. Lastra, <<Sindicalismo>>, p. 38. Lóyzaga, <<En torno>>, pp. 317-326. Mangan, ``Missing men'', pp. 170-188. Keech, ``England'', p. 7.} El día que más se prestaba para el ocio era el domingo; por la mañana se
asistía a los paseos (Zócalo y Canal de la Viga) y a la Iglesia. Por la tarde, la mayoría de las
personas acudía a los toros.\footnote{``Sunday life in Mexico'', \emph{Two Republics}, 15 de agosto 1898, p. 4. ``Liked Mexico'', \emph{Two Republics}, 4 de abril 1899, p. 5. Roca, Aguayo, <<Usos>>, 2004, pp. 103-128.} En un principio, se intentó establecer los domingos en la
mañana como el horario para la práctica deportiva, sin embargo, este horario no se consolidó
porque la Iglesia Metodista impedía romper el \emph{Sabbath} para practicar deportes.\footnote{``Baseball sermon'', \emph{Two Republics}, 5 de agosto 1887, p. 4. Baldwin, <<Diplomacia>>, 1986, p. 306.}

En los textos de Calderón, se observa una recurrente omisión o distorsión de hechos
que, en cierto modo, hacen posible y facilitan la fundación del Club Pachuca y el surgimiento
del fútbol. Por ejemplo, Calderón considera que los británicos de todas las clases sociales
contaban con demasiado tiempo libre, por lo que <<cansados de la calma provinciana... decidieron fundar el primer equipo en México del deporte mágico llamado fútbol.>>\footnote{Calderón, \emph{Pachuca la cuna del fútbol}, p. 17.}

La iniciativa de formar un equipo de fútbol ---según Calderón--- surgió del minero
William Blamey, quien, al visitar la ciudad de México notó con sorpresa que los alumnos de
algunos colegios ingleses pateaban el balón sin mucho sentido ante su desesperado profesor, <<que a gritos trataba de explicarles hacia donde debían patear el esférico y la manera correcta de hacerlo>>. Posteriormente, cuando Blamey regresó a Pachuca, con mucho entusiasmo <<informó a sus compañeros que el fútbol había llegado a México, y se propuso formar un equipo entre los <<hijos de la oscuridad>> que pasaban casi todo el tiempo bajo tierra. La idea gustó sobremanera, por lo que mineros y técnicos se apuntaron en la lista de Blamey.\footnote{Calderón señala que en noviembre de 1900 se reunieron los mineros de Santa Gertrudis, La Blanca y Real del Monte para constituir formalmente al \emph{Pachuca Athletic Club}. Calderón, \emph{Pachuca la cuna del fútbol}, p. 20,
21. Carlos Calderón, <<Pachuca: La cuna del fútbol>>, \emph{Mediotiempo}, 2 de junio 2007. \url{http://www.mediotiempo.com/futbol/editoriales/carlos-calderon/2007/06/pachuca-la-cuna-del-futbol}}

Este relato, aunque no está fehacientemente comprobado, ni se sabe de dónde lo
recupera Calderón, ha sido reproducido en varios textos y en su momento se consideró como
la prueba más clara de que Pachuca era la cuna del fútbol mexicano, porque en consideración
de Calderón, fue el lugar dónde se fundó el primer club de fútbol formalmente organizado.
Sin embargo, en posteriores trabajos Calderón parece desechar este relato, pues ya no
menciona que Blamey fue el que tuvo la idea de formar el \emph{Pachuca Athletic Club}, sino que fueron los jugadores de \emph{cricket}, <<junto con los ingenieros y mineros que venían de la fábrica siderúrgica de Thames Ironworks...>> los fundadores del \emph{Pachuca Athletic Club}.\footnote{Calderón, <<Orígenes (\textsc{ii})>>, pp. 1-5. Calderón, <<Orígenes (\textsc{iii})>>, pp. 1-8.}

En los textos no académicos resulta muy común las contradicciones y también es
común que se retracten de sus dichos previos y tengan que replantear sus postulados. Esto se
debe a que no se realiza una eficiente investigación, pues, por una parte, se procede de forma
tendenciosa y parcial, pues sólo se busca imponer a una región (a Pachuca) como la cuna del
fútbol mexicano, por tanto, únicamente se compilan los hechos que son funcionales al
objetivo que se pretende lograr, mientras que otros datos y otras fuentes que podrían revelar nueva información y replantear los paradigmas y el rumbo historiográfico de la historia del
fútbol mexicano, han quedado de antemano descartadas.\footnote{En palabras de Gabriel Angelotti, esta forma de proceder resulta ambigua, porque puede interpretarse como <<producto del olvido, del descuido o como resultado de una conducta intencionada por parte de un grupo hegemónico.>> Angelotti, <<El origen del fútbol>>, p. 8,9.}

Las razones por las que Calderón se desdice de sus aseveraciones anteriores, es
porque toda su atención se ha centrado en confirmar que Pachuca es la cuna del fútbol
mexicano, también porque los hechos que presenta no han sido confirmados y porque
desconoce la historia de México en general y la historia del fútbol en particular (por esa
causa, muchos datos los considera obvios o sabidos). Por otra parte, a medida que se
profundiza en la revisión de fuentes, nuevos datos se van encontrando y que revelan que el
surgimiento del fútbol en México es muy diferente a como previamente se había establecido.

Estos aspectos han propiciado que Calderón cambie su postura acerca de cómo fue
que surgió el fútbol en México, pues en un artículo publicado en 2014, vuelve a modificar sus planteamientos aseverando poner fin <<a una disyuntiva generada desde hace muchos años sobre cúal es el primer equipo de fútbol en México>>. Calderón señala que por fortuna y gracias a la digitalización de archivos, nueva información ha aparecido y de una vez por todas se confirma que Pachuca es la cuna del fútbol mexicano.\footnote{Calderón, <<El Pachuca Athletic Club no nació en 1900>>, pp. 1-6.}

La nueva información a la que hace referencia Calderón, son en específico, varias notas periodísticas publicadas en el \emph{Two Republics}. En una de ellas, con fecha del cuatro de octubre de 1895 menciona la <<reciente creación del \emph{Pachuca Athletic Club} y en donde se indica que se jugarán todos los deportes, menos el \emph{cricket}...>>.\footnote{Calderón, <<El Pachuca Athletic Club no nació en 1900>>, pp. 1-6.} En una nota posterior, fechada el seis de octubre de 1895, se hace referencia a la reunión celebrada entre el \emph{Pachuca Cricket Club}, el \emph{Velasco Cricket Club} y el \emph{Pachuca Football Club}, con el objetivo de fusionarse en una sola institución (el \emph{Pachuca Athletic Club}) donde el \emph{cricket}, el tenis, el
fútbol y otros deportes pudieran ser practicados y fomentados.\footnote{``Pachuca Athletic Club'', \emph{Mexican Herald}, 6 de octubre 1895, p. 8. Calderón, <<El Pachuca Athletic Club no nació en 1900>>, pp. 1-6.}

Según Calderón, estas notas, además de evidenciar que el \emph{Pachuca Athletic Club} no surge en 1900 sino en 1895, representan una nueva pista que indica que previamente ya existía un equipo de fútbol (el \emph{Pachuca Football Club}), por lo que se dio a la tarea de buscar otras notas donde se hiciera referencia a este equipo y sus actividades. Calderón no pudo localizar datos referentes a la fundación del \emph{Pachuca Football Club}, pero en su lugar encontró
en una nota donde se menciona que la comunidad británica celebró un \emph{picnic} y como parte del programa de actividades, se disputó un partido de fútbol, donde los jóvenes atletas demostraron que ninguno de ellos había perdido sus habilidades y vigor en este deporte, por lo que esperaban enfrentarse contra el \emph{<<Mexican Team>>}. (¿Acaso se refiere al equipo del \emph{Mexican Athletic Club} que se fundó el {2 de octubre de 1892?}).\footnote{``Pachuca points'', \emph{Two Republics}, 8 de noviembre 1892, p. 4. Calderón, <<El Pachuca Athletic Club no nació en 1900>>, pp. 1-6.}

Aunque la nota no menciona que este encuentro fuera disputado por el \emph{Pachuca
Football Club}, Calderón así lo interpreta y además considera que estos datos representan <<un antecedente directo>> de la posterior fusión de equipos de donde surge el \emph{Pachuca Athletic Club}, por lo que concluye que <<Bajo la observación de estos documentos históricos, podemos aseverar que el \emph{Pachuca Athletic Club} es el primer club de fútbol constituido en México, salvo que nuevos datos en un futuro muestren lo contrario.\footnote{En consideración de Carlos Calderón, sólo faltaba <<definir exactamente donde botó el primer balón en
nuestro país, si fue en la ciudad de México, Real del Monte o el mismo Pachuca, ya que por un lado los colegios ingleses de la capital, los \emph{Cornish} situados en Real del Monte y en Pachuca se pelean esa distinción.>> Calderón, <<El Pachuca Athletic Club no nació en 1900>>, pp. 1-6.}

Como ya se ha mencionado, en los textos de Carlos Calderón se observa una
deficiente investigación, pues, por una parte, en su afán de confirmar que Pachuca es la cuna
del fútbol, descarta sin reparo otras posibilidades. Es decir, Calderón no contempla ni le
interesa revisar información referente a otras regiones o ciudades, pues parte de una sola
hipótesis de trabajo: ¿Es Pachuca la cuna del fútbol mexicano? Y como el experimento se
realiza consultando información que sólo hace referencia a la historia del fútbol en Pachuca,
la hipótesis siempre termina confirmándose.

Si los trabajos de Carlos Calderón tuvieran como único fin historiar el fútbol en
Pachuca no habría ninguna objeción y hasta sería irrelevante revisar sus planteamientos, pero
el problema es que sus resultados buscan tener alcances nacionales, ya que pretende imponer
sus conclusiones como verdades únicas y universales, incluso para áreas y regiones donde el
fútbol todavía no se ha historiado, esta forma de proceder propicia las contradicciones, pues
a medida que nueva información aparece (sobre todo de áreas y regiones diferentes a
Pachuca), se refutan y se modifican los planteamientos tenidos como paradigmas y a su vez,
se reavivan los debates y las controversias.

Si bien ---dice José Gaos--- no es posible que un historiador pueda recopilar todos los
documentos existentes de su tema de estudio, debe en contraparte, ser capaz de reconstruir la historia y <<aportar novedades>> a partir de la información que dispone y que considera suficiente para explicar cómo surge y se desarrolla un fenómeno.\footnote{Gaos, <<Notas sobre la historiografía>>, pp. 481-508.} En el caso de Carlos Calderón, su proceder en lugar de dilucidar con claridad la manera en que el fútbol se desarrolló, se observa que su trabajo con las fuentes empíricas no es eficiente, pues de forma
reiterada cae en contradicciones.

Si el \emph{Pachuca Athletic Club} se fundó en 1895 por iniciativa de los jugadores de
\emph{cricket}, entonces ¿cómo encaja el relato del minero William Blamey que Calderón, en un
principio, presentó como la prueba más clara de que Pachuca era la cuna del fútbol mexicano?
¿De dónde lo recuperó? ¿Acaso lo inventó? ¿Sus informantes le proporcionaron datos falsos
y lo engañaron? ¿O será que malinterpreta la información?

Cabe la posibilidad de que Calderón malinterprete los datos empíricos, porque toma
literal lo que dicen las fuentes y porque no confirma y no contextualiza los datos con los que
trabaja.\footnote{Según José Gaos, para que un planteamiento sea considerado como verdadero debe ser <<verificable por todo sujeto posible>>. Por otra parte, una forma de comprobar la veracidad de los datos es por medio de la hermenéutica, que a decir de Douglas Booth es un dispositivo que permite entender <<cómo las personas interpretan el mundo>> y cómo <<representan sus experiencias.>> Booth ``Theory'' p. 12. Gaos, <<Notas sobre la historiografía>>, p. 504.} Es decir, Calderón no realiza ningún tipo de crítica para comprobar la veracidad de sus evidencias, sólo se remite a <<repetir lo dicho por sus fuentes>>, no determina si <<son auténticas o fraguadas>>, tampoco cuestiona quién las creó, con qué propósitos o cómo fueron creadas y difundidas.\footnote{González, \emph{El oficio de historiar}, p. 227. Osmond, Phillips, ``Sources'', p. 36.}

Hacer Historia no consiste en buscar datos y fechas en papeles viejos, ya que la verdad
histórica no se encuentra intrínsecamente en los documentos, tal y como asume la corriente
positivista, sino que debe considerarse que algunas fuentes primarias (como los periódicos)
no siempre son exactas, por lo que antes de utilizarse, deben ser verificadas para no caer en contradicciones y en anacronismos y también, para evitar las falsas atribuciones y
significados a los hechos y a los conceptos.\footnote{Luis González señala que la construcción de una explicación histórica requiere de <<abundantes testimonios, pero no crudos. El que caza una liebre y la sirve con todo y pelos es tan mal cazador como el que entrega al lector exquisitas piezas documentales sin someterlas previamente al lavado y la cocción de las operaciones críticas>>. González, \emph{El oficio de historiar}, p. 51. Osmond, Phillips, ``Sources'', p. 39.}

Por otra parte, aunque un documento sea auténtico, si la interpretación que se realiza
es inexacta, no será posible construir un conocimiento histórico veraz, en su lugar se
producirá y difundirá información incorrecta que generará una percepción distorsionada del
fenómeno estudiado. En los textos de Carlos Calderón son muy recurrentes las distorsiones
históricas (¿intencionales?) derivadas de una equivocada interpretación y no es el único caso,
por el contrario, es un rasgo compartido por muchos de los no académicos que se han
interesado en indagar en el pasado del fútbol mexicano.

\section*{\mdseries\large\textsc{El surgimiento del fútbol mexicano: nuevos datos y nuevas interpretaciones}}
\addcontentsline{toc}{section}{El surgimiento del fútbol mexicano: nuevos datos y nuevas interpretaciones}

\noindent Las malinterpretaciones (voluntarias o involuntarias) de los documentos son muy recurrentes
en las afirmaciones de los no académicos, ya que, en su consideración, al presentar evidencia
empírica se presenta la verdad histórica. Sin embargo, se olvidan de que las palabras y los
conceptos no tienen un significado fijo, por tanto, se hace necesario establecer la
correspondencia del dato encontrado con su contexto histórico (contrastándolo con otros
documentos y testimonios que le son contemporáneos), para con ello constatar si la
interpretación que se realiza de las fuentes le está atribuyendo un significado correcto a las
evidencias encontradas en los documentos históricos.\footnote{Booth, ``Theory'', p. 21. Osmond, Phillips, ``Sources'', p. 35, 36. González, \emph{El oficio de historiar}, p. 231. Moreno, <<La investigación empírica>>, p. 72, 73. Bloch, \emph{Apología para la historia}, p. 159, 160.}

La Historia, según Murray Phillips y Gary Osmond, no se descubre en las fuentes,
sino que se construye a partir de las metodologías y enfoques empleados para analizar un
fenómeno y con las interpretaciones que se realizan de los documentos.\footnote{Osmond, Phillips, ``Sources'', p. 39, 40, 41.} Pero, cuando una
interpretación es incorrecta, se tergiversa el sentido de los hechos y la Historia se distorsiona,
siendo los anacronismos, las principales malinterpretaciones que se observan en los textos.
En efecto, en el caso de la historia del fútbol mexicano, erróneamente, se juzga con los
criterios del presente, sin considerar que algunos conceptos tenían un significado diferente
en un determinado momento y lugar. Los anacronismos son comunes en los textos de Carlos
Calderón, pero también se notan en las nuevas evidencias presentadas recientemente y con
las que se pretende probar que Pachuca es la verdadera cuna del fútbol mexicano.

El 6 de noviembre del 2014, la prensa mexicana publicó que la \textsc{lxii} legislatura del Congreso de Hidalgo, declaró a la ciudad de Pachuca como <<Cuna del futbol mexicano...>> El decreto se basó <<en documentos y publicaciones del siglo \textsc{xix}>>, principalmente, en el semanario \emph{El Minero de Pachuca}, donde se menciona que el 12 de mayo de 1889, en la Plaza Hipódromo de Pachuca, se celebró <<un partido de fútbol entre la Mina \emph{El Rosario} y la Mina \emph{La Joya}>>.\footnote{Janet Barragán, <<Pachuca es declarada Cuna del Fútbol por el Congreso de Hidalgo>>, \emph{Milenio}, 6 de noviembre 2014, \url{http://laaficion.milenio.com/futbol/Pachuca-Cuna-Futbol-Congreso-Hidalgo_0_404359932.html}} En dicha nota, que a la fecha es <<la reseña periodística más antigua de un partido en nuestro país>>, se establece que el encuentro de fútbol terminó <<con un enfrentamiento entre ambos equipos>>. Aunque no se tienen más datos sobre este encuentro, la prensa de la época dio a conocer públicamente que los participantes en el partido de fútbol protagonizaron una pelea y hoy en día estos hechos han sido presentados como la evidencia documental que demuestra que <<la Bella Airosa (Pachuca) es el primer lugar donde se jugó un partido de fútbol.>>\footnote{Juan Ricardo Montoya, <<Legisladores hidalguenses declaran a Pachuca como Cuna del Fútbol mexicano>>, \emph{La Jornada}, 6 de noviembre 2014, \url{http://www.jornada.unam.mx/ultimas/2014/11/06/legisladoreshidalguenses-declaran-a-pachuca-como-cuna- del-futbol-mexicano-9978.html} Reportero de Guardia, <<Oficialmente: Pachuca, Cuna del Fútbol Mexicano>>, \emph{El Sol de Hidalgo}, 7 de noviembre 2014, \url{http://www.oem.com.mx/elsoldehidalgo/notas/s3260.htm}}

El Congreso hidalguense, al conocer estos <<testimonios impresos>> aprobó por unanimidad la propuesta presentada por el diputado por el distrito 17 de Jacala, Javier Amador y con este acto se disiparon todas las dudas sobre <<donde se jugó por primera vez el fútbol en el país...>> ya que <<a la fecha no existe algún documento que sea anterior a las publicaciones citadas que reseñen que ya se practicaba de manera oficial y organizada el fútbol en algún otro lugar del país.>>\footnote{Janet Barragán, <<Pachuca es declarada Cuna del Fútbol por el Congreso de Hidalgo>>, \emph{Milenio}, 6 de noviembre 2014, \url{http://laaficion.milenio.com/futbol/Pachuca-Cuna-Futbol-Congreso-Hidalgo_0_404359932.html} Reportero de Guardia, <<Oficialmente: Pachuca, Cuna del Fútbol Mexicano<, \emph{El Sol de Hidalgo}, 7 de noviembre 2014, \url{http://www.oem.com.mx/elsoldehidalgo/notas/s3260.htm}}

Finalmente, luego de aprobaba la propuesta, el Congreso hidalguense procedió al reconocimiento de todos aquellos quienes <<durante años, han contribuido con trabajos de investigación...>> y que hoy en día hicieron posible que Pachuca fuera reconocida como la cuna del fútbol mexicano. En la nota se menciona a <<Daniel Zárate Ramírez, Aída Suárez Chávez, Juan Manuel Menes Llaguno, Luis Corrales Vivar, Carlos Calderón y Francisco Coca.>> personajes que han realizado <<aportaciones que están documentadas, pero sobre todo sustentadas en una investigación seria y sobre todo trascendente.>>\footnote{Reportero de Guardia, <<Oficialmente: Pachuca, Cuna del Fútbol Mexicano>>, \emph{El Sol de Hidalgo}, 7 de noviembre 2014, \url{http://www.oem.com.mx/elsoldehidalgo/notas/s3260.htm}}

Para el Congreso del Estado de Hidalgo y para los partidarios de Pachuca, la nota periodística publicada en 1889 en \emph{El Minero de Pachuca}, es la evidencia documental que, sin lugar a dudas, demuestra que la ciudad de Pachuca es la cuna del fútbol mexicano. Sin embargo, un análisis más profundo revela que esa nota periodística ha sido malinterpretada, debido a la incomprensión de algunos de sus conceptos, que en última instancia han producido una percepción distorsionada de la historia del fútbol mexicano.

Es decir, se interpreta que la nota publicada en 1889 en \emph{El Minero de Pachuca}, hace referencia a uno de los primeros partidos de fútbol \emph{soccer} disputados en México, pero, ¿En realidad fue así? ¿Es posible que la interpretación que se le ha dado a dicha nota esté equivocada? ¿Será acaso que se está confundiendo el significado del concepto fútbol? Se debe tener en cuenta que una cosa es lo que dicen los documentos y otra lo que se entiende y lo que se interpreta que dicen, pues comúnmente, cuando no se comprende el contexto histórico donde tienen lugar los hechos que se analizan, la percepción de un fenómeno se distorsiona y se hace una interpretación incorrecta de los documentos.

Francisco Moreno señala que <<la comprensión inicial>> de un dato recién descubierto debe considerarse como un <<juicio probable>> en lugar de una evidencia contundente e incuestionable, ya que el material empírico no puede ofrecer respuestas inmediatas porque no está ordenado, por tanto, el historiador debe comprender la relevancia de dicha información en su momento y lugar para verificar su veracidad y posteriormente establecer el significado correcto que el fenómeno en cuestión tiene para el presente.\footnote{Marc Bloch señala que, si todos los hechos desprovistos de explicación fueran verdaderos, <<la historia se reduciría a una sucesión de apuntes burdos sin gran valor intelectual.>> Moreno, <<La investigación empírica>>, pp. 81-87. Bloch, \emph{Apología para la Historia}, p. 115.}

Más que una prueba contundente que demuestra dónde surgió el fútbol, la nota de 1889 debe ser considerada como un juicio probable que se ha malinterpretado, pues sus atribuciones no fueron confirmadas y aun así se da por hecho que hace referencia al \emph{soccer}, descartando de antemano que se pueda tratar de otro estilo de fútbol.\footnote{Lewis, señala que una fecha o un concepto no prueba ni explica el origen o el desarrollo de un proceso histórico, por el contrario, se vuelven irrelevantes cuando no se contextualizan y no se explica su pertinencia para aclarar un proceso. Lewis, ``Innovation not invention'', pp. 475-488.} El problema es que anacrónicamente se interpreta la palabra fútbol como sinónimo de \emph{soccer}, tal y como se hace en el presente, sin considerar que, en 1889, el concepto fútbol tenía un significado distinto.

No se duda de la autenticidad de la nota publicada en \emph{El Minero de Pachuca}, es decir, queda claro que en 1889 se celebró un partido de fútbol. Sin embargo, esto no significa que se tratara de un encuentro de fútbol \emph{soccer} tal y como algunos han asumido, por lo que se debe cuestionar ¿A qué estilo de fútbol hace referencia la nota? ¿Se trataba de \emph{soccer} o algún otro estilo de fútbol? ¿Cuál era el significado que tenía el concepto fútbol en 1889? ¿En 1889 los británicos residentes en México lo utilizaban como sinónimo de \emph{soccer}? ¿Desde cuándo fútbol y \emph{soccer} se volvieron sinónimos? ¿En 1889 el concepto fútbol a qué se refería?

Según Gavin Kitching, el concepto fútbol es un término problemático porque dependiendo del momento y lugar, se ha utilizado como un genérico y como un sinónimo para referirse a diversos estilos e implementos.\footnote{Kitching, ``Old football'', pp. 1740-1743.} En la actualidad, en países como Argentina, Brasil, México o España, donde el fútbol asociación (o \emph{soccer}) es el estilo preponderante, el concepto fútbol y el concepto \emph{soccer}, indistintamente se usan como sinónimos.\footnote{Tony Collins señala que el término \emph{soccer} se vuelve sinónimo de fútbol después de la primera guerra mundial (1918). Dunning, Sheard, \emph{Barbarians}, p. 19, 20. Dunning, Curry, \emph{Association football}, pp. 12-21. Collins, \emph{Rugby's great split}, p. \textsc{xvii}.} En cambio, en países como Estados Unidos, Canadá, Australia o Nueva Zelanda, donde el fútbol americano o el \emph{rugby} son preponderantes, el término \emph{soccer} no se usa como sinónimo de fútbol, sino que es utilizado para referirse en exclusiva al fútbol asociación y así evitar confusiones con otros estilos.

Por otra parte, en el siglo \textsc{xix}, el término fútbol era utilizado indistintamente para referirse a todos los estilos de fútbol existentes como el \emph{rugby}, el \emph{soccer}, el fútbol americano, el fútbol canadiense, el fútbol australiano y el fútbol gaélico.\footnote{Dunning, Sheard, \emph{Barbarians}, p. 19, 20. Dunning, Curry, \emph{Association football}, pp. 12-21.} Pero también, el concepto fútbol era utilizado para referirse al balón. En efecto, en el siglo \textsc{xix}, el balón (sea cual fuere su forma y sea cual fuere el estilo de fútbol que se practicaba) era conocido como el <<fútbol>>. Por ejemplo, en Manchester, Eric Dunning encontró en los archivos, una ley que prohibía jugar ``with the foot-ball'' (con el fútbol).\footnote{Dunning, Sheard, \emph{Barbarians}, p. 19, 20. Dunning, Curry, ``Public schools, status rivalry'', pp. 31-36.}

En México, durante el último cuarto del siglo \textsc{xix} y las dos primeras décadas del siglo \textsc{xx}, el concepto fútbol también se utilizaba para referirse al balón.\footnote{Carlos Calderón erróneamente interpreta que el fútbol \emph{soccer} se practicaba en Pachuca desde 1895, porque encontró una nota donde se menciona la existencia del \emph{Pachuca Football Club}, sin embargo, el que un club tuviera en su membrete la palabra <<fútbol>>, no significa que practicara el fútbol \emph{soccer}, por ejemplo, el \emph{Dunedin Football Club}, el \emph{Wellington Football Club} y el \emph{Christchurch Football Club}, todos de Nueva Zelanda, eran clubes de \emph{rugby}. Guoth, ``loss of identity'', pp. 189-195. Calderón, <<El Pachuca Athletic Club>>, pp. 1-6.} Por ejemplo, en 1899 el \emph{Mexican Herald} publicó que unos jóvenes interesados en organizar un equipo, le escribieron al \emph{coach} White de la Universidad de Missouri para que les enviara ``a new football'' (un nuevo fútbol), el cual esperaban recibir muy pronto y una vez que lo recibieran, comenzarían con los juegos de práctica.\footnote{``Passing day'', \emph{Mexican Herald}, 19 de noviembre 1899, p. 16.}

También, en 1903 el \emph{Mexican Herald} explicaba por qué había una notable diferencia en los marcadores de los partidos de fútbol jugados en México y los reportados en Estados Unidos, señalando que la diferencia se debía a que en ambos países se practicaban diferentes estilos de fútbol.\footnote{En los Estados Unidos, de forma similar, las primeras notas que hacen referencia a la práctica del fútbol también han generado confusiones entre los historiadores, porque no se especifica de qué estilo se estaba hablando, si \emph{rugby}, \emph{soccer} o fútbol americano. Trouille, ``Association football'', pp. 455-476. P. 459.} En México, se cultivaba el \emph{soccer}, donde un gol anotado contaba un punto. Mientras que, en los Estados Unidos, se practicaba un estilo de fútbol (denominado como americano) derivado del \emph{rugby} y con reglas diferentes, ya que un gol pateado desde el campo contaba tres puntos mientras que una anotación (\emph{touchdown}) cinco.\footnote{``Football game'', \emph{Mexican Herald}, 29 de noviembre 1903, p. 5.}

Al contextualizar cuál era el significado del concepto fútbol en el siglo \textsc{xix}, se pone en evidencia que la interpretación dada a la nota de 1889 no es correcta, porque fehacientemente, no se puede confirmar que haga referencia al fútbol \emph{soccer}, por lo que es posible que se trate de otro estilo de fútbol. Ahora bien ¿Cómo saber qué estilo de fútbol fue el que se practicó en el partido reseñado en \emph{El Minero de Pachuca} en 1889?

En primer lugar, difícilmente se podrá establecer de forma indiscutible qué estilo de fútbol se practicó, ya que los datos disponibles para trabajar son escasos (sólo se cuenta con la nota de 1889). Sin embargo, sí es posible construir una figuración a partir de los indicios, esas <<realidades ocultas y marginales>> no evidentes a primera vista por fragmentarias, pero que permiten penetrar en las <<estructuras profundas>> de un hecho histórico (en nuestro caso, los detalles de la nota de 1889).\footnote{Aguirre, <<Indicios>>, pp. 15-38.}

Carlos Antonio Aguirre señala que en las ciencias sociales difícilmente se alcanzan las verdades exactas, sin embargo, es posible establecer aproximaciones cualitativas razonadas y fundamentadas que nos permiten conocer las realidades sociales de algún proceso o fenómeno.\footnote{Aguirre, <<Indicios>>, p. 22.} En base a lo anterior, por medio de los indicios, se desarrollará una interpretación (construida por un razonamiento por analogía y sustentada por diversos datos y testimonios) que establece un <<juicio probable>> sobre cuál estilo de fútbol se practicó en el partido reseñado en la nota de 1889.\footnote{Moreno, <<La investigación>>, p. 83. Bloch, \emph{Apología de la historia}, p. 159, 160. González, \emph{El oficio de historiar}, pp. 244-258.}

En la nota de 1889 se menciona que las minas \emph{El Rosario} y \emph{La Joya} celebraron en Pachuca un partido de fútbol que terminó en una pelea. ¿De verdad, el primer partido o uno de los primeros partidos de fútbol celebrados en México terminó en una riña? ¿La autoridad tuvo conocimiento de los hechos? ¿Qué hizo al respecto? ¿Se levantó un acta? ¿Se detuvo a los rijosos? O de nuevo ¿Es posible que la supuesta pelea se trate de una malinterpretación?

En mi consideración, me parece que hubo una malinterpretación, en este caso de \emph{El Minero de Pachuca}, que interpretó como una riña, la práctica de algún estilo de fútbol como el \emph{rugby}, donde el contacto físico es más constante e hiciera parecer que los participantes estuvieran peleando, pues la revisión realizada a la información empírica muestra que entre 1902 y 1910 (periodo donde únicamente practicaron el fútbol \emph{soccer} equipos formados por británicos) no se suscitó ningún incidente que propiciara una pelea, ni siquiera en el llamado encuentro internacional que era el partido más importante y que más pasiones levantaba, nunca se reportó algún incidente violento o riña.\footnote{``International game january 9'', \emph{Mexican Herald}, 16 de noviembre 1909, p. 10.}

Por el contrario, si algo caracterizaba a los británicos era el gran deportivismo que mostraban tanto dentro como fuera del terreno de juego. Por ejemplo, del \emph{Pachuca Athletic Club} se decía que era de los pocos equipos que poseían <<la cualidad de saber perder y ganar>> porque sin replicar, acataban todas las decisiones de los árbitros y ya sea perdiendo o ganando, siempre demostraban <<la misma caballerosidad con sus contrincantes.>>\footnote{<<Unión Nacional de Fútbol>>, \emph{Teatro y Deportes}, 14 de febrero 1919, p. 5.}

Para los británicos, la práctica deportiva se realizaba entre caballeros, por tanto, preferían perder antes que ser vistos o considerados como tramposos, ventajosos, escandalosos o rijosos. Por ejemplo, en el partido disputado entre el equipo \emph{Rovers} y el \emph{Reforma} el árbitro marcó un penal señalando que R. N. Penny del \emph{Rovers} cometió una falta sobre Claude M. Butlin del \emph{Reforma}. Sin embargo, en una muestra de juego limpio y caballerosidad, Butlin airadamente le reclamó al árbitro que no había sido fauleado.\footnote{``Game with Reforma protested by Rovers'', \emph{Mexican Herald}, 16 de enero 1913, p. 4.} Otro caso similar se presentó en el partido entre Pachuca y España, donde luego de un choque en el área, el árbitro determinó marcar un penal a favor del Pachuca, pero antes de verse favorecido, Hammond, el capitán pachuqueño, le hizo saber al árbitro que no hubo tal falta, sino un choque derivado de la disputa de la pelota.\footnote{Gedeon, <<El colegio de árbitros>>, \emph{Arte y Deportes}, 22 de agosto, 1918, p. 1.}

Para los británicos el juego limpio (\emph{fair play}) era una especie de código moral que, por una parte, contrarrestaba las acciones violentas y deshonestas y por otra, enfatizaba la práctica justa y equitativa de los deportes, pues, de no ser así, se corría el riesgo de que ganar a toda costa fuera más importante y si esto se volviera la norma, la práctica deportiva ya no sería un encuentro amistoso entre iguales, sino una lucha, sin honor, entre enemigos.\footnote{El juego limpio formaba parte de la ideología del deportista \emph{amateur}. El amateurismo, fue <<una creación de la clase media de mediados del siglo \textsc{xix}, que enfatizó la primacía moral del juego limpio sobre la persecución de la victoria...>> Pope, \emph{Patriotic games}, p. 39, 40. Holt, \emph{Sport and the British}, pp. 99-104.}

Para los británicos, el juego limpio fue el hilo conductor para demostrar esa caballerosidad que permitía socializar en harmonía y estrechar los lazos entre los miembros de la comunidad británica radicada en México. Por ejemplo, en el partido que el Club Reforma disputó contra el Pachuca, resulta que el Club Reforma sólo pudo reunir a siete de sus jugadores. Mr. Sobey, el capitán del Pachuca, al tener conocimiento de la problemática, ofreció al Reforma varios de sus jugadores para que completaran el equipo, en lugar de aprovechar la ventaja numérica. Posteriormente, al término del encuentro, el Pachuca ofreció una recepción para celebrar al Reforma, por su victoria en el encuentro.\footnote{``Go to Pachuca today'', \emph{Mexican Herald}, 8 de septiembre 1906, p. 9. <<El séptimo juego de football>>, \emph{El Imparcial}, 3 de noviembre 1906, p. 4. ``Winning of championship how game was played'', \emph{Mexican Herald}, 7 de febrero 1907, p. 11.}

En los primeros años de funcionamiento de la \emph{Mexico Amateur Association Football League} (1902-1910), en lugar de pleitos y disputas, lo que se observa es una gran camaradería entre los equipos británicos. De hecho, fue hasta el ingreso de los equipos españoles y mexicanos cuando empezaron las peleas y los altercados violentos entre futbolistas y espectadores. El cronista deportivo Gedeon señalaba que hasta antes de 1913, el público que asistía a los partidos de fútbol era <<escaso>> y <<escogido>> porque sólo asistían los miembros de la comunidad británica y porque la entrada a los clubes era en exclusiva para los socios. Pero en el momento en que españoles y mexicanos ingresaron a la liga, los partidos se convirtieron en duelo de porras.\footnote{Gedeon, <<Antaño y ogaño>>, \emph{Arte y Deportes}, 1 de agosto 1918, p. 1.}

En efecto, de forma reiterada la prensa estuvo mencionado que cada vez que los equipos españoles y mexicanos jugaban entre sí, las pasiones se desbordaban, porque muchas de las rencillas y conflictos enconados entre ambas sociedades detonaban en el fútbol.\footnote{``España team beats the Mexico eleven'', \emph{Mexican Herald}, 1 de diciembre 1913, p. 3. <<Football>>, \emph{Teatro y Deportes}, 6 de diciembre 1918, p. 2.} Por ejemplo, en el partido <<amistoso>> disputado entre el \emph{Club España} y el \emph{Junior Club} se señala que hubo muchos roces violentos debido a que los jugadores españoles no jugaban <<con la limpieza que el juego requiere y propinaban puntapiés a diestra y siniestra, venga o no el caso.>> Este partido tuvo como saldo que dos jugadores del \emph{Junior} salieran del partido por los golpes recibidos y que el árbitro marcara cuatro penales en contra del \emph{España}.\footnote{Little Ball, <<Football>>, \emph{Revista de Revistas}, 4 de junio 1916, p. 4.}

En otra ocasión, jugaron el \emph{Deportivo Español} y el \emph{Tigres} y en este partido, el árbitro expulsó al mexicano López y al español Cabarga. <<Al primero por liarse a bofetadas con García De León por quien sabe qué provocación de éste y a Cabarga por mezclarse en una pelotera (bronca) que se armó fuera del campo...>>\footnote{<<Por los campos del balompié>>, \emph{Teatro y Deportes}, 4 de enero 1919, p. 5.} Otra cosa que se hace notar, es que los espectadores (en su mayoría españoles), no se comportaban como <<verdaderos \emph{sportmens}>>, ya que no toleraban que los concurrentes mostraran su <<predilección por el club contrario>> pues de inmediato se molestaban y trataban de acallar las voces de animación con gritos, silbidos y <<acciones soeces>> que en más de una ocasión propiciaron que los aficionados invadieran el campo de juego y se liaran a golpes junto con los futbolistas.\footnote{Little Ball, <<Football>>, \emph{Revista de Revistas}, 4 de junio 1916, p. 4. Little Ball, <<Football>>, \emph{Revista de Revistas},
21 de mayo 1916, p. 13. Gedeon, <<Antaño y ogaño>>, \emph{Arte y Deportes}, 1 de agosto 1918, p. 1.}

A diferencia de los británicos, el público español y mexicano se comportaba en el fútbol \emph{soccer} de forma descortés y maleducada, porque desde el inicio de los partidos protestaban y reclamaban las decisiones arbitrales <<con palabras gruesas y malsonantes>>. Si la derrota de su equipo era inminente, arengaban a sus futbolistas para que golpearan a los rivales y cuando esto ocurría, festejaban y aplaudían las patadas y las zancadillas. Finalmente, insultaban y abucheaban a la porra contraria que por supuesto también hacía lo propio.\footnote{Los insultos más reiterados hacia los españoles eran <<Gachupín>>, <<Empeñero>> y <<Muertos de hambre>>. Al parecer, españoles y mexicanos se comportaban en el fútbol, tal y como lo hacían en los toros, donde se silbaba y se abucheaba a los toreros cobardes o de estilos y nacionalidad diferente a la propia. ``España team beats the Mexico eleven'', \emph{Mexican Herald}, 1 de diciembre 1913, p. 3. Gedeon, <<Antaño y ogaño>>, \emph{Arte y Deportes}, 1 de agosto 1918, p. 1. Armando Quimera, <<Football>>, \emph{Arte y Deportes}, 6 de septiembre 1918, p. 24. <<Unión Nacional de Football>>, \emph{Teatro y Deportes}, 31 de enero 1919, p. 4.}

Por ejemplo, en el partido disputado entre el \emph{Junior Club} y el \emph{España B}, luego de una controvertida jugada donde se reclamaba un fuera de lugar, un partidario de los españoles <<profirió palabras injuriosas contra los del Junior...>> Y fue entonces que <<saltó a la defensa un simpatizador de este último y se armó la bronca que dio por resultado que hubiera golpes por ambos lados y que algunos resultaran un poco mal heridos.>>\footnote{Little Ball, <<El último incidente de football>>, \emph{Revista de Revistas}, 26 de noviembre 1916, p. 17.} En otra ocasión, el futbolista mexicano Pancho Gómez, incitado por los aficionados, golpeó en la cara al español Ibarreche quien se negó a responder a la ofensa y cuando se pensaba que el lío se había terminado, los aficionados de ambos bandos <<vuelven a invadir el campo y se arma el zafarrancho de todos los diablos. Palos, piedras, navajas y bastones salen a relucir y causan los consiguientes desperfectos en trajes y sombreros y alguno que otro chichón en las respectivas macetas de los rijosos.>>\footnote{<<Football>>, \emph{Teatro y Deportes}, 6 de diciembre 1918, p. 2.}

Para evitar que los escándalos y las peleas, el cronista <<Guardameta>> sugería que las autoridades debían enviar los suficientes gendarmes para guardar el orden y <<sacar del campo lo mismo a los que usan lenguaje soez como a los escandalosos, ya sean jugadores o del público.>>\footnote{Guardameta, <<Españita y Tigres>>, \emph{Teatro y Deportes}, 11 de enero 1919, p. 14. Don Facundo, <<Football>>, \emph{Teatro y Deportes}, 7 de febrero 1919, p. 5.} Por su parte, la dirigencia de la liga de fútbol comenzó a restringir el acceso a los partidos, implementando una especie de derecho de admisión, pues se buscaba que una audiencia más selecta fuera la que concurriera a los partidos.\footnote{Don Facundo, <<Football>>, \emph{Teatro y Deportes}, 21 de febrero 1919, p. 5.}

Según la prensa, los equipos británicos, practicaban el fútbol con el fin de fraternizar y socializar, por esa razón además de sacrificarse por el bien común de la liga y de conocer ampliamente los reglamentos, se caracterizaban por aceptar estoicamente las decisiones arbitrales que les eran adversas. En cambio, los equipos <<latinos>> se decía que practicaban el fútbol <<de mala fe>>, porque no conocían las reglas (o las interpretaban de forma partidista), ya que su único interés era el de <<obtener los premios de los diferentes concursos donde participaban...>> Y porque su carácter carecía de la <<suficiente fuerza de voluntad>> para contenerse ante las provocaciones.\footnote{<<Para los jueces>>, \emph{Arte y Deportes}, 29 de agosto 1918, p. 4, 5. Alfonso Kuntz, <<El ideal de los deportes>>, \emph{Arte y Deportes}, 4 de octubre 1918, p. 14. Gedeón, <<El público del football>>, \emph{Arte y Deportes}, 26 de octubre 1918, p. 1.}

La incapacidad de dominar los impulsos del temperamento fue considerada una de las <<características raciales>> de los equipos españoles y mexicanos que, en gran medida, fue uno de los factores que propiciaron las constantes fricciones entre los distintos equipos y muy posiblemente, la causa que llevó a los equipos británicos a abandonar la liga y practicar el fútbol de forma privada en el interior de sus clubes.\footnote{``Soccer league has proved big success'', \emph{Mexican Herald}, 22 de enero 1914, p. 3.}

Según muestra la evidencia empírica, los británicos no estuvieron involucrados ni en escándalos ni en peleas, porque un escándalo era visto como una afrenta al honor, así que preferían perder antes de ser tachados como rijosos o malos perdedores. La caballerosidad y el deportivismo que los británicos mostraban cuando practicaban el fútbol \emph{soccer}, hacen poco probable que el partido reseñado por \emph{El Minero de Pachuca} terminara en una pelea, por tanto, considero que el semanario malinterpretó los hechos y a su vez, la nota que publicaron, también ha sido malinterpretada. En su lugar, resulta muy posible que se tratara de un estilo de fútbol diferente al \emph{soccer} y donde el contacto físico fuera recurrente ---como el \emph{rugby}--- y la falta de conocimiento sobre este estilo de fútbol (y del fútbol en general) propició que los testigos presenciales de este encuentro lo interpretaran como una pelea multitudinaria.

Dicho de otro modo, en la nota de 1889 se reportó que los equipos de las minas \emph{El Rosario} y \emph{La Joya} se enfrascaron en una pelea, porque el reportero o los informantes de \emph{El Minero de Pachuca}, no sabían, ni entendían con total certeza qué era el fútbol, ni cuántos estilos de fútbol existían o cómo se practicaban cada uno de ellos. Por esa razón se publicó, sin conocimiento de causa, que los mineros terminaron envueltos en una riña, sin considerar que, en algunos estilos de fútbol, la rudeza y las acciones violentas forman parte del juego.

Según Melvin Adelman, la prensa se ha vuelto la fuente principal para los historiadores, pero, también señala que los reportes hemerográficos no son exactos porque son tendenciosos, parciales y subjetivos.\footnote{Adelman, citado en, Osmond, Phillips, ``Sources'', p. 39, 40.} Sin embargo, aunque la prensa no sea objetiva, es una fuente insustituible para la Historia, porque a través de ella es posible conocer y acceder a los hechos y sucesos que han impactado la vida cotidiana de una sociedad o nación y que al paso del tiempo se convierten en el material empírico que permite conocer las ideas, prácticas, estilos de vida y comportamientos de una sociedad en un período específico.

En lo tocante a las notas deportivas, éstas (con o sin razón) han sido tergiversadas pues resulta común que los reporteros sobredimensionen y exageren los hechos deportivos con el fin de atrapar el interés de los aficionados y así vender más diarios.\footnote{El finado cronista deportivo, Pedro Septién se le apodó <<El Mago>> porque sus crónicas estaban llenas de magia (inventos) que por mucho superaban a los eventos que narraba. Por ejemplo, en 1946 de forma emotiva y memorable narró para la radio mexicana la pelea entre Joe Louis y Bill Con, sin siquiera ver el combate, sólo utilizó los informes del teletipo. Ariel Velázquez, <<Fallece 'El Mago' Septién>>, \emph{El Universal}, 19 de diciembre 2013, \url{http://www.eluniversal.com.mx/deportes/2013/fallece-39el-mago-39-septien-974063.html}} En otras ocasiones, los reporteros no fueron testigos presenciales de los hechos que narran, sino que recibieron la información de terceros, quienes transmitieron su visión e interpretación de los eventos deportivos tal y como los percibieron y a su vez los reporteros reinterpretaron esta información conforme a su criterio y opiniones.

Me parece que una situación similar se presentó en la nota de 1889, es decir, considero que el reportero de \emph{El Minero de Pachuca} no fue testigo presencial del partido de fútbol que se reseñó, sino que un tercero fue quien le pasó los informes del evento. Incluso, aunque hubiera estado presente, difícilmente tendría pleno conocimiento del fútbol en cualquiera de sus estilos, pues recordemos que este fue (se supone) el primer partido disputado en México, por tanto, es posible que no entendiera las acciones donde hubiera rudeza y despliegues violentos y las confundieran con una riña y fue esta interpretación distorsionada, lo que finalmente se publicó en \emph{El Minero de Pachuca}.\footnote{En 1887, \emph{El Nacional}, publicó una nota donde se observa que la prensa mexicana desconocía por completo lo que eran los deportes, ya que se menciona que algunos diarios de la capital publicaron erróneamente que los empleados de los ferrocarriles Nacional y Central iban a organizar un club para dar bailes, cuando en realidad iban a jugar al béisbol. Los diarios confundieron el \emph{baseball} con un baile ya que el término \emph{ball} en inglés se utiliza para referirse tanto a los bailes como a las pelotas, así que cuando se habló de practicar el \emph{baseball}, se pensó que se trataba de un nuevo estilo de baile. <<Bailes y pelotas>>, \emph{El Nacional}, 26 de julio 1887, p. 3.}

¿Es posible que el reportero o el informante de \emph{El Minero de Pachuca} hayan confundido las acciones y jugadas del \emph{rugby} con una pelea? En mi consideración, creo muy posible que algunas jugadas características del \emph{rugby} como el \emph{scrum}, el \emph{ruck}, el \emph{maul}, o el \emph{line out}, pudieron ser interpretadas como una riña multitudinaria, ya que la rudeza y la violencia eran recurrentes y para quienes no estuvieran muy habituados a este tipo de despliegues, parecería que los participantes se estuvieran peleando.

En la actualidad, gracias a la imposición de regulaciones más estrictas, la mayoría de las jugadas del \emph{rugby} se han depurado por lo que la rudeza innecesaria y los roces violentos han disminuido significativamente. Sin embargo, en el siglo \textsc{xix} el \emph{rugby} era uno de los estilos de fútbol más peligrosos, debido a que los golpes, las patadas y demás acciones violentas estaban permitidas, por lo que las lesiones eran comunes (incluso había decesos).\footnote{Entre 1890 y 1893, se registraron 71 fallecimientos en el \emph{rugby} inglés. Chaduneli, <<La evolución del rugby>>, pp. 116 121. Dunning, Sheard, \emph{Barbarians}, pp. 85-87.}

Según Eric Dunning y Kenneth Sheard, el \emph{rugby} era más peligroso que el \emph{soccer} por la gran cantidad de jugadores que tomaban parte en un partido (veinte por equipo) que, al disputarse la posesión del balón, se enfrascaban de forma constante en un \emph{scrum} o \emph{melé} (la principal jugada) donde libremente hacían uso del \emph{hacking} (patear las espinillas del rival) para hacerse del balón, dispersar a los contrarios y abrirse paso hacia la zona de anotación.\footnote{El \emph{Scrum} o \emph{Méle} se define como un combate cerrado, desorganizado y multitudinario. Dunning, Sheard, \emph{Barbarians}, pp. 85-87. Solá, <<Historia del rugby>>, pp. 6-11.}

Aunque cada partido de \emph{rugby} dejaba como saldo varios huesos rotos o dislocados por causa del \emph{hacking}, los partidarios de este estilo de fútbol señalaban que el \emph{hacking} era sumamente necesario para el desarrollo del juego, porque sin esta táctica, el \emph{rugby} se volvería lento, ya que los jugadores se verían inmersos en constantes \emph{scrums}, los cuales se alargarían indefinidamente, volviendo tedioso el juego para los espectadores, pues los equipos se la pasarían empujándose el uno al otro para controlar el balón y difícilmente podrían llevarlo a la zona de anotación.\footnote{Según los partidarios del \emph{rugby}, la importancia del \emph{hacking} era que rompía el \emph{scrum} haciendo posible que los equipos pudieran hacerse con la posesión del balón y acercarse a la zona de anotación y conseguir puntos, además era del agrado de los espectadores, porque hacía del \emph{rugby} un deporte intenso y movido. En varias ocasiones (1863, 1871, 1890), se intentó regular el \emph{hacking}, por ejemplo, se prohibió patear con los talones, se estableció que únicamente se podía patear por debajo de las rodillas y se prohibió usar \emph{navvies}; calzado con punta de metal. Cuando el \emph{hacking} se abolió, el \emph{scrum} tomó mayor relevancia en el juego y se volvió una jugada de fortaleza, lo que hizo necesario que delanteros (también conocidos como \emph{bulldogs} o \emph{battle horses}) de mayor estatura y fortaleza tomaran parte en el juego para que se encargaran de empujar a los rivales para ganar el balón y una mejor posición en el terreno de juego. Dunning, Sheard, \emph{Barbarians}, pp. 69-101. Chaduneli, <<La evolución>>, pp. 116 121. Collins, \emph{A social history}, p. 20, 21. Solá, <<Historia del rugby>>, p. 8.}

En mi consideración, fue un \emph{scrum} (la disputa por la posesión del balón) o fue el \emph{hacking} (la táctica implementada para romper el \emph{scrum} y donde los participantes indiscriminadamente se pateaban) lo reseñado en 1889 por \emph{El Minero de Pachuca} como una pelea. Sin embargo, se hace necesario presentar otros testimonios que le den mayor peso a esta interpretación que se está construyendo.

¿Por qué el \emph{rugby}, a pesar de lo rudo y violento, era uno de los estilos de fútbol más populares entre los británicos? Mientras que para algunos sectores de la sociedad británica (principalmente la clase alta) la violencia era algo repulsivo y representaba una contradicción para los códigos morales.\footnote{Algunos clubes que en un principio practicaban el \emph{rugby}, decidieron cambiarse al \emph{soccer} porque consideraron que el \emph{rugby} era un estilo sumamente peligroso. Harvey, \emph{Football}, p. 211.} Para otros, (en específico los sectores populares) el \emph{rugby} y sus despliegues violentos resultaban más atractivos y más emocionantes que el fútbol \emph{soccer}.\footnote{Dunning, Sheard, \emph{Barbarians}, p. 93}

El \emph{rugby} era atractivo porque la violencia desplegada en cada partido fue admirada y se interpretó como <<pruebas>> de virilidad y fuerza entre naciones, regiones, comunidades o grupos.\footnote{Szymanski, Zimbalist, \emph{National pastime}, p. 34. Dunning, Sheard, \emph{Barbarians}, p. 105.} De hecho y aunque pudiera parecer contradictorio, el \emph{rugby} entre 1870 y 1880 fue más popular que el \emph{soccer}, porque en ese periodo, el \emph{rugby} era un estilo de fútbol colectivo y solidario, mientras que el \emph{soccer} (la forma de jugarse) era más individualista, por esa razón, los practicantes del \emph{rugby} eran el doble de numerosos que los de \emph{soccer}.\footnote{Collins, \emph{Rugby's}, p. 9. Collins, \emph{A social history}, p. 25. Guoth, <<Loss of identity>>, pp. 187-207. Harvey, \emph{Football}, p. 208.}

Sin embargo, el \emph{rugby} perdería terreno respecto al \emph{soccer}, porque dentro del \emph{rugby} se suscitó una lucha de clases que dio origen a dos organismos (socialmente opuestos), que se disputaban el control de este estilo de fútbol. El \emph{soccer} y el \emph{rugby} no solo se distinguen por las habilidades y las técnicas establecidas para su práctica (el \emph{soccer} el uso de los pies o \emph{dribbling}, el \emph{rugby} el uso de las manos o \emph{handling}) sino también por las ideas, prácticas y los valores culturales que se les han adherido a lo largo de su historia.\footnote{Lloyd Hill, ``Football as code'', p. 14.}

Según Tony Collins, las actividades deportivas son un micro cosmos donde se reflejan los conflictos de la sociedad que las cultiva (nacionalismo, raza, clase, género). En el caso del \emph{rugby}, este estilo se volvió un sitio de conflicto social entre las expresiones de clase de los sectores populares y los códigos culturales de la elite.\footnote{Collins, \emph{A social history}, p. \textsc{xv}. White, ``Rugby union'', p. 58.} Hasta antes de 1870, el fútbol \emph{rugby} en el Reino Unido era practicado principalmente por los miembros de la elite, por lo que este estilo de fútbol tenía cierto prestigio social. El \emph{rugby} era una actividad cultivada en varias \emph{Public Schools} de prestigio como Rugby, Cheltenham o Marlborough y que se empleaba para instruir lecciones morales y moldear la personalidad de los hijos de la elite (para evitar la masturbación y la homosexualidad), para mantener saludable el cuerpo y la mente, además de templar y fortalecer su carácter y hacerlos aptos para gobernar.\footnote{Szymanski, Zimbalist, \emph{National pastime}, p. 34, 35. Dejonghe, \emph{The popularity of football}, p. 2. Collins, \emph{A social history}, pp. 11-26. Dunning, Sheard, \emph{Barbarians}, p. 93.}

En un principio, el \emph{rugby} era practicado en exclusiva por alumnos de las \emph{Public Schools} y por los miembros de clubes privados.\footnote{Collins, \emph{Rugby's}, p. 9.} Sin embargo, el jugar reiteradamente con los mismos coequiperos durante una temporada (seis meses) volvía la práctica tediosa, aburrida y rutinaria, así que se hizo necesario buscar otros clubes y equipos a quienes enfrentarse, sin importar que pertenecieran a otro estatus social.\footnote{Collins, \emph{A social history}, p. 13. Dunning, Sheard, \emph{Barbarians}, p. 78.}

Fue la necesidad de competir contra otros clubes y equipos lo que abrió la puerta a los sectores populares para practicar el \emph{rugby}. En primera instancia, la participación de las clases bajas en el \emph{rugby} se dio a partir del patrocinio de los terratenientes y del tutelaje de organizaciones diversas como iglesias, escuelas o tabernas, pero posteriormente, las clases populares comenzaron a formar sus clubes y equipos en torno a sus centros de trabajo como los centros textiles en Manchester o alrededor de las minas en Cornwall.\footnote{Harvey, \emph{Football}, p. 208. Dunning, Sheard, \emph{Barbarians}, p. 102. Collins, \emph{Rugby's}, pp. 16-25. White, ``Rugby union'', p. 57. Keech, ``England'', p. 8.}

Otros factores que facilitaron la participación de los sectores populares en el \emph{rugby} fueron el crecimiento de las áreas urbanas, el incremento del tiempo libre gracias a la reducción de la jornada laboral (en 1870 se estableció trabajar medio día los sábados) y la mejora en la red ferroviaria que en conjunto permitieron la expansión del \emph{rugby} a otras áreas y regiones, estableciendo una red <<interdependiente>> donde de forma paulatina se integraban nuevos equipos y clubes, desvaneciendo con ello el aislamiento local y haciendo posible la construcción de rivalidades deportivas interregionales.\footnote{Dunning, Sheard, \emph{Barbarians}, p. 78. Dunning, Curry, \emph{Association football}, p. 120, 121. Collins, \emph{Rugby's}, pp. 16-24. White, ``Rugby union'', p. 57. Keech, ``England'', p. 7.}

Fueron las rivalidades entre clubes, ciudades y regiones las que propiciaron que las clases bajas se integraran a la práctica del \emph{rugby}.\footnote{En un periodo de diez años (entre 1875 y 1885) la mayoría de los clubes de \emph{rugby} en Yorkshire provenían de las clases trabajadoras. El \emph{rugby} fue para las clases bajas una actividad que las sacaba de la rutina de la vida laboral. Una de las rivalidades interregionales más importantes fue la de Yorkshire y Lancashire, sólo superada por el encuentro internacional de \emph{rugby} entre Inglaterra y Escocia. Harvey, \emph{Football}, p. 208. Dunning, Sheard, \emph{Barbarians}, p. 105. Collins, \emph{Rugby's}, pp. \textsc{xix}-14. Collins, \emph{A social history}, pp. 21-24.} Inicialmente se pensó que el \emph{rugby} sería capaz de eliminar las barreras y los conflictos entre las clases sociales y de promover la solidaridad y cooperación entre ellas. Sin embargo, en lugar de conciliar, el \emph{rugby} exacerbó las divisiones ya existentes, porque para las clases bajas el \emph{rugby} fue un espacio que les permitió participar en la vida pública (antes negada) y construir una identidad colectiva y expresar los valores culturales con los que se identificaban.\footnote{Tranter, \emph{Sport, economy}, pp. 37-56. Gálvez, Stavrianeas, <<El rugby>>, p. 81. White, ``Rugby union'', p. 57.}

En efecto, la \emph{Rugby Football Union} (el organismo que regulaba el \emph{rugby}), en un principio, evaluó como exitoso el crecimiento del \emph{rugby} en la década de 1870 a 1880. Pero posteriormente, la participación de las clases bajas fue interpretada como una intromisión que amenazaba <<la exclusividad de su juego>>.\footnote{Dunning, Sheard, \emph{Barbarians}, p. 45.} A los equipos de la elite les molestaba ser derrotados por quienes consideraban como <<inferiores>>, pero también les preocupaba que las clases bajas se estaban apropiando del \emph{rugby}, ya que estaban cambiando sus aspectos técnicos (se volvió más rudo) y que lo estaban encaminando hacia el profesionalismo.\footnote{Argumentando que el \emph{rugby} se había vuelto más violento y no deseaban exponerse a una lesión, los equipos de la elite comenzaron a negarse a jugar con los equipos de las clases bajas. En algunos casos, luego de ser derrotados, los clubes de la elite se desbandaban o cambiaban de estilo y empezaban a jugar \emph{soccer}. Dunning, Sheard, \emph{Barbarians}, p. 103. Dunning, Curry, \emph{Association football}, p. 121. Collins, \emph{Rugby's}, pp. 9-31. White, ``Rugby union'', p. 57. Collins, \emph{A social history}, pp. 24-45.}

El \emph{rugby} al igual que el \emph{soccer}, ha sido una actividad que permite la construcción de identidades de diversos tipos. En un nivel general el \emph{rugby} ha sido utilizado para reforzar la identidad nacional, en un nivel más particular, se ha utilizado para reforzar la masculinidad. Tanto la nacionalidad como la masculinidad no son cualidades inherentes, sino aprendidas por lo que deben ser reiteradas de forma constante.\footnote{Kassimeris, ``The semiotics'', pp. 190-202.}

La masculinidad es descrita por Bourdieu como una representación social que se estructura mediante las diferencias entre los géneros y que requiere ser reiterada de forma constante a través de la práctica de actividades estereotipadas como propias de cada sexo, para que sus atribuciones sean reconocidas por los miembros de una sociedad en específico.\footnote{Bourdieu, \emph{Dominación}, pp. 21-40.} Dicho de otro modo, la masculinidad no es sólo un aspecto biológico, sino también una condición social que se adquiere y se reafirma en el día a día para que un individuo sea reconocido por los miembros de una sociedad y que ha sido vista como un criterio de diferenciación con lo femenino y con los individuos que no se ajustan con el modelo establecido de masculinidad.

Según Sergio Moreno, el nacer varón no hace a un hombre, sino que se debe demostrar la masculinidad en todo momento, ante otros hombres y de manera pública realizando actividades estereotipadas como propias del sexo masculino, actividades que primordialmente requieren fuerza física, violencia o peligro, todo esto para ser reconocido socialmente como un hombre, por ser capaz de desplegar fortaleza y valentía.\footnote{Moreno, \emph{Masculinidades}, pp. 11-18.} En el caso del \emph{rugby}, este deporte fue codificado como un \emph{habitus} (espacio cultural) donde las clases bajas reproducían y consumían una serie de bienes y valores culturales que les eran propios o que pretendían adquirir.\footnote{Los valores culturales son generados y adquiridos por medio de la socialización. Los valores culturales sólo pueden ser consumidos al comprenderse su significado. Hill, ``Football as code'', pp. 14-17.}

Es decir, el \emph{rugby} fue visto como una actividad que podía recuperar y reafirmar la masculinidad de los individuos, ya que la participación en un partido de \emph{rugby} implicaba realizar un despliegue físico intenso y violento sobre todo en el \emph{scrum}, donde indiscriminadamente, los jugadores de ambos equipos se daban de patadas.\footnote{En el \emph{rugby} se utilizaban unas botas (llamadas \emph{navvies}) con punta de metal endurecida para patear con mayor violencia. También fue común que los jugadores, como una muestra de valentía, desdeñaban deliberadamente utilizar algún tipo de protección para las piernas. Dunning, Sheard, \emph{Barbarians}, p. 71. Collins, \emph{Rugby's}, p. 4.} El \emph{scrum} fue lo que le dio al \emph{rugby} su reputación de deporte violento y fue también la gran oportunidad que tenían los individuos para ser reconocidos como hombres por sus pares, demostrando su valor y fortaleza física en exhibiciones públicas, donde se disputaban la supremacía deportiva y su honor como varones realizando despliegues rudos y violentos.

Para Dunning y Curry, la rudeza desplegada, fue lo que hizo atractivo al \emph{rugby} para los grupos donde tradicionalmente la masculinidad seguía siendo un rasgo distintivo de la conducta de los varones. Por esa razón, el \emph{rugby} fue el estilo más arraigado en las zonas mineras.\footnote{Dunning, Sheard, \emph{Barbarians}, p. 101.} Lo relevante de estos datos para nuestro estudio es que los personajes a los que tradicionalmente se les ha atribuido la introducción del fútbol \emph{soccer} a México, eran mineros provenientes de Cornwall, una región donde la práctica del \emph{rugby} era generalizada y formaba parte de su identidad cultural.\footnote{Seward, ``Cornish rugby'', pp. 78-94.}

En efecto, la versión más arraigada que pretende explicar el surgimiento del \emph{soccer} en México señala que, un grupo de mineros <<ingleses>> (también conocidos como \emph{cornish}) provenientes de Cornwall se asentaron en Pachuca y trajeron consigo el fútbol \emph{soccer}.\footnote{Angelotti, \emph{La dinámica}, pp. 27-36. Angelotti, \emph{Chivas y Tuzos}, p. 173. Ovalle, \emph{Historia del fútbol}, pp. 40-56. Meneses, Ávalos, <<La investigación>>, p. 36, 37. Zamora, \emph{El equipo}, pp. 4-13. Cit, \emph{El libro de oro}, pp. 9-23.} Sin embargo, los indicios revelan que los mineros de Cornwall practicaban el \emph{rugby}, no el \emph{soccer}, ya que en esta región el \emph{rugby} tenía una larga tradición histórica, incluso se menciona que en Cornwall se inventó el \emph{rugby}.\footnote{En Cornwall, desde hace cuatro siglos se practica el \emph{Hurling}, una actividad que se considera el precursor del \emph{rugby} moderno, ya que algunas de las principales jugadas y reglas del actual \emph{rugby} (como el \emph{scrum} y la ilegalidad del pase frontal) ya existían en el \emph{Hurling}. Seward, ``Cornish rugby'', p. 80,81.}

Al igual que la minería y la religión, el \emph{rugby} en Cornwall era sumamente importante en el desarrollo de la vida cotidiana, por esa razón se formó la \emph{Cornish Rugby Football Union}, un organismo que estuvo fomentando y regulando el \emph{rugby} en Cornwall y áreas circundantes y que fue factor esencial para ganar el campeonato de la región, la medalla de plata representando a Inglaterra en las Olimpiadas de 1908 y para lograr la internacionalización del \emph{rugby} de Cornwall, al establecer vínculos que permitieron la visita de las selecciones de Nueva Zelanda, Sudáfrica y Australia.\footnote{Las características que mejor definían a los habitantes de Cornwall era ser minero, metodista y practicante del \emph{rugby}. La influencia de la \emph{Cornish Rugby Football Union} incluía las áreas de Devon, Somerset y Gloucestershire y según John Bale, Cornwall, proporcionalmente, tenía la mayor cantidad de clubes de \emph{rugby} afiliados de toda Inglaterra. Seward, ``Cornish rugby'', pp. 84-92.}

Estos antecedentes nos indican que el \emph{rugby} era muy importante para los \emph{cornish}, no sólo como un símbolo de masculinidad, sino también fue un elemento de identidad nacional aspecto clave para la difusión del rugby a otros países como Sudáfrica, Australia o Nueva Zelanda, donde gran cantidad de mineros \emph{cornish} fueron a trabajar luego del estancamiento de la economía que por varias décadas mantuvo en la pobreza a la región.\footnote{Seward, ``Cornish rugby'', pp. 83-92.}

Los \emph{cornish}, sea cual fuera el lugar donde migraran, seguían practicando el \emph{rugby} ya que, como ya se ha mencionado, ha sido un elemento cultural que contribuyó a mantener y reforzar su masculinidad y su sentido de pertenencia y en aquellos sitios donde hubo una gran concentración de \emph{cornish}, el \emph{rugby} se estableció como el estilo de fútbol preponderante. Por ejemplo, en Taranaki, Nueva Zelanda, hubo una importante concentración de mineros provenientes de Cornwall y de Devon un condado vecino.\footnote{Guoth, ``Loss of identitiy'', p. 190.} Misma situación encontramos en las zonas de Witwatersrand y Transvaal, en Sudáfrica, donde miles de \emph{cornish} llegaron a trabajar en las minas de oro y diamantes y donde también el \emph{rugby} se estableció como el principal estilo de fútbol.\footnote{Según John Nauright, cerca del 25\% de los mineros que trabajaban en Witwatersrand provenían de Cornwall. Por otra parte, se estima que entre 17500 y 18000 mineros \emph{cornish} fueron a trabajar a la zona de Transvaal. Finalmente, el censo británico de 1881 señalaba que, derivado de la migración, la población de Cornwall había disminuido 8.9\% (cerca de 50 mil personas) con respecto a 1871. Nauright, ``Cornish miners'', pp. 1-22. Hill, ``Football as code'', pp. 14-20.}

En el caso mexicano, algunas versiones señalan que la introducción del fútbol \emph{soccer} se debe a los primeros \emph{cornish} (o sus descendientes) que se asentaron en las zonas mineras del estado de Hidalgo entre 1825 y 1850.\footnote{La migración \emph{cornish} era de dos tipos: temporal o permanente. En el primer tipo, los individuos trabajaban por periodos de tres a seis meses y luego retornaban a Cornwall. En el otro, luego de migrar y laborar por algún tiempo, enviaban por sus familias. En México, el tipo temporal fue el más recurrente. Nauright, ``Cornish miners'', p. 11.} Sin embargo, se debe considerar que la primera oleada de \emph{cornish} fue un grupo hermético,\footnote{Los primeros \emph{cornish} asentados en México no fueron tan numerosos y se mantuvieron segregados de la sociedad mexicana por al menos una década. Por otra parte, Michael Costeloe señala que en 1851 había un estimado de 745 británicos radicando en México. Costeloe, ``To bowl'', p. 117. Saavedra, Sánchez, <<Minería y espacio>>, pp. 82-101. ``The Mexican connection'', pp. 6-13.} que durante la primera mitad del siglo \textsc{xix} tuvo que lidiar con la inestabilidad política y económica del país,\footnote{Según Silvestre Villegas, la información que los británicos tenían acerca de las vías de comunicación no fue fidedigna y esto generó retrasos y pérdidas para las empresas mineras, ya que la ausencia de ferrocarril y el mal estado de los caminos, impidieron la transportación de maquinaria de Veracruz a las zonas mineras, causas que a la larga contribuyeron con la bancarrota y el cierre de las minas. Villegas, <<Los intereses británicos>>, 2010, pp. 337-341. Saavedra, Sánchez, <<Minería y espacio>>, pp. 88-98.} con lo complejo de las costumbres y tradiciones de la sociedad mexicana y finalmente con el rompimiento de las relaciones entre México y el Reino Unido.\footnote{El cierre de las minas, la bancarrota de las empresas británicas y el rompimiento de las relaciones diplomáticas entre México y la Gran Bretaña en 1861, provocaron que los \emph{cornish} abandonaran la zona minera de Hidalgo para buscar trabajo en otras regiones mineras o para regresar a su país. Saavedra, Sánchez, <<Minería y espacio>>, pp. 88-98. Alatriste, <<Aspectos económicos>>, p. 136.}

También, se debe tener en cuenta que el fútbol \emph{soccer} surgió hasta 1863 y según Gavin Kitching, tuvo que pasar una década para que se difundiera por todo el Reino Unido y lograra establecerse como uno de los pasatiempos favoritos de la sociedad británica. Por lo tanto, resulta poco probable que la primera oleada de \emph{cornish} haya introducido el fútbol (ya sea \emph{soccer} o \emph{rugby}) a México.\footnote{Kitching, ``Old football'', pp. 1736-1738.} Según muestran los indicios, fue la segunda oleada de \emph{cornish} la que introdujo el fútbol a México.\footnote{Fue a partir del porfiriato que la situación política y económica del país se vuelven favorables para la introducción del fútbol. En el porfiriato se inició con la modernización del país y se reestablecieron las relaciones diplomáticas y comerciales con el Reino Unido, aspecto que reactivaría el flujo de migrantes de la Gran Bretaña a México. Heath, <<Los primeros escarceos>>, p. 88. Pérez, <<Reestablecimiento>>, p. 32. Beezley, <<Estilo porfiriano>>, pp. 266-279.} Pero esta segunda oleada de \emph{cornish}, no introdujo el fútbol \emph{soccer}, sino el fútbol \emph{rugby} y el primer partido celebrado en ese estilo de fútbol, fue el reseñado en 1889 por \emph{El Minero de Pachuca}.

Los \emph{cornish} dejaron tras de sí varios indicios aun perceptibles de su presencia en México.\footnote{Durante el porfiriato, en la zona minera del estado de Hidalgo, estuvieron radicando un estimado de 350 mineros \emph{cornish}. ``The Mexican connection'', p. 11.} Estos indicios son algunos de sus rasgos culturales más distintivos, como los \emph{pastes}, la religión metodista, la arquitectura y por supuesto el fútbol \emph{rugby}.\footnote{``The Mexican connection'', p. 11. Saavedra, Sánchez, <<Minería y espacio>>, pp. 88-98. Alatriste, <<Aspectos económicos>>, p. 96.} Todas estas actividades siguieron siendo cultivadas por los \emph{cornish} al arribar a México, ya que les eran significativas porque les permitían reafirmar su identidad cultural y la masculinidad.\footnote{Carlo Ginzburg señala que el hombre aprendió cómo rastrear y cazar porque a lo largo del tiempo supo entender la forma de vida de las presas, al grado de poder anticiparse a sus movimientos. Es decir, el cazador supo reconocer e interpretar los indicios que una presa deja tras de sí como huellas, pelos, heces y que en su conjunto le han permitido hacer una lectura interpretativa su comportamiento. Aguirre, <<Indicios>>, pp. 15-38. Ginzburg, <<Indicios>>, p. 7.}

Si para los \emph{cornish} el \emph{rugby} era un importante medio para reforzar su masculinidad, ¿Por qué los \emph{cornish} dejarían de practicar el \emph{rugby}? ¿Habría alguna razón para ya no practicarlo? ¿Es posible que los \emph{cornish} al llegar a México renunciaran a la práctica \emph{rugby} y se cambiaran al \emph{soccer}? En mi opinión, los \emph{cornish} no dejaron de practicar el \emph{rugby}, más aún, siempre estuvieron pendientes de lo que acontecía en este deporte, por ejemplo, en 1908 el \emph{Mexican Herald} publicó que los \emph{cornish} radicados en Pachuca estaban de regocijo y sentían muy orgullosos luego de enterarse que el equipo representativo de Cornwall había conquistado el campeonato regional de \emph{rugby}.\footnote{``Cornish are rejoicing'', \emph{Mexican Herald}, 29 de abril 1908, p. 7.}

El júbilo desbordado de los \emph{cornish} al saber que el equipo de su terruño se había alzado con el campeonato, nos indica que el interés por el \emph{rugby} continuaba persistiendo entre los \emph{cornish} y que esta actividad seguía teniendo gran importancia social. Ahora bien, ¿Si los mineros \emph{cornish} introdujeron el \emph{rugby}, entonces quién introdujo el \emph{soccer} a México? ¿Será acaso que los \emph{cornish} primero practicaron el \emph{rugby} y luego se cambiaron al \emph{soccer}? ¿De qué forma se puede explicar el surgimiento del fútbol \emph{soccer} en México?

En mi parecer, los \emph{cornish} no dejaron de practicar el \emph{rugby} y tampoco cambiaron de estilo, sino que de manera errónea se les atribuye la introducción del \emph{soccer}, ya que no se considera la posibilidad de que existieran dos equipos: uno de \emph{rugby} y uno de \emph{soccer}. William Beezley reporta que en 1895 en Pachuca <<los mineros \emph{cornish} organizaban competencia de lucha vernácula (¿\emph{hurling} o \emph{rugby}?) y sus primos ingleses jugaban fútbol.>> Líneas más adelante Beezley señala que en la ciudad de México los ingleses asistían a los partidos de \emph{cricket} en el \emph{Reforma Athletic Club} y menciona también que <<habían formado un equipo de \emph{rugby} para retar al del \emph{Rugby Union Football Club} de Pachuca...>>\footnote{Beezley, <<Estilo Porfiriano>>, p. 267, 268.}

El que existiera un equipo de \emph{rugby} y un equipo de \emph{soccer}, no se debe sólo a la preferencia de un estilo u otro, sino también, por la divergencia de clase entre ambos equipos. En efecto, el equipo de \emph{rugby} fue formado por los mineros \emph{cornish} (mineros miembros de la clase baja) y cuyo primer partido se celebró en 1889, según la reseña \emph{El Minero de Pachuca}. El \emph{soccer} en cambio, surgió en la ciudad de México en 1892 (en su forma más organizada) y era practicado por miembros de la elite (contratistas, administradores, ingenieros, empleados de oficina y también inversionistas de las empresas mineras).

A diferencia de los mineros \emph{cornish}, los practicantes del fútbol \emph{soccer} eran miembros de las clases media y alta, gozaban de una buena posición social, algunos habían estudiado en universidades de prestigio (Oxford), tenían su propio negocio y mantenían vínculos cercanos con la aristocracia y las más altas autoridades británicas residentes en México. Por ejemplo, se menciona que el escoces McNabb, personaje que durante varios años estuvo jugando al \emph{soccer} en diversos equipos como \emph{Pachuca} o \emph{British club}, partiría rumbo a Parral, Chihuahua donde iba a trabajar en una compañía minera de ese sitio. McNabb fue despedido con una cena en su honor donde estuvo presente el vicecónsul británico, así como sus compañeros de equipo. También, los destacados jugadores H. F. M. Crookshanks y J. J. McFarlane dejaron la ciudad de México para ir a radicar a Puebla, donde habían abierto una funeraria. Ambos eran graduados de la universidad de Oxford.\footnote{``Another departure'', \emph{Mexican Herald}, 12 de enero 1903, p. 2. ``Good football men leave'', \emph{Mexican Herald}, 30 de diciembre 1903, p. 8.}

En México, al igual que en otros países, se introdujeron y se practicaron varios estilos de fútbol (incluso en algunos momentos coexistieron), pues hubo partidarios de ambos tipos de fútbol, sin embargo, al final se establecería la versión que contaba con más practicantes y más apoyos, pues no se debe perder de vista que la implantación de cualquier versión de fútbol requería una inversión de esfuerzos, tiempo y dinero y en algunos casos los esfuerzos, el tiempo y el dinero invertido no fueron suficientes para lograr captar la atención y poder establecer en definitiva la práctica de algún estilo de fútbol.\footnote{México pudo ser un país de \emph{rugby} como Sudáfrica o Nueva Zelanda, sin embargo, el \emph{rugby} no pudo establecerse en definitiva porque en México no hubo tantos practicantes de este estilo (a lo mucho unos cientos de \emph{cornish}) como en aquellos países donde hubo miles. Los escasos practicantes hacían muy difícil encontrar rivales con quien enfrentarse, por lo que los equipos terminaban desbandándose. También es posible que las críticas de la sociedad y la prensa mexicanas hacia la violencia del \emph{rugby} hicieron que este estilo dejara de practicarse, pues a partir de 1896 ya no se encontraron notas referentes a la práctica del \emph{rugby}. Guoth, ``Loss of identity'', pp. 189-202. Nauright, ``Cornish miners, pp. 1-22. Hill, ``Football as code'', pp. 14-20. ``The Mexican connection'', p. 11.}

En nuestro país, la existencia de dos equipos que practicaban distintos estilos de fútbol, nos hace considerar, por una parte, que la nota de 1889 se está malinterpretando y se confunde la práctica del \emph{rugby} con la introducción del \emph{soccer}. También, se hace evidente que la historia del fútbol mexicano es diferente y más compleja de como previamente se había pensado, pues los indicios muestran que Pachuca no es la cuna del fútbol \emph{soccer} tal y como se ha dicho, ya que las evidencias disponibles han sido malinterpretadas y en última instancia han establecido una visión distorsionada del desarrollo histórico del fútbol mexicano.

El principal problema con la historia del fútbol mexicano, es que mayoritariamente ha sido contada por no historiadores, que únicamente relatan anécdotas, pero que poco o nada saben de cómo hacer Historia. Por otra parte, la aparición de nueva información modificará sustancialmente lo que hasta hoy se sabe acerca del surgimiento y desarrollo del fútbol mexicano y, asimismo, se modificarán los derroteros historiográficos.

De hecho, ya está cambiando, puesto que los datos más recientemente encontrados nos señalan que el primero de noviembre de 1891, en San Cristóbal de las Casas, Chiapas, se celebró el que hasta ahora es el partido de fútbol \emph{soccer} más antiguo del que se tenga registro. Este encuentro (catalogado como amistoso) fue disputado entre los equipos \emph{Pearson's Wanderers} y \emph{San Cristobal Swifts} y fue presenciado por una gran cantidad de <<finas personas>> de la localidad. En la nota se menciona que la mayoría de los jugadores del \emph{San Cristobal}, nunca había practicado el fútbol \emph{soccer}, por lo que estaban en franca desventaja con sus contrarios, que desde tiempo atrás lo practicaban y esto quedó de manifiesto en la gran habilidad mostrada por los \emph{players} del \emph{Wanderers} para <<driblar>> a sus contrarios.\footnote{Al final, el partido terminó un gol a cero a favor del \emph{Pearson's Wanderers}. \emph{Spectator}, ``Football at San Cristobal'', \emph{Daily Angloamerican}, 3 de noviembre 1891, p. 2.}

\newpage
\pagestyle{empty}
\null\vfill

\newpage
\pagestyle{fancy}
\fancyhf{}
\fancyhead[CO]{\small\textit{Historia e historiografía del fútbol mexicano}}
\fancyhead[CE]{\small\textit{Miguel Ángel Esparza Ontiveros}}
\fancyfoot[RO,LE]{\small\thepage}
\renewcommand{\headrulewidth}{0pt}
\pagenumbering{arabic}
\setcounter{page}{99}
\chapter*{\centering\mdseries\Large\textsc{Conclusiones}}
\addcontentsline{toc}{chapter}{\mdseries Conclusiones}

\noindent El presente ejercicio revisionista, tuvo como objetivo principal el de establecer las bases de una nueva agenda historiográfica del fútbol mexicano. En primer lugar, se hace necesario revisar y cuestionar todas las premisas que hoy en día se presentan como las <<verdades>> únicas y absolutas para explicar el origen del fútbol en México, pues a medida que se realicen más estudios revisionistas de la historia del fútbol mexicano, nuevos conocimientos y nuevos derroteros historiográficos surgirán y con ellos, nuevas premisas y explicaciones de cómo surgió y se desarrolló el fútbol \emph{soccer} en México.

Sin importar qué tantos libros y artículos se hayan publicado, la historia del fútbol en México apenas comienza a escribirse (académicamente hablando) pues hasta ahora, sólo se ha consultado una mínima parte de todo el universo de fuentes disponibles, principalmente, las fuentes referentes a ciudades como Pachuca, Real del Monte, Orizaba y la ciudad de México, por lo que todavía queda gran cantidad de información por explorar, que, dicho sea de paso, resulta humanamente imposible que una sola persona sea capaz de consultar y agotar por completo, por tanto, lejos estamos de poder establecer con certeza en dónde se ubica la cuna del fútbol mexicano, eso será tarea de las siguientes generaciones de historiadores.

En contraparte, gracias a la digitalización de archivos y documentos, se han descubierto nuevos datos pertenecientes a otras ciudades y regiones del país y que indican que los orígenes del fútbol mexicano son diferentes a como se había pensado. Mientras que por medio siglo de forma reiterada se estuvo señalando que Pachuca era la cuna del fútbol mexicano, la información empírica más reciente muestra que en 1891, en San Cristóbal de las Casas, Chiapas, se celebró uno de los primeros partidos de fútbol \emph{soccer} de los que se tiene registro.\footnote{``Football at San Cristobal'', \emph{Daily Angloamerican}, 3 de noviembre 1891, p. 2.} En ese mismo sentido, fue en la ciudad de México el 2 octubre de 1892, cuando por primera vez se disputó un partido de fútbol \emph{soccer} con apego a las reglas y donde complementariamente se fundó un club con toda la formalidad (el \emph{Mexican Athletic Club}).\footnote{``Football'', \emph{Two Republics}, 21 de septiembre 1892, p. 4.}

Por otra parte, recientemente apareció una nota del año 1889 donde se reseña que en Pachuca se celebró un partido de fútbol (que supuestamente terminó en una pelea) y la cual se ha presentado como la evidencia que comprueba que fue en Pachuca donde por primera vez se practicó el fútbol \emph{soccer}. Sin embargo, considero que dicha nota ha sido malinterpretada. En primer lugar, dudo mucho que el reportero o el informante de \emph{El Minero de Pachuca} conocieran los diferentes estilos de fútbol, por lo tanto, es muy probable que se tratara de un partido de fútbol \emph{rugby} más que de una pelea.

Tampoco creo que \emph{El Minero de Pachuca} estuviera interesado en dar cobertura a la naciente práctica del fútbol, sino que el verdadero objetivo de la nota era el de criticar el comportamiento de los considerados como civilizados y superiores, haciendo del conocimiento público la presunta pelea que protagonizaron los británicos, por esa razón, ya no se han encontrado otras notas donde se haga referencia a otros partidos de fútbol.

También creo que esta nota ha sido anacrónicamente malinterpretada, lo cual ha producido una idea distorsionada del origen del fútbol mexicano, pues por el hecho de que aparece la palabra <<fútbol>>, se piensa que hace referencia al fútbol \emph{soccer}, sin considerar que podría tratarse de otro estilo, pues para ese momento, el término fútbol se utilizaba para referirse al balón y como un genérico para los diferentes estilos de fútbol.

Historiográficamente hablando, en la historia del fútbol mexicano existen algunos problemas que han limitado el establecimiento de un campo de estudio especializado y autónomo dedicado a cultivar esta subdisciplina y me refiero, en primer lugar, a que los datos empíricos no son comprobados ni contextualizados, sino que se presentan tal cual se encuentran en los documentos y son interpretados literalmente desde la perspectiva del presente y, en segundo término, el interés principal ha sido el descubrir cuándo surgió el fútbol \emph{soccer}, en lugar de explicar sus causas (cómo surgió y cómo fue adoptado en cada ciudad y región de nuestro país).

También se hace necesario evitar las generalizaciones y en su lugar se deben realizar estudios con una perspectiva más local o regional, pues se debe considerar que todavía hay áreas, ciudades y regiones donde el fútbol no se ha estudiado, por tanto, resulta erróneo tratar de establecer una interpretación general de la historia del fútbol mexicano, a partir del análisis de una sola localidad, región o incluso equipo, pues la historia del fútbol mexicano no fue homogénea, ni lineal, ni intencionada y ni tampoco intervino un solo personaje o grupo, sino que en ella tomaron parte gran cantidad de personas en diferentes áreas y regiones del país, dentro de un proceso interdependiente y no planificado.

En su lugar, me parece que se debe seguir un camino similar al trazado por la historiografía revisionista de la Revolución mexicana y me refiero a que se deben realizar la mayor cantidad posible de estudios locales y regionales de la historia del fútbol mexicano, para posteriormente realizar una síntesis general (mediante una perspectiva comparada), para ubicar con mayor certeza dónde se localiza la pretendida cuna del fútbol mexicano.\footnote{González, \emph{El oficio de historiar}, pp. 65-111. Mangan, ``Missing Men'', p. 170. Serrano, <<Historiografía regional mexicana>>, pp. 49-57. Knight, <<Interpretaciones recientes>>, pp. 24-39.}

% Bibliografía.

\chapter*{\centering\mdseries\Large\textsc{Bibliografía}}
\addcontentsline{toc}{chapter}{\mdseries Bibliografía}

\begin{hangparas}{1cm}{1}

\noindent Aguilar García, Carolina Yeveth, <<Entre la verdad y la mentira. Control y censura inquisitorial en torno a las reliquias en la Nueva España>>, \emph{Letras Históricas}, núm. 7, otoño 2012, pp. 13-32. \\

\noindent Aguirre Rojas, Carlos Antonio, <<Indicios, lecturas indiciarias, estrategia indiciaria y saberes populares. Una hipótesis sobre los límites de la racionalidad burguesa moderna>> \emph{História \& Ensino}, vol. 13, núm. 1, set. 2007, pp. 09-44. \\

\noindent Alabarces, Pablo, <<¿De qué hablamos cuando hablamos de Deporte?>>, \emph{Nueva Sociedad}, núm. 154, marzo-abril 1998, pp. 74-86. \\

\noindent --- <<El deporte en América Latina>>, \emph{Razón y Palabra}, núm. 69, 2009, version digital en línea, fecha de consulta: 6 de mayo, 2016. \url{www.razonypalabra.org.mx} \\

\pagebreak

\noindent --- <<Deporte y sociedad en América Latina: un campo reciente, una agenda en construcción>>, \emph{Anales de Antropología}, vol. 49-1, 2015, pp. 11-28. \\

\noindent Alanís Enciso, Fernando, <<Los extranjeros en México, la inmigración y el gobierno:
¿Tolerancia o intolerancia religiosa?, 1821-1830>>, \emph{Historia Mexicana}, vol. 45, núm. 3, enero-marzo 1996, pp. 539, 566. \\

\noindent Alberro, Solange, <<El primer medio siglo de \emph{Historia Mexicana}>>, \emph{Historia Mexicana}, vol. \textsc{l}, núm. 4, abril-junio 2001, pp. 643-653. \\

\noindent Altuve, Eloy, <<Deporte ¿Fenómeno natural y eterno o creación socio-histórica?>> \emph{Espacio Abierto: cuaderno venezolano de Sociología}, vol. 18, núm. 1, 2009, pp. 7-23. \\

\noindent Anaya Merchant, Luis, <<La construcción de la memoria y la revisión de la Revolución>>,
\emph{Historia Mexicana}, vol. \textsc{xviv}, núm. 4, 1995, pp. 525-536. \\

\noindent Anderson, Benedict, \emph{Comunidades imaginadas, reflexiones sobre el origen y difusión del nacionalismo}, México, \textsc{fce}, 1993. \\

\noindent Angelotti, Gabriel, \emph{La dinámica del fútbol en México. La construcción de identidades colectivas en torno al Club de fútbol} Pachuca \emph{en nuestros días}, Tesis de Maestría, Colegio de Michoacán, 2004. \\

\noindent --- <<El origen del fútbol en México: narrativas tejidas en torno al primer club de fútbol y su trascendencia en la actualidad>>, s/e, 2007, pp. 1-23. \\

\noindent --- <<El estudio del fútbol ¿un ámbito periférico para la antropología en
México?>>, \emph{Revista de Antropología Experimental}, núm. 10, 2010, texto 12, pp. 211-222. \\

\noindent --- \emph{Chivas y tuzos íconos de México, identidades colectivas y capitalismo de compadres en el fútbol nacional}, México, Colegio de Michoacán, 2010. \\

\noindent Bairner, Alan, \emph{Sport, nationalism, and globalization: European and North American perspectives}, New York, \textsc{suny} Press, 2001. \\

\noindent Barrón, Luis, <<México: historia de un fútbol internacional. Una entrevista con Heriberto Murrieta>>, \emph{Istor}, año 15, núm. 57, 2014, pp. 93-100. \\

\noindent Bass, Amy, “State of the field: Sport History and the ‘cultural turn’”, \emph{The Journal of American History}, June 2014, 101 (1), pp. 148-172. \\

\noindent Beezley, William, <<El estilo porfiriano: deportes y diversiones de fin de siglo>>, \emph{Historia Mexicana}, vol. 33, núm. 2, (oct.-dic. 1983), pp. 265-284. \\

\noindent --- \emph{Judas at the Jockey Club and other episodes of Porfirian Mexico}, Lincoln and London, University of Nebraska, 1987. \\

\noindent --- \emph{La identidad nacional mexicana: la memoria, la insinuación y la cultura popular en el siglo \textsc{xix}}, México, El Colegio de la Frontera Norte, 2008. \\

\noindent Bernat Montesinos, Antonio, <<Estrategias de revisionismo histórico y pedagogía del odio>>,
\emph{Anuario de Pedagogía}, núm. 9, 2007, pp. 41-102. \\

\noindent Booth, Douglas, \emph{The field, truth and fiction in sport history}, London, Routledge, 2005. \\

\noindent --- “Theory”, en S. W. Pope, John Nauright (editors), \emph{Routledge companion to Sport History}, London, USA, Routledge, 2010, pp. 12-34. \\

\noindent Bourdieu, Pierre, \emph{La dominación masculina}, Anagrama, Barcelona, 2000. \\

\noindent Bloch, Marc, \emph{Apología para la historia o el oficio de historiador}, México, \textsc{fce}, 2001. \\

\noindent Brown, Matthew, (2015), “British informal empire and the origins of association football in South America”, \emph{Soccer \& Society}, 16:2-3, pp. 169-182. \\

\pagebreak

\noindent Burke, Peter, <<Obertura, la nueva historia, su pasado, su futuro>> en Peter Burke, \emph{Formas de hacer historia}, Barcelona, Alianza Editorial, 1996, pp. 11-37. \\

\noindent --- \emph{La Revolución historiográfica francesa. La Escuela de los Annales 1929-1984}, Barcelona, Gedisa, 1999. \\

\noindent Calderón, Carlos, \emph{Pachuca la cuna del fútbol}, México, Grupo Banta Imagen, 2001. \\

\noindent --- <<Orígenes del fútbol en México (\textsc{ii})>> \emph{Cuadernos del fútbol}, Centro de Investigaciones de Historia y Estadística del Fútbol Español (\textsc{cihefe}), núm. 43, mayo 2013, \url{http://www.cihefe.es/cuadernosdefutbol/} \\

\noindent --- <<Orígenes del fútbol en México (\textsc{iii})>>, \emph{Cuadernos de Fútbol}, Centro de Investigaciones de Historia y Estadística del Fútbol Español (\textsc{cihefe}), núm. 44, junio 2013, \url{http://www.cihefe.es/cuadernosdefutbol/} \\

\noindent --- <<El Pachuca Athletic Club no nació en 1900>>, \emph{Cuadernos de Fútbol}, Centro de Investigaciones de Historia y Estadística del Fútbol Español (\textsc{cihefe}), núm. 53, 1 de abril 2014, \url{http://www.cihefe.es/cuadernosdefutbol/} \\

\noindent Chaduneli, Besik, <<La evolución del rugby: de deporte violento a deporte regulado>>, \emph{Revista Ciencias de la Salud}, vol. 5 (2), 2007, pp. 116 121. \\

\noindent Camargo, Walter César, <<La construcción de la historiografía de la Revolución mexicana: críticas y nuevas perspectivas>>, \emph{Algarrobo-MEL.com.ar}, a2, n2, 2013, pp. 1-20. \\

\noindent Cit y Mulet, Juan, \emph{El libro de oro del fútbol mexicano}, México, Costa Amic, 1962. \\

\noindent Cho, Younghan, “Introduction”, \emph{Soccer \& Society}, 2013, vol. 14, núm. 5, pp. 579-587. \\

\noindent Collins, Tony, \emph{Rugby's great split. Class, culture and the origins of rugby league football}, London, New York, Routledge, 2006. \\

\noindent --- \emph{A social history of English rugby union}, UK, Routledge, 2009. \\

\noindent --- (2015), “Early football and the emergence of modern soccer, c. 1840-1880”,
\emph{The International Journal of the History of Sport}, vol. 32, núm. 9, pp. 1127-1142. \\

\noindent Costeloe, Michael, “To bowl a Mexican Maiden over: cricket in Mexico, 1827-1900”,
\emph{Bulletin of Latin American Research}, vol. 26, núm. 1, 2007, pp. 112-124. \\

\noindent Curry, Graham, “Playing for money: James J. Lang and emergent soccer professionalism in Sheffield”, \emph{Soccer \& Society}, vol. 5, núm. 3, Autumn 2004, pp. 336-355. \\

\pagebreak

\noindent --- and Dunning, Eric, “The problem with revisionism: how new data on the origins of modern football have led to hasty conclusions”, \emph{Soccer \& Society}, 2013, vol. 14, núm. 4, pp. 429-445. \\

\noindent --- (2014) “The origins of football debate: comments on Adrian Harvey’s historiography”, \emph{The International Journal of the History of Sport}, 31:17, pp. 2158-2163. \\

\noindent --- and Dunning, Eric, \emph{Association football. A study in figurational sociology}, London, New York, Routledge, 2015. \\

\noindent Dejonghe, Trudo, \emph{The popularity of football games in the world. Is there a relation with hegemonic power?} MA thesis, Lessius Hogeschool Antwerpen, 2007. \\

\noindent Dosse, Francois, <<La historia intelectual después del \emph{linguistic turn}>>, \emph{Historia y Grafía}, 2004, núm. 23, pp. 17-54. \\

\noindent Dunning, Eric, “Sport in space and time: ‘civilizing processes’, trajectories of state-formation and the development of modern sport” \emph{International Review for the Sociology of Sport}, 1994, vol. 29, núm. 4, pp. 331-345. \\

\noindent --- \emph{El fenómeno deportivo, estudios sociológicos en torno al deporte, la violencia y la civilización}, Barcelona, Paidotribo, 2003. \\

\noindent --- and Curry, Graham, “Public schools, status rivalry and the development of football”, en Eric Dunning, Dominic Malcolm, Ivan Waddington (editors), \emph{Sport Histories. Figurational studies of the development of modern sports}, London, New York, Routledge, 2004. \\

\noindent --- and Sheard, Kenneth, \emph{Barbarians, gentlemen and players. A sociological study of the development of rugby football}, London, New York, Routledge, 2005. \\

\noindent --- and Curry, Graham, \emph{Association football. A study in figurational sociology}, London, New York, Routledge, 2015. \\

\noindent Elias, Norbert, Dunning, Eric, \emph{Deporte y ocio en el proceso de la civilización}, México, textsc{fce}, 1986. \\

\noindent Esparza, Miguel, \emph{La nacionalización de los deportes en la ciudad de México, 1880-1928}, Tesis de Doctorado, Instituto Mora, 2014. \\

\noindent Espinoza, Silvina, <<La vida privada de los goles. Entrevista con Juan Villoro>>, \emph{Revista de la Universidad de México}, vol. 28, núm. 6, 2006, pp. 86-90. \\

\noindent Fábregas, Andrés, \emph{Lo Sagrado del Rebaño. El fútbol como integrador de identidades}, México, Colegio de Jalisco, 2001. \\

\noindent --- <<Identidades en juego: el fútbol en Jalisco>> en Luis Antonio González (compilador) \emph{Encuentros sociales y diversiones}, México, Secretaría de Cultura, Gobierno del Estado de Jalisco, 2005. \\

\noindent --- <<El fútbol en Chiapas (México): ¿un símbolo de identidad?>>, \emph{Revista de Dialectología y Tradiciones Populares}, julio-di\-ciembre 2006, vol. \textsc{lxi}, núm. 2, pp. 145-161. \\

\noindent --- <<Lo sagrado del rebaño: el nacimiento de un símbolo>>, \emph{Razón y Palabra}, núm. 69, 2009, versión digital en línea, fecha de consulta: 24 de abril, 2016. \url{www.razonypalabra.org.mx} \\

\noindent Findlay, Bill, “It’s a Dutch invention, but We started it in Scotland. The strange case of
Scottish football”, \emph{Études Écossaises}, 11, (2008), pp. 261-273. \\

Florescano, Enrique, \emph{Historia de las historias de la nación mexicana}, México, Taurus, 2002. \\

Gálvez, Javier, Stavrianeas, Statinos, <<El rugby en la Inglaterra del S. \textsc{xix}: ¿filosofía o
manipulación social?>>, \emph{Materiales para la Historia del Deporte}, No. 11, 2013, pp. 78-88. \\

Galeano, Eduardo, \emph{El fútbol a sol y sombra y otros escritos}, México, Siglo \textsc{xxi}, 2002. \\

\noindent Gaos, José <<Notas sobre la historiografía>>, \emph{Historia Mexicana}, vol. 9, núm. 4, abril 1960, pp. 481-508. \\

\noindent García Aguirre, Feliciano, <<Santa Gertrudis: una maquiladora del siglo pasado>>, \emph{Sotavento}, núm. 3, 1997, pp. 207-225. \\

\noindent Green, Geoffrey, \emph{The history of football association}, 4 vols. London, The Naldrett Press. 1961. \\

\noindent Ginzburg, Carlo, \emph{El queso y los gusanos. El cosmos según un molinero del siglo \textsc{xvi}}, Barcelona, Ediciones Península, 1976. \\

\noindent --- <<Indicios, raíces de un paradigma de inferencias indiciales>>, en Carlo Ginzburg, \emph{Mitos, emblemas, indicios. Morfología e historia}, Barcelona, Gedisa, 2008, pp. 185-239. \\

\noindent González, Luis, <<Silvio Zavala y el quehacer histórico en México>>, \emph{Historia Mexicana}, vol. 39, núm. 1, Homenaje a Silvio Zavala II (jul-sep, 1989), pp. 7-19. \\

\noindent González, Luis, \emph{El oficio de historiar}, México, Colegio de Michoacán, 1999. \\

\noindent Guerra, Francois Xavier, \emph{México del antiguo régimen a la Revolución}, México, \textsc{fce}, 1988. \\

\noindent Guoth, Nick, “Loss of identity: New Zealand soccer, its fundations and its legacies”, \emph{Soccer \& Society}, vol. 7, num. 2-3, april-july 2006, pp. 187-207. \\

\noindent Harvey, Adrian, (2001), “An epoch in the annals of national sport: football in Sheffield and the creation of modern soccer and rugby”, \emph{The International Journal of the History of Sport}, 18:4, pp. 53-87. \\

\noindent --- \emph{Football: the first hundred years. The untold story}, London, Routledge, 2005. \\

\noindent --- (2013) “The emergence of football in Nineteenth Century England: the historiographic debate”, \emph{The International Journal of the History of Sport}, 30: 18, pp. 2154- 2163. \\

\noindent Hay, Roy, (2010), “A tale of two footballs: the origins of Australian football and association football revisited”, \emph{Sport in Society}, 13:6, pp. 952-969. \\

\noindent Heath, Hilarie, <<Los primeros escarceos del imperialismo en México: las casas comerciales
británicas, 1821-1867>>, \emph{Historias}, 22, 1989, pp. 77-89. \\

\noindent Herrera González, Patricio, <<La sociedad salarial mexicana y su compleja integración social
en un contexto revolucionario>>, \emph{Relaciones}, 124, otoño 2010, vol. \textsc{xxxi}, pp. 125, 140. \\

\noindent Hill, Lloyd, (2010), “Football as code: the social diffusion of ‘soccer’ in South Africa”,
\emph{Soccer \& Society}, 11:1-2, pp. 12-28. \\

\noindent Holt, Richard, (2014), “Historians and the History of Sport”, \emph{Sport in History}, 34:1, pp. 133. \

\noindent --- \emph{Sport and the British: A modern history}, UK, Oxford University Press, 1989. \\

\noindent Huerta Rojas, Fernando, \emph{El juego del hombre. Deporte y masculinidades entre obreros}, México, Plaza y Valdés, \textsc{buap}, 1999. \\

\noindent Jarvie, G., Reid, I. A., “Sport, nationalism and culture in Scotland”, The Sport Historian, 19,1 (May 1999), pp. 97-124. \\

\noindent Jauretche, Arturo, \emph{Política nacional y revisionismo histórico}, Obras completas vol. 7, Buenos Aires, Corregidor, 2006, pp. 15-74. \\

\noindent Kassimeris, Christos, “The semiotics of European football”, \emph{Soccer \& Society}, 2014, Vol. 15, No. 2, pp. 190-202. \\

\noindent Keech, Marc, “England and Wales”, en James Riordan, Arnd Krüger (editors), \emph{European cultures of sport: Examining the nations and regions}, \textsc{usa}, Intellect Books, 2003. \\

\noindent Kitching, Gavin, (2011), “Old football and the new codes: some thoughts on the ‘origins of football’ debate and suggestions for further research”, \emph{The International Journal of the History of Sport}, 28:13, pp. 1733-1749. \\

\noindent Knight, Alan, <<Interpretaciones recientes de la Revolución mexicana>>, \emph{Secuencia}, 1989, 13, enero-abril, pp. 23-43. \\

\noindent --- <<Punto de vista revisionismo y Revolución: México comparado con Inglaterra y Francia>>, \emph{Boletín del Instituto de Historia Argentina y Americana, <<Dr. Emilio Ravignani>>}, núm. 10, 1994, pp. 91-127. \\

\pagebreak

\noindent Kuri, Ariel Rodríguez, <<Ganar la sede. La política internacional de los Juegos Olímpicos de 1968>>, \emph{Historia Mexicana}, vol. 64, núm. 1, (jul-sep 2014), pp. 243-289. \\

\noindent Lastra Lastra, José Manuel, <<El sindicalismo en México>>, \emph{Anuario Mexicano de Historia del Derecho}, 2002, vol. \textsc{xiv}, pp. 37-85. \\

\noindent Leese, Alex, (2015), “Illustrating the Auld Enemies: analysis of William Ralston’s depiction of the first international football match between Scotland and England”, \emph{Soccer \& Society}, 16:2-3, pp. 183-199. \\

\noindent Lever, Janet, \emph{La locura por el futbol}, México, \textsc{fce}, 1985. \\

\noindent Lewis, R. W., (2010), “Innovation not invention: A reply to Peter Swain regarding the professionalization of Association football in England and its diffusion”, \emph{Sport in History}, Vol. 30, No. 3, pp. 475-488. \\

\noindent Lóyzaga, Octavio, <<En torno a la jornada laboral>>, \emph{Alegatos}, núm. 58, septiembre-diciembre 2004, pp. 317-326. \\

\noindent Macías, Cesar Federico, <<El fútbol y el Bajío en la primera mitad del siglo \textsc{xx}>>, \emph{Razón y Palabra}, núm. 69, 2009, versión digital en línea, fecha de consulta: 24 de abril, 2016. \url{www.razonypalabra.org.mx} \\

\pagebreak

\noindent Matute, Álvaro, <<Orígenes del revisionismo historiográfico de la Revolución Mexicana>>,
\emph{Signos Históricos}, vol. 1, núm. 3, junio 2000, pp. 29-48. \\

\noindent --- <<Estudios de Historia Moderna y Contemporánea de México>>, \emph{Historia Mexicana}, vol. 50, núm. 4, abr-jun. 2001, pp. 779-789. \\

\noindent Mangan, J. A., “Missing men: schoolmasters and the early years of Association Football”,
\emph{Soccer \& Society}, Vol. 9, No. 2, April 2008, pp. 170-188. \\

\noindent --- and Hickey, C., “Pioneering further a field: beyond England”, \emph{Soccer \& Society}, Vol. 9, No. 5, December 2008, pp. 690-726. \\

\noindent Malcolm, Dominic, “Cricket civilizing and de-civilizing processes in the imperial game”, en Eric Dunning, Dominic Malcolm, Ivan Waddington, (editors), \emph{Sport histories. Figurational studies of the development of modern sports}, London, Routledge, 2004. \\

\noindent McGehee, Richard V., “Mexico and Central America”, en S. W. Pope, John Nauright
(editors), \emph{Routledge companion to Sport History}, London, \textsc{usa}, Routledge, 2010. \\

\noindent Meneses, Guillermo Alonso, Ávalos González, Juan Manuel, <<La investigación del fútbol y sus nexos con los estudios de comunicación. Aproximaciones y ejemplos>>, \emph{Comunicación y Sociedad}, núm. 2, jul-dic 2013, pp. 33-64. \\

\noindent Miganjos, Eduardo, <<La Revolución Mexicana y los nuevos enfoques historiográficos, entrevista con Gloria Villegas>>, \emph{Tzintzun. Revista de Estudios Históricos}, núm. 14, julio diciembre 1991, pp. 144-158. \\

\noindent Moreno Juárez, Sergio, \emph{Masculinidades en la ciudad de México, durante el porfiriato. Una aproximación bibliográfica}, \textsc{uam}, tesis de licenciatura, México, 2007. \\

\noindent Moreno Parada, Francisco, <<La investigación empírica en ciencias sociales>>, \emph{Cuaderno de Difusión Científica}, núm. 33, 1993, pp. 72-87. \\

\noindent Morrow, Don, “Canadian sport history: a critical essay”, \emph{Journal of Sport History}, Vol. 10, No. 1, (Spring, 1983), pp. 67-79. \\

\noindent Nauright, John, “Cornish miners and Witwatersrand gold mines in South Africa, c. 1890-1904”, \emph{Cornish History}, pp. 1-22. \\

\noindent Ovalle, Luis Carlos, \emph{Historia del fútbol en la ciudad de Aguascalientes. De los equipos románticos al sueño de un equipo profesional, 1910-1965}, Tesis de Maestría, Universidad Michoacana de San Nicolás Hidalgo, 2007. \\

\noindent Parra, Alma, <<Los orígenes de la industria eléctrica en México: las compañías británicas de
electricidad (1900-1929)>>, \emph{Historias}, núm. 19, octubre-marzo, 1988, pp. 139-158. \\

\noindent Pérez, Juan Antonio, <<Restablecimiento y consolidación de las relaciones entre México y Gran Bretaña durante el porfiriato (1884-1893)>>, \emph{Amicus Curiae}, vol. 1, núm. 3, enero-abril 2015, pp. 11-36. \\

\noindent Pérez-Rayón, Nora, <<La sociología de lo cotidiano: Discursos y fiestas cívicas en el México de 1900: La historia en la conformación de la identidad nacional>>, \emph{Revista Sociológica}, Año 8, Núm. 23, septiembre-diciembre, 1993, pp. 1-22. \\

\noindent Phillips, Murray G., “Deconstructing sport history: the postmodern challenge”, \emph{Journal of Sport History}, Vol. 28, No. 3, (Fall, 2001), pp. 327-343. \\

\noindent Pope, S. W., Nauright, John, “Introduction”, en S. W. Pope, John Nauright (editors),
\emph{Routledge companion to Sport History}, London, \textsc{usa}, Routledge, 2010, pp. 1-12. \\

\noindent Pope, S. W., \emph{Patriotic games: sporting traditions in the American imagination}, 1876-1926, \textsc{usa}, Oxford University Press, 1997. \\

\noindent Ramírez, Juan Rogelio, <<Lineamientos para un análisis de las identidades sociodeportivas en el fútbol>>, \emph{Sociológica}, año 26, núm. 73, mayo-agosto, 2011, pp. 153-181. \\

\noindent Ramírez, Carlos, \emph{¿Cuál es la historia al día del fútbol mexicano?} México, Editorial Novaro, 1960. \\

\noindent Ribera Carbó, Eulalia, <<Moviendo telares e iluminando la ciudad. De la industria local a la globalización empresarial en la electrificación de Orizaba, México, 1890-1919>>, \emph{Simposio Internacional Globalización, innovación y construcción de redes técnicas urbanas en América y Europa, 1890-1930, Barcelona}, 2012, pp. 1-33. \\

\noindent Riess, Steven A., “The new sport history”, \emph{Reviews in American history}, Vol. 18, No. 3, (Sep. 1990), pp. 311-325. \\

\noindent Rookwood, Joel, Buckley, Charles, “The significance of the Olympic soccer tournament
from 1908-1928”, \emph{Journal of Olympic History}, Vol. 15, No. 3, (November 2007), pp. 6-15. \\

\noindent Ruck, Rob, (2014), “The field of Sport History at critical mass”, \emph{The Journal of American History}, June, pp. 192-194. \\

\noindent Saavedra, Elvira, Sánchez, María, <<Minería y espacio en el distrito minero Pachuca-Real del
Monte en el siglo \textsc{xix}>>, \emph{Investigaciones Geográficas}, núm. 65, 2008, pp. 82-101. \\

\noindent Sánchez Menchero, Mauricio, <<Hacia una historia cultural de las diversiones públicas. Estudios culturales sobre el juego, la risa y el sobrecogimiento>>, \emph{Estudios sobre las Culturas Contemporáneas}, dic. vol. \textsc{xiii}, núm. 26, 2007, pp. 25-45. \\

\noindent San Miguel, Pedro L., <<Mito e historia en la épica campesina: John Womack y la revolución mexicana>>, \emph{Secuencia}, núm. 76, enero-abril 2010, pp. 133-156. \\

\pagebreak

\noindent Schell, William, \emph{Integral outsiders, the American colony in Mexico City, 1876-1911}, Scholarly Resources Inc. Wilmington Delaware, 2001. \\

\noindent Segura, Trejo, M., Fernando, <<Una pincelada de fútbol e historia>>, \emph{Istor}, año \textsc{xv}, núm. 57, 2014, pp. 3-8. \\

\noindent Serrano Álvarez, Pablo, <<Historiografía regional mexicana. Tendencias y enfoques metodológicos, 1968-1990>>, \emph{Relaciones}, 72, otoño 1997, vol. \textsc{xviii}, pp. 49-57. \\

\noindent Seward, Andy, “Cornish rugby and cultural identity: a socio-historical perspective”, \emph{The Sport Historian}, No. 18, 2 (Nov. 1998), pp. 78-94. \\

\noindent Simiyu Njororai, Wycliffe, “Colonial legacy, minorities and association football in Kenya”,
\emph{Soccer \& Society}, Vol. 10, No. 6, November 2009, pp. 866-882. \\

\noindent Solá, Jordi, <<Historia del rugby>>, \emph{Apunts}, 1999 (29), pp. 6-11. \\

\noindent Struna, Nancy L., “Social history and sport”, en Jay Coakley, Eric Dunning (editors),
\emph{Handbook of sport studies}, London, Sage, 2006, pp. 187-203. \\

\noindent Swain, Peter, Harvey, Adrian, (2012), “On Bosworth field or the playing fields of Eton and Rugby? Who really invented modern football?”, \emph{The International Journal of the History of Sport}, 29:10, pp. 1425-1445. \\

\noindent --- (2014) “The origins of football debate: the ‘Grander design and the involvement of the lower classes’, 1818-1840”, \emph{Sport in History}, 34:4, pp. 519-543. \\

\noindent --- (2014), “The origins of football debate: the continuing demise of the dominant paradigm, 1852-1856”, \emph{The International Journal of the History of Sport}, 31:17, pp. 2212- 2229. \\

\noindent --- (2015), “The origins of football debate: the evidence mounts, 1841-1851”, \emph{The International Journal of the History of Sport}, 32:2, pp. 299-317. \\

\noindent --- (2015), “The origins of football debate: football and cultural continuity, 1857- 1859”, \emph{The International Journal of the History of Sport}, 32:5, pp. 631-649. \\

\noindent Szymanski, Stefan, “A theory of the evolution of modern sport”, \emph{Journal of Sport History}, spring 2008, vol. 35, No. 1, pp. 1-32. \\

\noindent Szymanski, Stefan, Zimbalist, Andrew, \emph{National pastime: How Americans play baseball and the rest of the world play soccer}, \textsc{usa}, Brookings Institution Press, 2005. \\

\noindent Tranter, Neil, \emph{Sport, economy and society in Britain, 1750-1914}, London, Cambridge University Press, 1998. \\

\pagebreak

\noindent Tuck, Jason, Maguire, Joseph, “Making sense of global Patriotic Games: Rugby players, perceptions of national identity politics”, \emph{Football Studies}, No. 2, 1999, pp. 26-54. \\

\noindent Trouille, David, “Association football to fútbol: ethnic succession and the history of Chicago area, 1890-1920”, \emph{Soccer \& Society}, vol. 9, No. 4, octuber 2008, pp. 455-476. \\

\noindent Varela, Sergio, <<Goligarquías latinoamericanas. Fútbol profesional, poder público y el gran negocio mediático>>, \emph{Efdeportes}, año 12, núm. 111, agosto 2007, recuperado en \url{https://www.efdeportes.com/} \\

\noindent Vázquez, Josefina Zoraida, <<Historia Mexicana en el banquillo>>, \emph{Historia Mexicana}, vol. \textsc{xli}, núm. 1, 1991, pp. 11-23. \\

\noindent --- <<Cincuenta y tres años de las Memorias de la Academia Mexicana de la Historia>>, \emph{Historia Mexicana}, vol. \textsc{l}, núm. 4, abril-junio 2001, pp. 709-718. \\

\noindent Villegas Revueltas, Silvestre, <<Los intereses británicos en México y su nexo con la reforma
liberal>>, \emph{Jurídicas}, No. 253, 2010, pp. 337-353. \\

\noindent Villoro, Juan, \emph{Dios es redondo}, México, Anagrama, 2006. \\

\noindent Walton, J. K., (2012), “The origins of working class spectator sport: Lancashire, England, 1870-1914”, \emph{Historia y Comunicación Social}, vol. 17, pp. 125-140. \\

\noindent Walvin, James, \emph{The People's game: a social history of British football}, London, Allen Lane, 1975. \\

\noindent Ward, T., “Sport and national identity”, \emph{Soccer \& Society}, Vol. 10, No. 5, September 2009, pp. 518-531. \\

\noindent White, Andrew, “Rugby union in England civilizing processes and the de-institutionalization
of amateurism”, en Eric Dunning, Dominic Malcolm, Ivan Waddington, (editors), \emph{Sport histories. Figurational studies of the development of modern sports}, London, Routledge, 2004. \\

\noindent Wood, David, “Playing by the book: football in Latin American literature” \emph{Soccer \& Society}, 12:1, pp. 27-41. \\

\noindent Young, Percy, \emph{A history of British football}, London, Stanley Paul, 1968. \\

\noindent Zamora, Gerson, \emph{El equipo de fútbol Euzkadi en México, 1937-1939}, Tesis de licenciatura, 2010. \\

\noindent Zermeño, Guillermo, <<La historiografía en México: un balance (1940-2010)>>, \emph{Historia Mexicana}, vol. \textsc{xlii}, núm. 4, 2013, pp. 1695-1742. \\

\end{hangparas} % Finalizamos la sangría francesa

% Colofón

\newpage
\pagestyle{empty}
\null\vfill
\pdfbookmark{Colofón}{contents}

%\begin{center}
%\begin{minipage}{5cm}
%\footnotesize \textit{La fiesta del Sagrado Corazón de Jesús: un legado de la actividad devocional de la fábrica} La Estrella, \emph{Tlaxcala}, de Blanca Irma Alejo Aguilar, se terminó de maquetar en Cerrada de Colima {7301}, Col. Universidades, Puebla, México. La captura del texto se realizó con el editor de texto plano \href{https://www.xm1math.net/texmaker/}{\hologo{TeX}maker (\textsc{4.5})} y se diagramó, finalmente, en el sistema de composición tipográfica \href{https://www.latex-project.org/}{\hologo{LaTeXe}}.  En su formación se emplearon tipos \href{https://es.wikipedia.org/wiki/Linux_Libertine}{Linux Libertine} para el texto principal y {\href{http://libertine-fonts.org/}{\textsf{Linux Biolinum}}} para la portada; las imágenes y fotografías se manipularon en \textsc{\href{https://www.gimp.org/}{gimp}} ({2.8}). La impresión en \href{https://simple.wikipedia.org/wiki/Laser_printer}{\textit{laser printer}} estuvo bajo el cuidado del autor y se realizó en papel \emph{Bond} tamaño carta de 75\thinspace gr.
%\end{minipage}
%\end{center}
%\begin{center}
%\footnotesize Corrección ortográfica y estilo: \\ \href{julian2che@gmail.com}{Julián y Víctor Osorno}
%\end{center}
%\begin{center}
%\footnotesize Diseño y maquetación: \\ \href{noel_merino@yahoo.com.mx}{Noel Merino Hernández}
%\end{center}

\heartpar{\scriptsize \textit{Historia e historiografía del fútbol mexicano} de Miguel Ángel Esparza Ontiveros, se terminó de maquetar en Cerrada de Colima {7301}, Col. Universidades, Puebla, México. La captura del texto se realizó con el editor de texto plano \href{https://www.xm1math.net/texmaker/}{\hologo{TeX}maker (\textsc{4.5})} y se diagramó, finalmente, en el sistema de composición tipográfica \href{https://www.latex-project.org/}{\hologo{LaTeXe}}.  En su formación se emplearon tipos \href{https://es.wikipedia.org/wiki/Linux_Libertine}{Linux Libertine} para el texto principal y {\href{http://libertine-fonts.org/}{\textsf{Linux Biolinum}}} para la portada; las imágenes y fotografías se manipularon en \textsc{\href{https://www.gimp.org/}{gimp}} ({2.8}). La impresión estuvo bajo el cuidado del autor y se realizó en papel \href{https://en.wikipedia.org/wiki/Bond_paper}{\emph{Bond}} ahuesado \textsc{din\thinspace a5} de {75}\thinspace gr. Diseño y maquetación: \href{noel_merino@yahoo.com.mx}{Noel Merino Hernández}}

%\newpage
%\pdfbookmark{Contraportada}{Contraportada}
%\includepdf{03}

\end{document}